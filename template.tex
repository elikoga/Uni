\documentclass[a4paper,12pt]{article}
\usepackage[utf8]{inputenc}
\usepackage[ngerman]{babel}
\usepackage[top=1in, bottom=1.25in, left=1.25in, right=1.25in]{geometry}
\usepackage{minted}
\usepackage{blindtext}
\usepackage{fancyhdr}
\usepackage{titling}
\usepackage{amssymb}
\usepackage{mathtools}


\renewcommand{\footrulewidth}{0.4pt}

\setlength\headheight{15pt}
\setlength{\parskip}{1em}

\title{Document Template}
\author{Eli Kogan-Wang}
\date{\today}

\pagestyle{fancy}
\fancyhf{}
\lhead{\thetitle}
\rhead{\thedate}
\lfoot{\theauthor}
\rfoot{Page \thepage}


\begin{document}
% \maketitle
% \thispagestyle{fancy}
\renewcommand{\abstractname}{Abstract}
\begin{abstract}
  This is a LaTeX document template for writing university\footnote{
    I'm going to study at the Paderborn University} documents.
\end{abstract}
\section{Document}
\begin{minted}{nix}
{ ... }:
{
  asd = 12;
}
\end{minted}
\blinddocument
\blindmathpaper

\section{Content}

$\mathbb{N}$ is the set of natural numbers.
It can be defined in Haskell as a Peano Numeral by the following recursive definition:
\begin{minted}{haskell}
data PeanoNatural = Zero | Successor PeanoNatural
\end{minted}

From this definition, you can easily see how induction works.
Case analysis of a Natural numbers yields the following cases:

\begin{description}
  \item[\mintinline{haskell}{Zero}] This is the base case.
  \item[\mintinline{haskell}{Successor PeanoNatural}]
    This is the recursive case.
    The \mintinline{haskell}{PeanoNatural} passed as a parameter to the \mintinline{haskell}{Successor} constructor is structurally smaller.
\end{description}


In order to prove a statement for an arbitrary natural number,
you need to prove it for the base case,
as well as the recursive case with the assumption that the statement holds for the structurally smaller object.

This is called induction.
\end{document}
