\documentclass[a4paper,12pt]{article}
\usepackage[utf8]{inputenc}
\usepackage[ngerman]{babel}
\usepackage[top=1in, bottom=1.25in, left=1.25in, right=1.25in]{geometry}
\usepackage{minted}
\usepackage{blindtext}
\usepackage{fancyhdr}
\usepackage{titling}
\usepackage{amssymb}
\usepackage{mathtools}
\usepackage{hyperref}


\renewcommand{\footrulewidth}{0.4pt}

\setlength\headheight{15pt}
\setlength{\parskip}{1em}

\title{Document Template}
\author{Eli Kogan-Wang}
\date{\today}

\pagestyle{fancy}
\fancyhf{}
\lhead{\thetitle}
\rhead{\thedate}
\lfoot{\theauthor}
\rfoot{Page \thepage}


\begin{document}
% \maketitle
% \thispagestyle{fancy}
\renewcommand{\abstractname}{Abstract}
\begin{abstract}
    This is the notes in modelling for today.
\end{abstract}
\section{Document}

\paragraph*{Beweisen oder widerlegen}

\begin{itemize}
    \item Aussage, dass etwas immer gelten soll
    \item Aussage, dass etwas mindestens in einem Fall gelten soll.
\end{itemize}

\paragraph*{Fall 1}

Falls wahr, Beweis nötig (Induktion, )

Falls falsch, Gegenbeispiel reicht.

\paragraph*{Fall 2}

Falls wahr, Beispiel reicht.

Falls falsch, Gegenbeweis nötig.

\end{document}

