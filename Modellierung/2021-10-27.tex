\documentclass[a4paper,12pt]{article}
\usepackage[utf8]{inputenc}
\usepackage[ngerman]{babel}
\usepackage[top=1in, bottom=1.25in, left=1.25in, right=1.25in]{geometry}
\usepackage{minted}
\usepackage{blindtext}
\usepackage{fancyhdr}
\usepackage{titling}
\usepackage{amssymb}
\usepackage{mathtools}
\usepackage{hyperref}


\renewcommand{\footrulewidth}{0.4pt}

\setlength\headheight{15pt}
\setlength{\parskip}{1em}

\title{Document Template}
\author{Eli Kogan-Wang}
\date{\today}

\pagestyle{fancy}
\fancyhf{}
\lhead{\thetitle}
\rhead{\thedate}
\lfoot{\theauthor}
\rfoot{Page \thepage}


\begin{document}
% \maketitle
% \thispagestyle{fancy}
\renewcommand{\abstractname}{Abstract}
\begin{abstract}
    This is the notes in modelling for today.
\end{abstract}
%\section{Document}

\section{Modellierung Präsenzübung}

\paragraph{Aufgabe 1}

$R_a$, $R_b$, $R_c$ sind Relationen

(1) Reflexivität:

Nur die Relation $R_a$ ist reflexiv, weil für alle 2 Elemente $a,b$ der Relation
$R_a$ die Gleichung $a=b$ erfüllt.

(2) Irreflexivität:

Nur die Relation $R_b$ ist irreflexiv, weil keine 2 Elemente $a,b$ der Relation mit $a=b$ erfüllt sind.

(3) Symmetrie:

Nur die Relation $R_c$ ist symmetrisch, weil für alle 2 möglichen Elemente $(i,j)$, $(j,i)$,
beide genau dann, wenn in der Relation sind, wenn der jweilig
andere Element in der Relation ist.

(4) Asymmetrie:

Nur die Relation $R_b$ ist asymmetrisch, weil es kein Element $(a,b)$ der Relation $R_b$ gibt, zu dem ein Element $(b,a)$ in der Relation $R_b$ ist.

(5) Totalität:

Keine der Relationen $R_a$, $R_b$, $R_c$ ist total, weil bei keinem, für alle Elemente $(a,b)$, bei dem $a\neq b$, entweder $(a,b)$ oder $(b,a)$ in der Relation ist.



\end{document}

