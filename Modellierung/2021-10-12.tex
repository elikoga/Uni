\documentclass[a4paper,12pt]{article}
\usepackage[utf8]{inputenc}
\usepackage[ngerman]{babel}
\usepackage[top=1in, bottom=1.25in, left=1.25in, right=1.25in]{geometry}
\usepackage{minted}
\usepackage{blindtext}
\usepackage{fancyhdr}
\usepackage{titling}
\usepackage{amssymb}
\usepackage{mathtools}


\renewcommand{\footrulewidth}{0.4pt}

\setlength\headheight{15pt}
\setlength{\parskip}{1em}

\title{Document Template}
\author{Eli Kogan-Wang}
\date{\today}

\pagestyle{fancy}
\fancyhf{}
\lhead{\thetitle}
\rhead{\thedate}
\lfoot{\theauthor}
\rfoot{Page \thepage}


\begin{document}
% \maketitle
% \thispagestyle{fancy}
\renewcommand{\abstractname}{Abstract}
\begin{abstract}
  This is the notes in modelling for today.
\end{abstract}
\section{Document}

\paragraph*{Motivation der Modellierung}

Reale Sachverhältnisse in einen Kontext bringen, mit dem man Arbeiten kann.
Man bildet ein Modell mit dem Computer die Sachverhalte verarbeiten kann.

'\paragraph*{Modelle} sind Modelle

\paragraph*{Validierung von Modellen}

Für alle relevanten Operationen muss das Diagramm kommutieren, d.h.

$$opO(O)=I(opM(a(O)))$$

\paragraph*{Definition von Mengen}

Es gibt extensionale, intensionale und induktive Definitionen von Mengen.

\end{document}
