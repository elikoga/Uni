\documentclass[a4paper,12pt]{article}
\usepackage[utf8]{inputenc}
\usepackage[ngerman]{babel}
\usepackage[top=1in, bottom=1.25in, left=1.25in, right=1.25in]{geometry}
\usepackage{minted}
\usepackage{blindtext}
\usepackage{fancyhdr}
\usepackage{titling}
\usepackage{amssymb}
\usepackage{mathtools}
\usepackage{hyperref}


\renewcommand{\footrulewidth}{0.4pt}

\setlength\headheight{15pt}
\setlength{\parskip}{1em}

\title{Document Template}
\author{Eli Kogan-Wang}
\date{\today}

\pagestyle{fancy}
\fancyhf{}
\lhead{\thetitle}
\rhead{\thedate}
\lfoot{\theauthor}
\rfoot{Page \thepage}


\begin{document}
% \maketitle
% \thispagestyle{fancy}
\renewcommand{\abstractname}{Abstract}
\begin{abstract}
    This is the notes in modelling for today.
\end{abstract}
\section{Document}

\paragraph{SICP} Findet 21. Oktober 2021 14-19 Uhr bei der Zukunftsmeile 2 statt.

\paragraph{Graphen} als zweistellige Relationen

% itemized list
\begin{itemize}
    \item Definition
    \item Eigenschaften
\end{itemize}

\paragraph{Einfürung von Funktionen} als spezielle Relationen

\section{Neuer Vorlesungsteil}

\paragraph{Graphen} Die Folien fangen ca. bei \url{https://panda.uni-paderborn.de/pluginfile.php/2005623/mod_resource/content/2/part03-graphs-until37.pdf} an.


\end{document}
