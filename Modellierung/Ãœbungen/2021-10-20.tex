\documentclass[a4paper,12pt]{article}
\usepackage[utf8]{inputenc}
\usepackage[ngerman]{babel}
\usepackage[top=1in, bottom=1.25in, left=1.25in, right=1.25in]{geometry}
\usepackage{minted}
\usepackage{blindtext}
\usepackage{fancyhdr}
\usepackage{titling}
\usepackage{amssymb}
\usepackage{mathtools}
\usepackage{hyperref}


\renewcommand{\footrulewidth}{0.4pt}

\setlength\headheight{15pt}
\setlength{\parskip}{1em}

\title{Document Template}
\author{Eli Kogan-Wang}
\date{\today}

\pagestyle{fancy}
\fancyhf{}
\lhead{\thetitle}
\rhead{\thedate}
\lfoot{\theauthor}
\rfoot{Page \thepage}


\begin{document}
% \maketitle
% \thispagestyle{fancy}
\renewcommand{\abstractname}{Abstract}
\begin{abstract}
    This is the notes in modelling for today.
\end{abstract}
\section{Document}

\paragraph{Tutorin} Julia (grinjuk@mail.upb.de)

\paragraph{Aufgabe 1}

Voraussetzung zur Prüfungszulassung sind je $\ge$ 20\% der Punkte bei $\ge$ 9 (von 13) Übungszetteln.

\paragraph{Aufgabe 2}

\subparagraph{2.1}

Zu zeigen:

Für alle $n\in\mathbb{N}$ gilt:

$$1^2+2^2+3^2+\dots+n^2=\sum_{i=1}^n i^2=\frac{n(n+1)(2n+1)}{6}$$

I.A.

Für $n=1$ gilt:

$$\sum_{i=1}^n i^2=1^2=1=\frac{6}{6}=\frac{1(1+1)(2\cdot 1+1)}{6}$$

I.V.

Es sei ein $n\in\mathbb{N}$. Für dieses gilt:

$$\sum_{i=1}^n i^2=\frac{n(n+1)(2n+1)}{6}$$

I.S.

Zu zeigen:

$$\sum_{i=1}^{n+1} i^2=\frac{(n+1)((n+1)+1)(2(n+1)+1)}{6}$$


$$\begin{aligned}
        \sum_{i=1}^{n+1} i^2
         & =\sum_{i=1}^n i^2 + (n+1)^2                             \\
         & \underset{I.V.}{=}\frac{n(n+1)(2n+1)}{6}+(n+1)^2        \\
         & =\frac{n(n+1)(2n+1)}{6}+(n+1)^2                         \\
         & =\frac{n(n+1)(2n+1)}{6}+\frac{6\cdot(n+1)^2}{6}         \\
         & =\frac{n(n+1)(2n+1)+6\cdot(n+1)^2}{6}                   \\
         & =\frac{(n^2+1)(2n+1)+6\cdot(n^2+2n+1)}{6}               \\
         & =\frac{(n+1)(6(n+1)+n(2n+1))}{6}                        \\
         & =\frac{(n+1)(6n+6+2n^2+1n)}{6}                          \\
         & =\frac{(n+1)(2n^2+7n+6)}{6}                             \\
         & =\frac{(n+1)(n+2)(2n+3)}{6}                             \\
         &                                                  & \Box
    \end{aligned}$$

\subparagraph{2.2}

Zu zeigen:

Für alle $n\in\mathbb{N}_0$ gilt:

$$n^2+n\text{ ist gerade}$$

I.A.

Für $n=0$ gilt:

$$0^2+0=0$$

$0$ ist gerade, weil $\exists k\in\mathbb{N}_0: 2k=0$ mit $k=0$.

I.V.

Es sei ein $n\in\mathbb{N}_0$. Für dieses gilt:

$$n^2+n\text{ ist gerade}$$

Also $\exists k\in\mathbb{N}_0: 2k=n^2+n$. Wir nennen dieses $k$ $k_0$.

I.S.

Zu zeigen:

$$(n+1)^2+(n+1)\text{ ist gerade}$$

$$(n+1)^2+(n+1)=n^2+2n+1+n+1=n^2+n+2(n+1)\underset{I.V.}{=}2k_0+2(n+1)=2(k_0+n+1)$$

Nun ist $(n+1)^2+(n+1)$ gerade, weil $\exists k\in\mathbb{N}_0: 2k=(n+1)^2+(n+1)$ mit $k=k_0+n+1$.

$\Box$

\end{document}
