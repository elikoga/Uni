\documentclass[a4paper,12pt]{article}
\usepackage[utf8]{inputenc}
\usepackage[ngerman]{babel}
\usepackage[top=1in, bottom=1.25in, left=1.25in, right=1.25in]{geometry}
\usepackage{minted}
\usepackage{blindtext}
\usepackage{fancyhdr}
\usepackage{titling}
\usepackage{amssymb}
\usepackage{mathtools}
\usepackage{hyperref}


\renewcommand{\footrulewidth}{0.4pt}

\setlength\headheight{15pt}
\setlength{\parskip}{1em}

\title{Document Template}
\author{Eli Kogan-Wang}
\date{\today}

\pagestyle{fancy}
\fancyhf{}
\lhead{\thetitle}
\rhead{\thedate}
\lfoot{\theauthor}
\rfoot{Page \thepage}


\begin{document}
% \maketitle
% \thispagestyle{fancy}
\renewcommand{\abstractname}{Abstract}
\begin{abstract}
    This is the notes in modelling for today.
\end{abstract}
\section{Document}

\paragraph{Campus Consult} Praxiserfahrung

% itemized list
\begin{itemize}
    \item Consulting/Unternehmensberatung
    \item Arbeiten im Projektteam
    \item Praktische Arbeit
\end{itemize}

% Link to https://campus-consult.de/
% plain link
\url{https://campus-consult.de/}

\paragraph{Mengen} Anfangend Folie 4


% itemized list
\begin{itemize}
    \item Definition 2.1 ist der Mengenbegriff nach Cantor
    \item Heute wird mehr auf induktive Definitionen geachtet
    \item Def. der Natürlichen Zahlen mit Peano Axiomen ohne Namen eingeführt
    \item Defintion von Palindromen von §a§ und §n§
    \item Definition der Quadrate der natürlichen Zahlen
\end{itemize}

Induktive Definitionen sollen nicht immer bevorzugt werden.


% itemized list
\paragraph{Beweise} sind Beweise. Die wesentlichen Beweismethoden sind

\begin{itemize}
    \item Deduktion
    \item Widerlegung
    \item Induktion
\end{itemize}

\paragraph{Induktion} wird durch Induktion ausgeführt.\

\newpage

Beispiel eines Induktionsbeweises:

I.A. $n=1$ $n^2=1^2=1>1=n$

I.V. Für ein $n$ gilt: $n^2\ge n$

I.S. $n\rightsquigarrow n+1$. Zu zeigen: $(n+1)^2\ge n+1$

$$\begin{aligned}
        (n+1)^2=n^2+2n+1 & \underset{I.V.}{\ge} & n+2n+1 \\
                         & =                    & 3n+1   \\
                         & \ge                  & n+1    \\
                         & \quad                & \Box
    \end{aligned}$$

\paragraph{Strukturelle Induktion} wird eingeführt

Ein paar Operationen über Mengen werden eingeführt.

\end{document}
