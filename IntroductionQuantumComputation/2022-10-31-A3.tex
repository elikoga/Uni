\documentclass{article}
\usepackage{amsmath,amsthm}
\usepackage{amssymb,latexsym}
\usepackage{epsfig}
\usepackage{hyperref}
\usepackage{float}
\usepackage{fullpage}
\usepackage{enumerate}
\usepackage{paralist}
\usepackage{graphicx}
\usepackage{tikz}
\usetikzlibrary{quantikz}


\newtheorem{theorem}{Theorem}
\newtheorem{corollary}[theorem]{Corollary}
\newtheorem{lemma}[theorem]{Lemma}
\newtheorem{observation}[theorem]{Observation}
\newtheorem{proposition}[theorem]{Proposition}
\newtheorem{definition}[theorem]{Definition}
\newtheorem{claim}[theorem]{Claim}
\newtheorem{fact}[theorem]{Fact}
\newtheorem{assumption}[theorem]{Assumption}
\newtheorem{example}[theorem]{Example}
\newtheorem{conjecture}[theorem]{Conjecture}
\newtheorem{alg}[theorem]{Algorithm}
\newtheorem{protocol}[theorem]{Protocol}
\newtheorem{problem}[theorem]{Problem}

\newcommand{\ip}[2]{\left\langle #1 , #2\right\rangle}
\newcommand{\tr}{\trace}
\newcommand{\setft}[1]{\mathrm{#1}}
\newcommand{\lin}[1]{\setft{L}\left(#1\right)}
\newcommand{\density}[1]{\setft{D}\left(#1\right)}
\newcommand{\unitary}[1]{\setft{U}\left(#1\right)}
%\newcommand{\trans}[1]{\setft{T}\left(#1\right)}
\newcommand{\herm}[1]{\setft{Herm}\left(#1\right)}
\newcommand{\pos}[1]{\setft{Pos}\left(#1\right)}
\newcommand{\sep}[1]{\setft{Sep}\left(#1\right)}
\newcommand{\rank}[1]{\operatorname{rank}(#1)}
\newcommand{\ex}{\paragraph{Exercise.}}

\def\I{I}
\def\({\left(}
\def\){\right)}
\def\X{\mathcal{X}}
\def\Y{\mathcal{Y}}
\def\Z{\mathcal{Z}}
\def\W{\mathcal{W}}
\def\yes{\text{yes}}
\def\no{\text{no}}
\def\blog{\textup{log}}
\newcommand{\A}{\spa{A}}
\newcommand{\B}{\spa{B}}
\newcommand{\UA}{U_A}


\newcommand{\myparagraph}[1]{\paragraph{#1.}}

\newcommand{\eps}{\varepsilon}
\newcommand{\epssdp}{\varepsilon_{\rm sdp}}
% \newcommand{\bra}[1]{\langle #1|}
% \newcommand{\ket}[1]{|#1\rangle}
% \newcommand{\braket}[2]{\langle #1|#2\rangle}
% \newcommand{\ketbra}[2]{\ket{#1}{\bra{#2}}}
\newcommand{\ketbra}[2]{|#1\rangle\!\langle #2|}
\newcommand{\lmin}[1] {\lambda_{\operatorname{min}}(#1)}
\newcommand{\lmax}[1] {\lambda_{\operatorname{max}}(#1)}
\newcommand{\lhyes} {\operatorname{LH_{yes}}}
\newcommand{\lhno} {\operatorname{LH_{no}}}
\newcommand{\CQ}{\mathcal{CQ}}
\newcommand{\lh}{\operatorname{LH}}
\newcommand{\flh}{\operatorname{5-LH}}
\newcommand{\klhh}{\operatorname{k-LH}}
\newcommand{\qma}{\operatorname{QMA}}
\newcommand{\enc}[1]{\left<#1\right>}

\newcommand{\C}{C}
\newcommand{\Id}{Id} %CHECK
\newcommand{\Exs}[2]{E_{#1}[#2]} %CHECK

\newcommand{\beq}{\begin{equation}}
\newcommand{\eeq}{\end{equation}}

\newcommand{\trace}{{\rm Tr}}

%\newcommand{\dim}{\operatorname{dim}}
\newcommand{\norm}[1]{\left\|\,#1\,\right\|}       % norm
\newcommand{\pnorm}[1]{\left\|\,#1\,\right\|_p}       % norm
\newcommand{\onorm}[1]{\norm{#1}_{\mathrm{1}}}      % Euclidean norm for vectors
\newcommand{\enorm}[1]{\norm{#1}_{\mathrm{2}}}      % Euclidean norm for vectors
\newcommand{\trnorm}[1]{\norm{#1}_{\mathrm {tr}}}  % trace norm
\newcommand{\fnorm}[1]{\norm{#1}_{\mathrm {F}}}    % frobenius norm
\newcommand{\snorm}[1]{\norm{#1}_{\mathrm {\infty}}}    % spectral norm

\newcommand{\set}[1]{{\left\{#1\right\}}}    % braces for set notation
\newcommand{\ve}[1]{\mathbf{#1}}
\newcommand{\abs}[1]{\left\lvert #1 \right\rvert}
\newcommand{\optprod}{\OPT_P}
\newcommand{\opt}{\operatorname{OPT_1}}
\newcommand{\optt}{\operatorname{OPT_2}}
\newcommand{\newopt}{\operatorname{NEW-OPT}}
% \newcommand{\swap}{\operatorname{SWAP}}
\newcommand{\poly}{\operatorname{poly}}
\newcommand{\cc}{d^{\frac{k}{2}}}
\newcommand{\OPT}{{\rm OPT}}
\newcommand{\QMA}{{\rm QMA}}
\newcommand{\MQA}{{\rm MQA}}
\newcommand{\NP}{{\rm NP}}
\newcommand{\PP}{{\rm P}}
\newcommand{\PH}{{\rm PH}}
\newcommand{\BPP}{{\rm BPP}}
\newcommand{\BQP}{{\rm BQP}}
\newcommand{\TCSP}{{\rm 2-CSP}}

\newcommand{\CNOT}{{\rm CNOT}}

\newcommand{\complex}{{\mathbb C}}
\newcommand{\reals}{{\mathbb R}}
\newcommand{\ints}{{\mathbb Z}}
\newcommand{\nats}{{\mathbb N}}

\newcommand{\spa}[1]{\mathcal{#1}}
\newcommand{\dens}{\mathcal{D}(\spa{A}\otimes\spa{B})}
%\newcommand{\unitaries}{U(\spa{A}\otimes\spa{B})}

\newcommand{\LL}{\mathcal{L}}
\newcommand{\DD}{\mathcal{D}}
\newcommand{\HH}{\mathcal{H}}
\newcommand{\UU}{\mathcal{U}}

\newcommand{\klh}{MAX-$k$-local Hamiltonian}

\mathchardef\mhyphen="2D

\newcommand{\ayes}{A_{\rm yes}} %CHECK
\newcommand{\ano}{A_{\rm no}} %CHECK
\newcommand{\nl} {\mathcal{L}_1}


\bibliographystyle{alpha}

\begin{document}

\title{\vspace{-10mm}Introduction to Quantum Computation, UPB\\Winter 2022, Assignment 3\\{\large Eli Kogan-Wang (7251030, elikoga@mail.uni-paderborn.de)}}
\date{}
\maketitle

\section{Exercises}
\begin{enumerate}
  \item
        \begin{enumerate}
          \item Let $A\in\LL(\complex^d)$ be Hermitian. Prove that if for all $\ket{\psi}\in\complex^d$, $\bra{\psi}A\ket{\psi}\geq 0$, then $A$ has only non-negative eigenvalues. (Hint: Start by taking the spectral decomposition of $A$, and then make clever choices for $\ket{\psi}$.)

                Since $A$ is Hermitian, we have that $A=A^\dagger$ ($A$ is self-adjoint) and we have proven that all since $A$ is normal, $A$ is diagonizable/there exists a spectral decomposition of $A$.

                $$A=\sum_{i=1}^d \lambda_i \ketbra{\lambda_i}{\lambda_i}$$

                Where $\lambda_i$ are the eigenvalues of $A$ and $\ket{\lambda_i}$ are the corresponding eigenvectors such that $\{\ket{\lambda_i}\}$ is an orthonormal basis of $\complex^d$.

                Additionally, since $A$ is self-adjoint, we have that all eigenvalues $\lambda_i$ are real.

                Given $\bra{\psi}A\ket{\psi}\geq 0$, for all $\ket{\psi}\in\complex^d$, we can choose $\ket{\psi}=\ket{\lambda_i}$ for all $i$ and we have that:

                $$\bra{\lambda_i}A\ket{\lambda_i}\overset{\text{by calculation with spectr. decomp}}{=}\lambda_i\geq0$$

                Which is what we wanted to prove.

          \item Let $A\in\LL(\complex^d)$ be Hermitian. Prove that if $A$ has only non-negative eigenvalues, then for all $\ket{\psi}\in\complex^d$, $\bra{\psi}A\ket{\psi}\geq 0$. (Hint: Write $\ket{\psi}$ with respect to the eigenbasis of $A$.)

                Similiarly, let:

                $$A=\sum_{i=1}^d \lambda_i \ketbra{\lambda_i}{\lambda_i}$$

                Where $\lambda_i$ are the eigenvalues of $A$ and $\ket{\lambda_i}$ are the corresponding eigenvectors such that $\{\ket{\lambda_i}\}$ is an orthonormal basis of $\complex^d$.

                Now, since $\{\ket{\lambda_i}\}$ is an orthonormal basis of $\complex^d$, we have that:

                $$\ket{\psi}=\sum_{i=1}^d \mu_i \ket{\lambda_i}$$

                For some $\mu_i\in\complex$.

                And now:

                $$\begin{aligned}
                    \bra{\psi}A\ket{\psi} & =\sum_{i=1}^d \mu_i^\dagger \bra{\lambda_i}A\sum_{j=1}^d \mu_j \ket{\lambda_j} \\
                                          & =\sum_{i=1}^d \sum_{j=1}^d \mu_i^\dagger \mu_j \bra{\lambda_i}A\ket{\lambda_j} \\
                                          & =\sum_{i=1}^d \sum_{j=1}^d \mu_i^\dagger \mu_j \lambda_i \delta_{i,j}          \\
                                          & =\sum_{i=1}^d \mu_i^\dagger \mu_i \lambda_i                                    \\
                                          & =\sum_{i=1}^d \abs{\mu_i}^2 \lambda_i                                          \\
                  \end{aligned}$$

                And since $\abs{\mu_i}^2\geq0$ for all $i$, we have that $\bra{\psi}A\ket{\psi}\geq0$, for all $\ket{\psi}\in\complex^d$ as desired.

        \end{enumerate}
  \item Let $\ket{\psi}=\ket{-}\in\complex^2$. Suppose we measure in the $Z$ basis $B=\set{\ketbra{0}{0},\ketbra{1}{1}}$. What are the probabilities for each possible measurement outcome, and the corresponding post-measurement states?

        $$\ket{-}=\frac{1}{\sqrt{2}}\ket{0}-\frac{1}{\sqrt{2}}\ket{1}$$

        $$\begin{aligned}
            Pr(\text{outcome }\ket{0}: \ket{\psi}) & =\trace(\ketbra{0}{0}\ketbra{-}{-}\ketbra{0}{0})                         \\
                                                   & \overset{cyclic}{=}\trace(\ketbra{0}{0}\ketbra{0}{0}\ketbra{-}{-})       \\
                                                   & =\trace(\ketbra{0}{0}\ketbra{-}{-})                                      \\
                                                   & =\trace(\braket{0}{-}\braket{-}{0})                                      \\
                                                   & =\braket{0}{-}\braket{-}{0}                                              \\
                                                   & =\braket{0}{-}\braket{0}{-}^\dagger                                      \\
                                                   & =\abs{\braket{0}{-}}^2                                                   \\
                                                   & =\abs{\frac{1}{\sqrt{2}}\braket{0}{0}-\frac{1}{\sqrt{2}}\braket{0}{1}}^2 \\
                                                   & =\frac{1}{2}                                                             \\
          \end{aligned}$$

        Post measurement state if we measure $\ket{0}$:

        $$\begin{aligned}
            \ket{\psi'} & =\frac{\ket{0}\braket{0}{-}}{\sqrt{\frac{1}{2}}}      \\
                        & =\frac{\ket{0}\frac{1}{\sqrt{2}}}{\frac{1}{\sqrt{2}}} \\
                        & =\ket{0}                                              \\
          \end{aligned}$$

        $$\begin{aligned}
            Pr(\text{outcome }\ket{1}: \ket{\psi}) & =\trace(\ketbra{1}{1}\ketbra{-}{-}\ketbra{1}{1})                         \\
                                                   & \overset{cyclic}{=}\trace(\ketbra{1}{1}\ketbra{1}{1}\ketbra{-}{-})       \\
                                                   & =\trace(\ketbra{1}{1}\ketbra{-}{-})                                      \\
                                                   & =\trace(\braket{1}{-}\braket{-}{1})                                      \\
                                                   & =\braket{1}{-}\braket{-}{1}                                              \\
                                                   & =\braket{1}{-}\braket{1}{-}^\dagger                                      \\
                                                   & =\abs{\braket{1}{-}}^2                                                   \\
                                                   & =\abs{\frac{1}{\sqrt{2}}\braket{1}{0}-\frac{1}{\sqrt{2}}\braket{1}{1}}^2 \\
                                                   & =\frac{1}{2}                                                             \\
          \end{aligned}$$

        Post measurement state if we measure $\ket{1}$:

        $$\begin{aligned}
            \ket{\psi'} & =\frac{\ket{1}\braket{1}{-}}{\sqrt{\frac{1}{2}}}       \\
                        & =\frac{-\ket{1}\frac{1}{\sqrt{2}}}{\frac{1}{\sqrt{2}}} \\
                        & =-\ket{1}                                              \\
          \end{aligned}$$

  \item Consider the teleportation protocol we saw in class. Does it still work if we replace the use of the entangled Bell state $\ket{\phi^+}$ with the unentangled state $\ket{00}$ (i.e. Alice and Bob share the state $\ket{00}$)? How about if we use $\sqrt{2/5}\ket{00}+\sqrt{3/5}\ket{11}$ instead of $\ket{\phi^+}$?

        In class we saw for $\ket{\phi}=\alpha\ket{0}+\beta\ket{1}$ the following circuit.

        \begin{quantikz}
          \lstick{$\ket{\psi}$}\slice{0}&\ctrl{1}\slice{1} &\gate{H}\slice{2} &\meter{}\slice{3} &\cw        &\cwbend{2} &\cw \\
          \lstick[wires=2]{$\ket{\phi}$}&\targ{}  &\qw      &\meter{} &\cwbend{1} &\cw        &\cw \\
          &\qw      &\qw      &\qw      &\gate{X}   &\gate{Z}   &\qw\rstick{$\ket{\psi'}$}
        \end{quantikz}

        For $\ket{\phi}=\ket{\phi^+}=\frac{1}{\sqrt{2}}\ket{00}+\frac{1}{\sqrt{2}}\ket{11}$ we have analized the circuit and determined
        that $\ket{\psi'}=\ket{\psi}$.

        Now consider $\ket{\phi}=\ket{00}$, then we have

        $$\begin{aligned}
            \ket{\phi_0} & =\ket{\psi}\otimes\ket{00}                                \\
                         & =(\alpha\ket{0}+\beta\ket{1})\otimes\ket{00}              \\
                         & =\alpha\ket{0}\otimes\ket{00}+\beta\ket{1}\otimes\ket{00} \\
                         & =\alpha\ket{000}+\beta\ket{100}                           \\
                         & =(\alpha\ket{00}+\beta\ket{10})\otimes\ket{0}             \\
          \end{aligned}$$

        $$\begin{aligned}
            \ket{\phi_1} & =(\CNOT\otimes I)\ket{\phi_0}                             \\
                         & =\alpha\ket{00}\otimes\ket{0}+\beta\ket{11}\otimes\ket{0} \\
                         & =\alpha\ket{000}+\beta\ket{110}                           \\
                         & =\alpha\ket{0}\ket{00}+\beta\ket{1}\ket{10}               \\
          \end{aligned}$$

        $$\begin{aligned}
            \ket{\phi_2} & =(H\otimes I\otimes I)\ket{\phi_1}                             \\
                         & =\alpha\ket{+}\ket{00}+\beta\ket{-}\ket{10}                    \\
                         & =\alpha\ket{000}+\alpha\ket{100}+\beta\ket{010}-\beta\ket{110} \\
          \end{aligned}$$

        Alice measures $B=\{\ketbra{00}{00}\otimes I,\ketbra{01}{01}\otimes I,\ketbra{10}{10}\otimes I,\ketbra{11}{11}\otimes I\}$


        For outcome $\ket{00}$ we have $\Pr=\abs{\alpha}^2$ and $\ket{\psi'}=\frac{\alpha}{\abs{\alpha}}\ket{0}$.

        For outcome $\ket{01}$ we have $\Pr=\abs{\beta}^2$ and $\ket{\psi'}=\frac{\beta}{\abs{\beta}}\ket{0}$.

        For outcome $\ket{10}$ we have $\Pr=\abs{\alpha}^2$ and $\ket{\psi'}=\frac{\alpha}{\abs{\alpha}}\ket{0}$.

        For outcome $\ket{11}$ we have $\Pr=\abs{\beta}^2$ and $\ket{\psi'}=\frac{-\beta}{\abs{\beta}}\ket{0}$.


        We have transmitted no information and have destroyed our entangled pair.

        \vspace{0.5cm}\hrule\vspace{0.5cm}

        Now consider $\ket{\phi}=\sqrt{2/5}\ket{00}+\sqrt{3/5}\ket{11}$, then we have

        $$\begin{aligned}
            \ket{\phi_0} & =\ket{\psi}\otimes\ket{\phi}                                                                             \\
                         & =(\alpha\ket{0}+\beta\ket{1})\otimes(\sqrt{2/5}\ket{00}+\sqrt{3/5}\ket{11})                              \\
                         & =(\alpha\sqrt{2/5}\ket{000}+\alpha\sqrt{3/5}\ket{011}+\beta\sqrt{2/5}\ket{100}+\beta\sqrt{3/5}\ket{111}) \\
          \end{aligned}$$

        $$\begin{aligned}
            \ket{\phi_1} & =(\CNOT\otimes I)\ket{\phi_0}                                                                          \\
                         & =\alpha\sqrt{2/5}\ket{000}+\alpha\sqrt{3/5}\ket{111}+\beta\sqrt{2/5}\ket{110}+\beta\sqrt{3/5}\ket{101} \\
          \end{aligned}$$

        $$\begin{aligned}
            \ket{\phi_2} & =(H\otimes I\otimes I)\ket{\phi_1}                                                                                             \\
                         & =\alpha\sqrt{2/5}\ket{+}\ket{00}+\alpha\sqrt{3/5}\ket{+}\ket{11}+\beta\sqrt{2/5}\ket{-}\ket{10}+\beta\sqrt{3/5}\ket{-}\ket{01} \\
                         & =\frac{1}{\sqrt{2}}                                                                                                            \\
                         & \cdot(\alpha\sqrt{2/5}(\ket{0}+\ket{1})\ket{00}                                                                                \\
                         & +\alpha\sqrt{3/5}(\ket{0}+\ket{1})\ket{11}                                                                                     \\
                         & +\beta\sqrt{2/5}(\ket{0}-\ket{1})\ket{10}                                                                                      \\
                         & +\beta\sqrt{3/5}(\ket{0}-\ket{1})\ket{01})                                                                                     \\                                                                                                \\
                         & =\frac{1}{\sqrt{2}}                                                                                                            \\
                         & \cdot(\ket{00}(\alpha\sqrt{2/5}\ket{0}+\beta\sqrt{3/5}\ket{1})                                                                 \\
                         & +\ket{01}(\alpha\sqrt{3/5}\ket{1}+\beta\sqrt{2/5}\ket{0})                                                                      \\
                         & +\ket{10}(\alpha\sqrt{2/5}\ket{0}-\beta\sqrt{3/5}\ket{1})                                                                      \\
                         & +\ket{11}(\alpha\sqrt{3/5}\ket{1}-\beta\sqrt{2/5}\ket{0}))                                                                     \\
                         & =\frac{1}{\sqrt{2}}                                                                                                            \\
                         & \cdot(\alpha\sqrt{2/5}\ket{000}+\beta\sqrt{3/5}\ket{001}                                                                       \\
                         & +\alpha\sqrt{3/5}\ket{011}+\beta\sqrt{2/5}\ket{010}                                                                            \\
                         & -\alpha\sqrt{2/5}\ket{100}-\beta\sqrt{3/5}\ket{101}                                                                            \\
                         & -\alpha\sqrt{3/5}\ket{111}-\beta\sqrt{2/5}\ket{110})                                                                           \\
          \end{aligned}$$

        We shall consider the measurement outcomes:

        For $\ket{00}$ we have $$\begin{aligned}
            \enorm{(\ketbra{00}{00}\otimes I)\ket{\phi_2}}^2 & =\enorm{\frac{1}{\sqrt{2}}\cdot\alpha\cdot\sqrt{2/5}\ket{000}+\frac{1}{\sqrt{2}}\cdot\beta\cdot\sqrt{3/5}\ket{001}}^2 \\
                                                             & =\frac{1}{2}\cdot\abs{\alpha}^2\cdot\frac{2}{5}+\frac{1}{2}\cdot\abs{\beta}^2\cdot\frac{3}{5}                         \\
          \end{aligned}$$

        And the new state is $$\begin{aligned}
              & \frac{\frac{1}{\sqrt{2}}\cdot\alpha\cdot\sqrt{2/5}\ket{000}+\frac{1}{\sqrt{2}}\cdot\beta\cdot\sqrt{3/5}\ket{001}}{\sqrt{\frac{1}{2}\cdot\abs{\alpha}^2\cdot\frac{2}{5}+\frac{1}{2}\cdot\abs{\beta}^2\cdot\frac{3}{5}}} \\
            = & \frac{1}{\sqrt{2}}\cdot\frac{\alpha\cdot\sqrt{2/5}\ket{000}+\beta\cdot\sqrt{3/5}\ket{001}}{\sqrt{\frac{1}{2}\cdot\abs{\alpha}^2\cdot\frac{2}{5}+\frac{1}{2}\cdot\abs{\beta}^2\cdot\frac{3}{5}}}                        \\
            = & \frac{\alpha\cdot\sqrt{2/5}\ket{000}+\beta\cdot\sqrt{3/5}\ket{001}}{\sqrt{\abs{\alpha}^2\cdot\frac{2}{5}+\abs{\beta}^2\cdot\frac{3}{5}}}                                                                               \\
          \end{aligned}$$

        The other measurement outcomes behave similarly. We will write them out in their un-normalized form:

        As before, we may apply the $X$-gate, controlled by the measurement of the second qubit, to the third qubit.

        The new state for the third qubit is

        Outcome $\ket{00}$: $\alpha\sqrt{2/5}\ket{0}+\beta\sqrt{3/5}\ket{1}$

        Outcome $\ket{01}$: $\alpha\sqrt{3/5}\ket{0}+\beta\sqrt{2/5}\ket{1}$

        Outcome $\ket{10}$: $\alpha\sqrt{2/5}\ket{0}-\beta\sqrt{3/5}\ket{1}$

        Outcome $\ket{11}$: $\alpha\sqrt{3/5}\ket{0}-\beta\sqrt{2/5}\ket{1}$

        As before, we now apply the $Z$-gate, controlled by the measurement of the first qubit, to the third qubit.

        The new state for the third qubit is

        Outcome $\ket{00}$: $\alpha\sqrt{2/5}\ket{0}+\beta\sqrt{3/5}\ket{1}$

        Outcome $\ket{01}$: $\alpha\sqrt{3/5}\ket{0}+\beta\sqrt{2/5}\ket{1}$

        Outcome $\ket{10}$: $\alpha\sqrt{2/5}\ket{0}+\beta\sqrt{3/5}\ket{1}$

        Outcome $\ket{11}$: $\alpha\sqrt{3/5}\ket{0}+\beta\sqrt{2/5}\ket{1}$

        Our result is a state, that is strongly correlated with the original state $\ket{\psi}$.

        Since the outcome vectors are normalized, I don't believe that the result is a linear tranformation
        of the original state. It may or may not be difficult to recover the original state from the result.
\end{enumerate}

\end{document}
