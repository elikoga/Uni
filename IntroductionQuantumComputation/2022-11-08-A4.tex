\documentclass{article}
\usepackage{amsmath,amsthm}
\usepackage{amssymb,latexsym}
\usepackage{epsfig}
\usepackage{hyperref}
\usepackage{float}
\usepackage{fullpage}
\usepackage{enumerate}
\usepackage{paralist}
\usepackage{graphicx}
\usepackage{tikz}
\usetikzlibrary{quantikz}


\newtheorem{theorem}{Theorem}
\newtheorem{corollary}[theorem]{Corollary}
\newtheorem{lemma}[theorem]{Lemma}
\newtheorem{observation}[theorem]{Observation}
\newtheorem{proposition}[theorem]{Proposition}
\newtheorem{definition}[theorem]{Definition}
\newtheorem{claim}[theorem]{Claim}
\newtheorem{fact}[theorem]{Fact}
\newtheorem{assumption}[theorem]{Assumption}
\newtheorem{example}[theorem]{Example}
\newtheorem{conjecture}[theorem]{Conjecture}
\newtheorem{alg}[theorem]{Algorithm}
\newtheorem{protocol}[theorem]{Protocol}
\newtheorem{problem}[theorem]{Problem}

\newcommand{\ip}[2]{\left\langle #1 , #2\right\rangle}
\newcommand{\tr}{\trace}
\newcommand{\setft}[1]{\mathrm{#1}}
\newcommand{\lin}[1]{\setft{L}\left(#1\right)}
\newcommand{\density}[1]{\setft{D}\left(#1\right)}
\newcommand{\unitary}[1]{\setft{U}\left(#1\right)}
%\newcommand{\trans}[1]{\setft{T}\left(#1\right)}
\newcommand{\herm}[1]{\setft{Herm}\left(#1\right)}
\newcommand{\pos}[1]{\setft{Pos}\left(#1\right)}
\newcommand{\sep}[1]{\setft{Sep}\left(#1\right)}
\newcommand{\rank}[1]{\operatorname{rank}(#1)}
\newcommand{\ex}{\paragraph{Exercise.}}

\def\I{I}
\def\({\left(}
\def\){\right)}
\def\X{\mathcal{X}}
\def\Y{\mathcal{Y}}
\def\Z{\mathcal{Z}}
\def\W{\mathcal{W}}
\def\yes{\text{yes}}
\def\no{\text{no}}
\def\blog{\textup{log}}
\newcommand{\A}{\spa{A}}
\newcommand{\B}{\spa{B}}
\newcommand{\UA}{U_A}


\newcommand{\myparagraph}[1]{\paragraph{#1.}}

\newcommand{\eps}{\varepsilon}
\newcommand{\epssdp}{\varepsilon_{\rm sdp}}
% \newcommand{\bra}[1]{\langle #1|}
% \newcommand{\ket}[1]{|#1\rangle}
% \newcommand{\braket}[2]{\langle #1|#2\rangle}
% \newcommand{\ketbra}[2]{\ket{#1}{\bra{#2}}}
\newcommand{\ketbra}[2]{|#1\rangle\!\langle #2|}
\newcommand{\lmin}[1] {\lambda_{\operatorname{min}}(#1)}
\newcommand{\lmax}[1] {\lambda_{\operatorname{max}}(#1)}
\newcommand{\lhyes} {\operatorname{LH_{yes}}}
\newcommand{\lhno} {\operatorname{LH_{no}}}
\newcommand{\CQ}{\mathcal{CQ}}
\newcommand{\lh}{\operatorname{LH}}
\newcommand{\flh}{\operatorname{5-LH}}
\newcommand{\klhh}{\operatorname{k-LH}}
\newcommand{\qma}{\operatorname{QMA}}
\newcommand{\enc}[1]{\left<#1\right>}

\newcommand{\C}{C}
\newcommand{\Id}{Id} %CHECK
\newcommand{\Exs}[2]{E_{#1}[#2]} %CHECK

\newcommand{\beq}{\begin{equation}}
\newcommand{\eeq}{\end{equation}}

\newcommand{\trace}{{\rm Tr}}

%\newcommand{\dim}{\operatorname{dim}}
\newcommand{\norm}[1]{\left\|\,#1\,\right\|}       % norm
\newcommand{\pnorm}[1]{\left\|\,#1\,\right\|_p}       % norm
\newcommand{\onorm}[1]{\norm{#1}_{\mathrm{1}}}      % Euclidean norm for vectors
\newcommand{\enorm}[1]{\norm{#1}_{\mathrm{2}}}      % Euclidean norm for vectors
\newcommand{\trnorm}[1]{\norm{#1}_{\mathrm {tr}}}  % trace norm
\newcommand{\fnorm}[1]{\norm{#1}_{\mathrm {F}}}    % frobenius norm
\newcommand{\snorm}[1]{\norm{#1}_{\mathrm {\infty}}}    % spectral norm

\newcommand{\set}[1]{{\left\{#1\right\}}}    % braces for set notation
\newcommand{\ve}[1]{\mathbf{#1}}
\newcommand{\abs}[1]{\left\lvert #1 \right\rvert}
\newcommand{\optprod}{\OPT_P}
\newcommand{\opt}{\operatorname{OPT_1}}
\newcommand{\optt}{\operatorname{OPT_2}}
\newcommand{\newopt}{\operatorname{NEW-OPT}}
% \newcommand{\swap}{\operatorname{SWAP}}
\newcommand{\poly}{\operatorname{poly}}
\newcommand{\cc}{d^{\frac{k}{2}}}
\newcommand{\OPT}{{\rm OPT}}
\newcommand{\QMA}{{\rm QMA}}
\newcommand{\MQA}{{\rm MQA}}
\newcommand{\NP}{{\rm NP}}
\newcommand{\PP}{{\rm P}}
\newcommand{\PH}{{\rm PH}}
\newcommand{\BPP}{{\rm BPP}}
\newcommand{\BQP}{{\rm BQP}}
\newcommand{\TCSP}{{\rm 2-CSP}}

\newcommand{\CNOT}{{\rm CNOT}}

\newcommand{\complex}{{\mathbb C}}
\newcommand{\reals}{{\mathbb R}}
\newcommand{\ints}{{\mathbb Z}}
\newcommand{\nats}{{\mathbb N}}

\newcommand{\spa}[1]{\mathcal{#1}}
\newcommand{\dens}{\mathcal{D}(\spa{A}\otimes\spa{B})}
%\newcommand{\unitaries}{U(\spa{A}\otimes\spa{B})}

\newcommand{\LL}{\mathcal{L}}
\newcommand{\DD}{\mathcal{D}}
\newcommand{\HH}{\mathcal{H}}
\newcommand{\UU}{\mathcal{U}}

\newcommand{\klh}{MAX-$k$-local Hamiltonian}

\mathchardef\mhyphen="2D

\newcommand{\ayes}{A_{\rm yes}} %CHECK
\newcommand{\ano}{A_{\rm no}} %CHECK
\newcommand{\nl} {\mathcal{L}_1}


\bibliographystyle{alpha}

\begin{document}

\title{\vspace{-10mm}Introduction to Quantum Computation, UPB\\Winter 2022, Assignment 4\\{\large Eli Kogan-Wang}}
\date{}
\maketitle

\section{Exercises}
\begin{enumerate}
  \item
        \begin{enumerate}
          \item  Let $A,B\in\LL(\complex^d)$ be positive semi-definite matrices. Prove that $A+B$ is positive semi-definite.

                Since $A,B$ positive semi-definite: $\forall \ket{\psi}\in\complex^d$ we have $\bra{\psi}A\ket{\psi}\geq 0$ and $\bra{\psi}B\ket{\psi}\geq 0$.

                Now: $$\begin{aligned}
                    \bra{\psi}(A+B)\ket{\psi} & = \bra{\psi}A\ket{\psi} + \bra{\psi}B\ket{\psi} \\
                                              & \geq 0 + 0                                      \\
                                              & = 0
                  \end{aligned}$$

                By linearity of the Vector space $\LL(\complex^d)$, we have that $A+B$ is positive semi-definite.

          \item  Prove that if $\rho$ and $\sigma $ density matrices, then so is $p_1\rho+p_2\sigma$ for any $p_1,p_2\geq 0$ and $p_1+p_2=1$.

                Let $\rho$ and $\sigma$ be density matrices. Then $\rho$ and $\sigma$ are positive semi-definite hermitian matrices.
                Additionally, $\trace(\rho)=\trace(\sigma)=1$.

                Now, let $p_1,p_2\geq 0$ and $p_1+p_2=1$. Then, we have the following:

                Positive semi-definiteness:
                $$\forall \ket{\psi}\in\complex^d,
                  \bra{\psi}(p_1\rho+p_2\sigma)\ket{\psi}
                  =p_1\bra{\psi}\rho\ket{\psi}+p_2\bra{\psi}\sigma\ket{\psi}\geq p_10+p_20=0$$

                Hermitian:
                $$\begin{aligned}\forall \ket{\psi}\in\complex^d,
                     & \bra{\psi}(p_1\rho+p_2\sigma)\ket{\psi}^\dagger                            \\
                     & =p_1\bra{\psi}\rho\ket{\psi}^\dagger+p_2\bra{\psi}\sigma\ket{\psi}^\dagger \\
                     & =p_1\bra{\psi}\rho^\dagger\ket{\psi}+p_2\bra{\psi}\sigma^\dagger\ket{\psi} \\
                     & =p_1\bra{\psi}\rho\ket{\psi}+p_2\bra{\psi}\sigma\ket{\psi}                 \\
                     & =\bra{\psi}(p_1\rho+p_2\sigma)\ket{\psi}                                   \\
                  \end{aligned}
                $$

                Trace:
                $$\begin{aligned}
                    \trace(p_1\rho+p_2\sigma) & =p_1\trace(\rho)+p_2\trace(\sigma) \\
                                              & =p_11+p_21                         \\
                                              & =p_1+p_2                           \\
                                              & =1
                  \end{aligned}$$
        \end{enumerate}
  \item
        Suppose that with probability $1/3$, I give you state $\ket{0}\in\complex^2$, and with probability $2/3$, I give you state $\ket{-}$. Write down (i.e. as a $2\times 2$ matrix) the density matrix describing the state in your possession.

        Let $\ketbra{0}{0}$ be the density matrix for $\ket{0}$ and $\ketbra{-}{-}$ be the density matrix for $\ket{-}$.

        We will write out $\ketbra{-}{-}$ in terms of the computational basis $\{\ket{0},\ket{1}\}$:

        $$\ketbra{-}{-}=\frac{1}{2}\left(\ketbra{0}{0}-\ketbra{0}{1}-\ketbra{1}{0}+\ketbra{1}{1}\right)$$

        Then, the density matrix for the state in my possession is:

        $$\begin{aligned}
            \frac{1}{3}\ketbra{0}{0}+\frac{2}{3}\ketbra{-}{-} & =\frac{1}{3}\begin{bmatrix}1&0\\0&0\end{bmatrix}+\frac{2}{3}\cdot\frac{1}{2}\begin{bmatrix}1&-1\\-1&1\end{bmatrix} \\
                                                              & =\frac{1}{3}\begin{bmatrix}1&0\\0&0\end{bmatrix}+\frac{1}{3}\begin{bmatrix}1&-1\\-1&1\end{bmatrix}                 \\
                                                              & =\frac{1}{3}\begin{bmatrix}2&-1\\-1&1\end{bmatrix}                                                                 \\
                                                              & =\begin{bmatrix}\frac23&\frac{-1}3\\\frac{-1}3&\frac13\end{bmatrix}                                                \\
          \end{aligned}$$
  \item  Define bipartite state $\ket{\psi}=\alpha\ket{01}-\beta\ket{10}$. Let $\rho=\frac{1}{2}\ketbra{\Phi^+}{\Phi^+} + \frac{1}{2}\ketbra{\psi}{\psi}$. Compute $\trace_B(\rho)$.

        Remember that $\ket{\Phi^+}=\frac{1}{\sqrt{2}}(\ket{00}+\ket{11})$.

        And that $\trace_B(\rho)=\sum_{i=1}^{d_B}(\I_A\otimes\bra{i})\rho(\I_A\otimes\ket{i})$.

        Further we will inspect the density operator $\ketbra{\Phi^+}{\Phi^+}$:

        $$\begin{aligned}
            \ketbra{\Phi^+}{\Phi^+} & = \frac{1}{2}(\ket{00}+\ket{11})(\bra{00}+\bra{11})                                           \\
                                    & = \frac{1}{2}(\ketbra{00}{00}+\ketbra{00}{11}+\ketbra{11}{00}+\ketbra{11}{11})                \\
                                    & = \frac12\ketbra{00}{00}+\frac12\ketbra{00}{11}+\frac12\ketbra{11}{00}+\frac12\ketbra{11}{11} \\
          \end{aligned}$$

        And additionally, we will inspect the density operator $\ketbra{\psi}{\psi}$:

        $$\begin{aligned}
            \ketbra{\psi}{\psi} & = (\alpha\ket{01}-\beta\ket{10})(\alpha\bra{01}-\beta\bra{10})                                         \\
                                & = \alpha^2\ketbra{01}{01}-\alpha\beta\ketbra{01}{10}-\alpha\beta\ketbra{10}{01}+\beta^2\ketbra{10}{10} \\
          \end{aligned}$$

        We see that:

        $$\begin{aligned}
            \trace_B{\rho} & =\sum_{i=0}^1(\I_A\otimes\bra{i})\rho(\I_A\otimes\ket{i})                                                                                                                                                            \\
                           & =(I_A\otimes\bra{0})\rho(I_A\otimes\ket{0})+(I_A\otimes\bra{1})\rho(I_A\otimes\ket{1})                                                                                                                               \\
                           & =(I_A\otimes\bra{0})\frac{1}{2}\ketbra{\Phi^+}{\Phi^+} + \frac{1}{2}\ketbra{\psi}{\psi}(I_A\otimes\ket{0})+(I_A\otimes\bra{1})\frac{1}{2}\ketbra{\Phi^+}{\Phi^+} + \frac{1}{2}\ketbra{\psi}{\psi}(I_A\otimes\ket{1}) \\
                           & =(I_A\otimes\bra{0})\frac{1}{2}\ketbra{\Phi^+}{\Phi^+} + \frac{1}{2}\ketbra{\psi}{\psi}(I_A\otimes\ket{0})                                                                                                           \\
                           & +(I_A\otimes\bra{1})\frac{1}{2}\ketbra{\Phi^+}{\Phi^+} + \frac{1}{2}\ketbra{\psi}{\psi}(I_A\otimes\ket{1})                                                                                                           \\
                           & =(I_A\otimes\bra{0})                                                                                                                                                                                                 \\
                           & \frac{1}{2}(\frac12\ketbra{00}{00}+\frac12\ketbra{00}{11}+\frac12\ketbra{11}{00}+\frac12\ketbra{11}{11})                                                                                                             \\
                           & + \frac{1}{2}(\alpha^2\ketbra{01}{01}-\alpha\beta\ketbra{01}{10}-\alpha\beta\ketbra{10}{01}+\beta^2\ketbra{10}{10})                                                                                                  \\
                           & (I_A\otimes\ket{0})                                                                                                                                                                                                  \\
                           & +(I_A\otimes\bra{1})                                                                                                                                                                                                 \\
                           & \frac{1}{2}(\frac12\ketbra{00}{00}+\frac12\ketbra{00}{11}+\frac12\ketbra{11}{00}+\frac12\ketbra{11}{11})                                                                                                             \\
                           & + \frac{1}{2}(\alpha^2\ketbra{01}{01}-\alpha\beta\ketbra{01}{10}-\alpha\beta\ketbra{10}{01}+\beta^2\ketbra{10}{10})                                                                                                  \\
                           & (I_A\otimes\ket{1})                                                                                                                                                                                                  \\
                           & =(I_A\otimes\bra{0})\frac{1}{2}(\frac12\ketbra{00}{00})+\frac{1}{2}(\beta^2\ketbra{10}{10})(I_A\otimes\ket{0})                                                                                                       \\
                           & +(I_A\otimes\bra{1})\frac{1}{2}(\frac12\ketbra{11}{11})+ \frac{1}{2}(\alpha^2\ketbra{01}{01})(I_A\otimes\ket{1})                                                                                                     \\
                           & =\frac{1}{4}\ketbra{0}{0}+\frac{\beta^2}{2}\ketbra{1}{1}+\frac{1}{4}\ketbra{1}{1}+\frac{\alpha^2}{2}\ketbra{0}{0}                                                                                                    \\
                           & =\frac{1}{4}\ketbra{0}{0}+\frac{\alpha^2}{2}\ketbra{0}{0}+\frac{\beta^2}{2}\ketbra{1}{1}+\frac{1}{4}\ketbra{1}{1}                                                                                                    \\
                           & =\frac{1+2\alpha^2}{4}\ketbra{0}{0}+\frac{1+2\beta^2}{4}\ketbra{1}{1}                                                                                                                                                \\
          \end{aligned}$$

  \item  Let $\ket{\psi}=\alpha_0\ket{a_0}\ket{b_0}+\alpha_1\ket{a_1}\ket{b_1}$ be the Schmidt decomposition of a two-qubit state $\ket{\psi}$. Prove that for any single qubit unitaries $U$ and $V$, $\ket{\psi}$ is entangled if and only if $\ket{\psi'}=(U\otimes V)\ket{\psi}$ is entangled. (Hint: Prove that the Schmidt rank of $\ket{\psi}$ equals that of $\ket{\psi'}$. Also, you might find Lemma 1 of the Lecture 3 notes useful.)


        $$\begin{aligned}
            \ket{\psi'} & =(U\otimes V)\ket{\psi}                                                            \\
                        & =(U\otimes V)(\alpha_0\ket{a_0}\ket{b_0}+\alpha_1\ket{a_1}\ket{b_1})               \\
                        & =(U\otimes V)(\alpha_0\ket{a_0}\ket{b_0})+(U\otimes V)(\alpha_1\ket{a_1}\ket{b_1}) \\
                        & =\alpha_0(U\ket{a_0}\otimes V\ket{b_0})+\alpha_1(U\ket{a_1}\otimes V\ket{b_1})     \\
          \end{aligned}$$

        Where $\{U\ket{a_0}, U\ket{a_1}\}$ forms an Orthonormal basis for $\mathbb{C}^{d_A}$.
        And $\{V\ket{b_0}, V\ket{b_1}\}$ forms an Orthonormal basis for $\mathbb{C}^{d_B}$.
        We have previously shown that a unitary Operator preserves Orthonomality.

        Now we have a Schmidt Decomposition of $\ket{\psi'}$ with coefficients $\alpha_0$ and $\alpha_1$.

        Now $\ket{\psi}$ is entangled if and only if all the coefficients $\alpha_0$ and $\alpha_1$ are non-zero.
        Now $\ket{\psi'}$ is entangled if and only if all the coefficients $\alpha_0$ and $\alpha_1$ are non-zero.
        Now we have shown that $\ket{\psi}$ is entangled if and only if $\ket{\psi'}$ is entangled.
\end{enumerate}

\end{document}