\documentclass{article}
\usepackage{amsmath,amsthm}
\usepackage{amssymb,latexsym}
\usepackage{epsfig}
\usepackage{hyperref}
\usepackage{float}
\usepackage{fullpage}
\usepackage{enumerate}
\usepackage{paralist}
\usepackage{graphicx}


\newtheorem{theorem}{Theorem}
\newtheorem{corollary}[theorem]{Corollary}
\newtheorem{lemma}[theorem]{Lemma}
\newtheorem{observation}[theorem]{Observation}
\newtheorem{proposition}[theorem]{Proposition}
\newtheorem{definition}[theorem]{Definition}
\newtheorem{claim}[theorem]{Claim}
\newtheorem{fact}[theorem]{Fact}
\newtheorem{assumption}[theorem]{Assumption}
\newtheorem{example}[theorem]{Example}
\newtheorem{conjecture}[theorem]{Conjecture}
\newtheorem{alg}[theorem]{Algorithm}
\newtheorem{protocol}[theorem]{Protocol}
\newtheorem{problem}[theorem]{Problem}

\newcommand{\ip}[2]{\left\langle #1 , #2\right\rangle}
\newcommand{\tr}{\trace}
\newcommand{\setft}[1]{\mathrm{#1}}
\newcommand{\lin}[1]{\setft{L}\left(#1\right)}
\newcommand{\density}[1]{\setft{D}\left(#1\right)}
\newcommand{\unitary}[1]{\setft{U}\left(#1\right)}
\newcommand{\herm}[1]{\setft{Herm}\left(#1\right)}
\newcommand{\pos}[1]{\setft{Pos}\left(#1\right)}
\newcommand{\sep}[1]{\setft{Sep}\left(#1\right)}
\newcommand{\rank}[1]{\operatorname{rank}(#1)}
\newcommand{\ex}{\paragraph{Exercise.}}

\def\I{I}
\def\({\left(}
\def\){\right)}
\def\X{\mathcal{X}}
\def\Y{\mathcal{Y}}
\def\Z{\mathcal{Z}}
\def\W{\mathcal{W}}
\def\yes{\text{yes}}
\def\no{\text{no}}
\def\blog{\textup{log}}
\newcommand{\A}{\spa{A}}
\newcommand{\B}{\spa{B}}
\newcommand{\UA}{U_A}

\newcommand{\eps}{\varepsilon}
\newcommand{\epssdp}{\varepsilon_{\rm sdp}}
\newcommand{\bra}[1]{\langle #1|}
\newcommand{\ket}[1]{|#1\rangle}
\newcommand{\braket}[2]{\langle #1|#2\rangle}
\newcommand{\ketbra}[2]{\ket{#1}{\bra{#2}}}
\newcommand{\lmin}[1] {\lambda_{\operatorname{min}}(#1)}
\newcommand{\lmax}[1] {\lambda_{\operatorname{max}}(#1)}
\newcommand{\lhyes} {\operatorname{LH_{yes}}}
\newcommand{\lhno} {\operatorname{LH_{no}}}
\newcommand{\CQ}{\mathcal{CQ}}
\newcommand{\lh}{\operatorname{LH}}
\newcommand{\flh}{\operatorname{5-LH}}
\newcommand{\klhh}{\operatorname{k-LH}}
\newcommand{\qma}{\operatorname{QMA}}
\newcommand{\enc}[1]{\left<#1\right>}

\newcommand{\C}{C}
\newcommand{\Id}{Id} %CHECK
\newcommand{\Exs}[2]{E_{#1}[#2]} %CHECK

\newcommand{\beq}{\begin{equation}}
\newcommand{\eeq}{\end{equation}}

\newcommand{\trace}{{\rm Tr}}

\newcommand{\norm}[1]{\left\|\,#1\,\right\|}       % norm
\newcommand{\pnorm}[1]{\left\|\,#1\,\right\|_p}       % norm
\newcommand{\onorm}[1]{\norm{#1}_{\mathrm{1}}}      % Euclidean norm for vectors
\newcommand{\enorm}[1]{\norm{#1}_{\mathrm{2}}}      % Euclidean norm for vectors
\newcommand{\trnorm}[1]{\norm{#1}_{\mathrm {tr}}}  % trace norm
\newcommand{\fnorm}[1]{\norm{#1}_{\mathrm {F}}}    % frobenius norm
\newcommand{\snorm}[1]{\norm{#1}_{\mathrm {\infty}}}    % spectral norm

\newcommand{\set}[1]{{\left\{#1\right\}}}    % braces for set notation
\newcommand{\ve}[1]{\mathbf{#1}}
\newcommand{\abs}[1]{\left\lvert #1 \right\rvert}
\newcommand{\optprod}{\OPT_P}
\newcommand{\opt}{\operatorname{OPT_1}}
\newcommand{\optt}{\operatorname{OPT_2}}
\newcommand{\newopt}{\operatorname{NEW-OPT}}
\newcommand{\swap}{\operatorname{SWAP}}
\newcommand{\poly}{\operatorname{poly}}
\newcommand{\cc}{d^{\frac{k}{2}}}
\newcommand{\OPT}{{\rm OPT}}
\newcommand{\QMA}{{\rm QMA}}
\newcommand{\MQA}{{\rm MQA}}
\newcommand{\NP}{{\rm NP}}
\newcommand{\PP}{{\rm P}}
\newcommand{\PH}{{\rm PH}}
\newcommand{\BPP}{{\rm BPP}}
\newcommand{\BQP}{{\rm BQP}}
\newcommand{\TCSP}{{\rm 2-CSP}}

\newcommand{\complex}{{\mathbb C}}
\newcommand{\reals}{{\mathbb R}}
\newcommand{\ints}{{\mathbb Z}}
\newcommand{\nats}{{\mathbb N}}

\newcommand{\spa}[1]{\mathcal{#1}}
\newcommand{\dens}{\mathcal{D}(\spa{A}\otimes\spa{B})}
%\newcommand{\unitaries}{U(\spa{A}\otimes\spa{B})}

\newcommand{\LL}{\mathcal{L}}
\newcommand{\DD}{\mathcal{D}}
\newcommand{\HH}{\mathcal{H}}
\newcommand{\UU}{\mathcal{U}}

\newcommand{\ayes}{A_{\rm yes}} %CHECK
\newcommand{\ano}{A_{\rm no}} %CHECK


\begin{document}

\title{\vspace{-10mm}Introduction to Quantum Computation, UPB\\Winter 2022, Assignment 5\\{\large Due: Thursday, November 17, at start of tutorial}}
\date{}
\maketitle



\section{Exercises}
\begin{enumerate}
    \item Define $\ket{\psi}=\alpha\ket{0}+\beta\ket{1}\in\complex^2$, and consider projective measurement $M=\set{\ketbra{\psi}{\psi},I-\ketbra{\psi}{\psi}}$ with labels corresponding to outcomes $S=\set{1,-1}$, respectively. Suppose state $\ket{0}\in\complex^2$ is measured via $M$. What is the expected value for the measurement?
    \item In this question, we consider how well the CHSH game strategy from class fares if we use a \emph{less} entangled state as a shared resource between Alice and Bob. Specifically, imagine we use the same observables as before, but now we replace $\ket{\Phi^+}$ as a shared state with $\ket{\psi}=\alpha\ket{00}+\beta\ket{11}$. Intuitively, as $\alpha$ gets closer to $1$, this state becomes less entangled, and for $\alpha=1$, it becomes a product state (i.e. non-entangled).
        \begin{enumerate}
            \item What is the probability of winning the CHSH game with shared state $\ket{\psi}$? (Hint: Recall from Lecture 5 that for \emph{any} $\ket{\psi}$, the quantity $\trace(A\otimes B\ketbra{\psi}{\psi})$ equals $\Pr(\text{output same bits})-\Pr(\text{output different bits})$, i.e. the interpretation of this quantity does not depend on our choice of $\ket{\psi}$.)
            \item Based on your answer above, what is the probability of Alice and Bob winning with this strategy if $\alpha=1$, i.e. $\ket{\psi}$ is unentangled?
        \end{enumerate}

    \item This question studies a $3$-player non-local game called the \emph{GHZ game}. There are now three players, Alice, Bob and Charlie, each of which receives a question $q_a$, $q_b$, or $q_c$, respectively, such that $q_aq_bq_c\in\set{000,011,101,110}$. The players each return a bit $r_a,r_b,r_c\in\set{0,1}$, respectively, and win if
        \[
            q_a\vee q_b\vee q_c = r_a\oplus r_b\oplus r_c,
        \]
        where $\vee$ denotes the binary OR operation.

        An analysis similar to the CHSH game shows that the optimal winning classical strategy yields success probability $3/4$. Your task in this question is to analyze an optimal quantum strategy.

        The $3$-qubit state the players share is
        \[
            \ket{\psi}=\frac{1}{2}\left(\ket{000}-\ket{011}-\ket{101}-\ket{110}\right)\in\complex^8.
        \]
        Each player will use the same measuring strategy: Given input bit $0$, they will apply local unitary $U_0=I$, and if they get input $1$, they apply local unitary $U_1=H$. They then measure their qubit in the standard basis, and return the answer ($0$ or $1$). As for CHSH, we assume the labels for the measurement outcomes are $+1,-1$ for measurement outcomes $\ketbra{0}{0}$ and $\ketbra{1}{1}$, respectively.
        \begin{enumerate}
            \item Suppose Alice gets input $0$. What is her observable? What if she gets input $1$?
            \item Suppose the questions are $q_Aq_Bq_C=000$. What is the probability the players win?
            \item Suppose the questions are $q_Aq_Bq_C=011$. What is the probability the players win?
        \end{enumerate}
\end{enumerate}

\end{document}
