\documentclass{article}

\usepackage[utf8]{inputenc}
\usepackage[ngerman]{babel}
\usepackage{amssymb}
\usepackage{amsmath}

% Pseudocode
\usepackage{algorithm}
\usepackage[noend]{algpseudocode}

\setlength{\parindent}{0in}

\newcommand{\uebungsGruppe}{1}
\newcommand{\zettelNummer}{1}
\newcommand{\studierenderEins}{Eli Kogan-Wang (7251030)}
\newcommand{\studierenderZwei}{David Noah Stamm (7249709)}
\newcommand{\studierenderDrei}{Daniel Heins (7213874)}
\newcommand{\studierenderVier}{Tim Wolf (7269381)}

\newcounter{AufgabenCounter}
\setcounter{AufgabenCounter}{1}
\newcounter{TeilaufgabenCounter}
\newenvironment{aufgabe}{\section*{Aufgabe \theAufgabenCounter}\setcounter{TeilaufgabenCounter}{1}}{\stepcounter{AufgabenCounter}}
\newenvironment{teilaufgabe}{\paragraph*{\alph{TeilaufgabenCounter})}}{\stepcounter{TeilaufgabenCounter}}

\renewcommand{\to}{\textnormal{ to }}
\newcommand{\bigO}{\mathcal{O}}

\newcommand{\qed}{\hfill$\square$}

\begin{document}

\title{Datenstrukturen und Algorithmen \\ Heimübung \zettelNummer{}}
\author{\studierenderEins{} \\
  \studierenderZwei{} \\
  \studierenderDrei{} \\
  \studierenderVier{}}

\maketitle

\begin{aufgabe}

  Sei $G := (V, E)$ ein zusammenhängender, ungerichteter Graph
  mit exakt einem Pfad pro Knotenpaar
  und nicht-leerer, endlicher Knotenmenge.

  Zu zeigen:

  $$\sum_{v\in V}deg(v)=2(|V|-1)$$

  Alle über den Beweis betrachteten Graphen haben die beschriebenen Eigenschaften.

  Beweis über Induktion, über die Anzahl der Knoten $|V|$:

  \textbf{I.A.}\\
  Sei $G := (V, E)$ mit $|V|=1$.\\
  Also $E=\emptyset$ aus den Eigenschaften.\\
  Also $\forall v\in V,\deg(v)=0$.\\
  Also $\sum_{v\in V}deg(v)=0=2(|V|-1)$.\\
  \textbf{I.V.}\\
  Sei $n\in \mathbb{N}^+$.\\
  $\forall G := (V, E)$ mit $|V|=n$ gilt:
  $$\sum_{v\in V}deg(v)=2(n-1)$$
  \textbf{I.S.}\\
  Sei $G := (V, E)$ mit $|V|=n+1$.\\
  Aus den Eigenschaften von $G$ folgt:\\
  Es gibt einen Knoten $v_{Blatt}$ mit $deg(v_{Blatt})=1$.\\
  Wir betrachten den induzierten Teilgraph $G' := (V', E')$ mit $V'=V\backslash\{v_{Blatt}\}$.\\
  Für $G'$ gilt aus I.V.
  $$\sum_{v\in V'}deg_{G'}(v)=2(n-1)$$
  $E\backslash E'=\{v_{Blatt},v_{Parent}\}$
  $$\forall v\in V: deg_G(v)=\begin{aligned}\begin{cases}
        1             & \text{wenn }v=v_{Blatt}  \\
        deg_{G'}(v)+1 & \text{wenn }v=v_{Parent} \\
        deg_{G'}(v)   & \text{sonst}
      \end{cases}\end{aligned}$$

  Also:
  $$\sum_{v\in V}deg_{G}(v)=2(n-1)+2$$

  $$2(n-1)+2=2((n+1)-1)=2(|V|-1)$$

  \qed

\end{aufgabe}

\begin{aufgabe}
  Zu zeigen:
  $$\forall n\in \mathbb{N}:\sum_{i=1}^{n}i^3=\frac{n^2\cdot (n+1)^2}{4}$$

  Beweis über Induktion, über die $n$.

  \textbf{I.A.}\\
  $n=1$.\\
  Also $\sum_{i=1}^{1}i^3=1^3=1$.\\
  $\frac{1^2\cdot (1+1)^2}{4}=1$.\\
  \textbf{I.V.}\\
  Sei $n\in \mathbb{N}$.\\
  Es gilt:
  $$\sum_{i=1}^{n}i^3=\frac{n^2\cdot (n+1)^2}{4}$$
  \textbf{I.S.}\\
  Zu zeigen:
  $$\sum_{i=1}^{n+1}i^3=\frac{(n+1)^2\cdot ((n+1)+1)^2}{4}$$
  $$
    \begin{aligned}
      \sum_{i=1}^{n+1}i^3 & = (n+1)^3+\sum_{i=1}^{n}i^3                           \\
                          & \overset{I.V.}{=} (n+1)^3+\frac{n^2\cdot (n+1)^2}{4}  \\
                          & = \frac{4\cdot (n+1)^3}{4}+\frac{n^2\cdot (n+1)^2}{4} \\
                          & = \frac{4\cdot (n+1)^3+n^2\cdot (n+1)^2}{4}           \\
                          & = \frac{(4\cdot(n+1)+n^2) \cdot (n+1)^2}{4}           \\
                          & = \frac{(n^2+2\cdot2\cdot n + 2^2) \cdot (n+1)^2}{4}  \\
                          & = \frac{(n+2)^2 \cdot (n+1)^2}{4}                     \\
                          & = \frac{((n+1)+1)^2 \cdot (n+1)^2}{4}                 \\
                          & = \frac{(n+1)^2 \cdot ((n+1)+1)^2}{4}                 \\
    \end{aligned}
  $$
  \qed
\end{aufgabe}

\begin{aufgabe}
  \begin{teilaufgabe}
    Lineare Suche in Pseudocode:
    \begin{algorithm}[H]
      \caption{\textsc{LinSearch}($A[1, \dots, n], v$)}
      \begin{algorithmic}[1]
        \For{$i \gets 1 \to n$}
        \If{$A[i]=v$}
        \State \textbf{Return} $i$ \label{return:i}
        \EndIf
        \EndFor
        \State \textbf{Return} \textsc{Nil} \label{return:nil}
      \end{algorithmic}
    \end{algorithm}
  \end{teilaufgabe}
  \begin{teilaufgabe}
    Schleifeninvariante $I(i)$: $\forall j\in [1,i): A[j]\neq v$

    \textbf{Initialisierung}
    $$i=1\implies[1,i)=\emptyset\implies\forall j\in\emptyset: A[j]\neq v =: I(1)$$
    \textbf{Erhaltung}
    Angenommen $I(i)$.
    Wenn $A[i]=v$ dann brechen wir ab und das Ende des Schleifendurchlaufs wird nicht erreicht (siehe Termination).
    Sonst, wenn $A[i]\neq v$ dann:
    $$(I(i) := \forall j\in [1,i): A[j]\neq v) \land A[i]\neq v\iff\forall j\in [1,i+1): A[j]\neq v =: I(i+1)$$
    Also gilt $I(i+1)$ am Ende des Schleifendurchlaufs.

    \textbf{Terminierung}
    Die Schleife kann aus 2 Gründen beendet werden:
    \begin{enumerate}
      \item \textbf{Return} $i$ (Zeile \ref{return:i}) wenn $A[i]=v$.
            $I(i)$ gilt genau wie am Anfang des Schleifendurchlaufs,
            wie in Erhaltung beschrieben, da weder $i$ noch $A$ nach der erfrühten
            Termination verändert wurde.
            \label{return:i:description}
      \item \textbf{Return} \textsc{Nil} (Zeile \ref{return:nil}) wenn
            $i=n+1$ (bestimmte Modelle von \textbf{for} Enden mit $i=n$;
            dieselbe Argumentation gilt, aber die Korrektheit muss mühseliger begründet
            werden).
            $$I(n+1) := \forall j\in [1,n+1): A[j]\neq v \iff\forall j\in [1,n]: A[j]\neq v$$
            $I(n+1)$ gilt, da sonst Terminationsfall \ref{return:i:description} erreicht würde.
            Also: $v$ ist nicht in $A$ enthalten, also wird \textsc{Nil} zurückgegeben.
            \label{return:nil:description}
    \end{enumerate}
    Der Terminationsfall \ref{return:i:description} korrespondieren zum Fall, dass $i$ mit $x_i=v$ zurückgegeben wird.
    Der Terminationsfall \ref{return:nil:description} korrespondieren zum Fall, dass $\textbf{Nil}$ mit $\not\exists i\in [1,n]: x_i=v$ zurückgegeben wird.
    Damit erfüllt der Algorithmus das gewünschte Verhalten.
    \qed
  \end{teilaufgabe}
  \begin{teilaufgabe}
    Laufzeitbestimmung
    \begin{algorithm}[H]
      \caption{\textsc{LinSearch}($A[1, \dots, n], v$)}
      \begin{algorithmic}[1]
        \For{$i \gets 1 \to n$} \Comment{$\sum_{i=1}^nT(I)$}
        \If{$A[i]=v$} \Comment{$O(1)$}
        \State \textbf{Return} $i$ \Comment{$O(1)$}
        \EndIf
        \EndFor
        \State \textbf{Return} \textsc{Nil} \Comment{$O(1)$}
      \end{algorithmic}
    \end{algorithm}
    Laufzeit: $$\sum_{i=1}^n(O(1)+O(1))+O(1)=O(n)$$
  \end{teilaufgabe}
\end{aufgabe}
\end{document}
