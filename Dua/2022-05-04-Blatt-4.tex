\documentclass{article}

\usepackage[utf8]{inputenc}
\usepackage[ngerman]{babel}
\usepackage{amssymb}
\usepackage{amsmath}
\usepackage{stmaryrd}

\usepackage{graphicx}

% Pseudocode
\usepackage{algorithm}
\usepackage[noend]{algpseudocode}

\setlength{\parindent}{0in}

\newcommand{\uebungsGruppe}{110}
\newcommand{\zettelNummer}{4}
\newcommand{\studierenderEins}{Eli Kogan-Wang (7251030)}
\newcommand{\studierenderZwei}{David Noah Stamm (7249709)}
\newcommand{\studierenderDrei}{Daniel Heins (7213874)}
\newcommand{\studierenderVier}{Tim Wolf (7269381)}

\newcounter{AufgabenCounter}
\setcounter{AufgabenCounter}{1}
\newcounter{TeilaufgabenCounter}
\newenvironment{aufgabe}{\section*{Aufgabe \theAufgabenCounter}\setcounter{TeilaufgabenCounter}{1}}{\stepcounter{AufgabenCounter}}
\newenvironment{teilaufgabe}{\paragraph*{\alph{TeilaufgabenCounter})}}{\stepcounter{TeilaufgabenCounter}}

\renewcommand{\to}{\textnormal{ to }}
\newcommand{\bigO}{\mathcal{O}}

\newcommand{\qed}{\hfill$\square$}

% Load the setspace package
\usepackage{setspace}
\doublespacing

\begin{document}

\title{Datenstrukturen und Algorithmen \\ Heimübung \zettelNummer{}}
\author{\studierenderEins{} \\
  \studierenderZwei{} \\
  \studierenderDrei{} \\
  \studierenderVier{}}

\maketitle

\begin{aufgabe}
  Herleitung der geschlossenen Formel T(n): \\
  Angenommen $n  = 4^{k}$
  $$
    \begin{aligned}
      T(n) & = T\left(4^{k}\right)                                                                                                                        \\
           & = 4 T\left(\frac{4^{k}}{4}\right) + \sqrt{4^{k}} \cdot c_2                                                                                   \\
           & = 4 \cdot T\left(4^{k-1}\right) + 2^{k} \cdot c_2                                                                                            \\
           & = 4 \left(4 \cdot T\left(4^{k-2}\right)+ 2^{k-1} \cdot c_2\right) + 2^{k} \cdot c_2                                                          \\
           & = 4^{2} \underbrace{T\left(4^{k-2}\right)}_\text{sukzessives Einsetzen} + 2^{k+1} \cdot c_2 + 2^{k} \cdot c_2                                \\
           & = 4^{k} c_1 + \left(\sum_{j=0}^{k-1} 2^{k+i}\right)c_2                                                        & | \text{ geometrische Summe} \\
           & = 4^{k} c_1 + \left(2^{k} \cdot \sum_{j=0}^{k-1} 2^{i}\right)c_2                                                                             \\
           & = 4^{k} c_1 +  2^{k} \left(\frac{1-2^{k}}{1-2}\right)c_2                                                                                     \\
           & = 4^{k} c_1 +  2^{k} \left(2^{k}-1\right)c_2                                                                                                 \\
           & = 4^{k} \left(c_1 + c_2\right) - 2^{k}c_2
    \end{aligned}
  $$

  Korrektheitsbeweis der Formel durch vollständige Induktion:  \\
  Beh. : $$T(n) = T(4^{k}) = 4^{k} (c_1 + c_2) - 2^{k}c_2 $$

  IA: $n = 1$

  $T(1) = T(4^{0}) = c_1 = 4^{0} (c_1 + c_2) - 2^{0}c_2 $

  Iv: Es gilt: $T(n) = T(4^{k-1}) = 4^{k-1} (c_1 + c_2) - 2^{k-1}c_2 $

  Is: Zu Zeigen: $T(n) = T(4^{k}) = 4^{k} (c_1 + c_2) - 2^{k}c_2 $

  Nach Definition von $T(n)$ gilt (da $n > 1$ ist):

  $$
    \begin{aligned}
      T(n) & = T(4^{k})                                                                           \\
           & = 4 T(\frac{4^{k}}{4}) + \sqrt{4^{k}} \cdot c_2                                      \\
           & = 4 \cdot T(4^{k-1})+ 2^{k} \cdot c_2                                                \\
           & \stackrel{IV}{=} 4 \cdot (4^{k-1} (c_1 + c_2) - 2^{k-1}c_2) + \sqrt{4^{k}} \cdot c_2 \\
           & = 4^{k}(c_1 + c_2)-2^{k+1}c_2 + 2^{k}\cdot c_2                                       \\
           & = 4^{k} (c_1 + c_2) - 2^{k}c_2                                                       \\
    \end{aligned}
  $$

  \qed

\end{aufgabe}
\begin{aufgabe}
  Aus der Rekursionsgleichung folgt: $a = 2,  b = 2 $ und $f(n) = n\log n$ n\\
  $\Rightarrow $ $n^{\log_b(a)} = n^{\log_2(2)} = n^1$ \\
  $f(n) = n\log n = n^{\log_n(n\log_2n)} = n^{\log_n(n)+\log_2(n)} = n^{1+\log_2(n)}$ \\
  Das Mastertheorem kann nur angewendet werden, \\
  falls f(n) $\subset\mathcal{O}(n^{1- \varepsilon }) \lor f(n) \subset \Theta(n)
    \lor (f(n) \subset \Omega(n^{1+\varepsilon}) \land a\cdot f(\frac{n}{b}) \leq c \cdot f(n))   $

  Es gilt trivialerweise:
  $$
    \begin{aligned}
      n^{1 + \log_2(n)}                 & \not\subset\mathcal{O}(n)                    \\
      \Rightarrow n^{1 + \log_2(n)}     & \not\subset\mathcal{O}(n^{1- \varepsilon } ) \\
      \text{und\quad }n^{1 + \log_2(n)} & \not\subset \Theta(n)
    \end{aligned}
  $$
  Ist $f(n) \subset \Omega(n^{1+\varepsilon})$ ? \\
  $\iff \exists c, n_0 >0$, so dass $\forall n: n^{1+\log_2(n)} \ge n_0: \geq c \cdot n^{1+\varepsilon} $ \\
  Dies für $ c = 1 $ und $ \varepsilon = \frac{1}{10}$ erfüllt \\
  $ \Rightarrow f(n) \subset \Omega(n^{1+\varepsilon}) $, aber die zweite Bedingung ist nicht erfüllt\\
  $2 \cdot \frac{n}{2} \log(\frac{n}{2})= \leq c \cdot n \cdot log (n)  $ \\
  $n \cdot \log(\frac{n}{2}) \leq c \cdot n \cdot log (n)  $ \\
  Diese Bedingung wird nur von c = 1 erfüllt, da

  $ \displaystyle \lim_{n \rightarrow \infty} \frac{n \cdot \log(\frac{n}{2})}{n \cdot log (n)}$ \\
  $ = \displaystyle \lim_{n \rightarrow \infty} \frac{\log(\frac{n}{2})}{log (n)}$ \\
  Da $\displaystyle \lim_{n \rightarrow \infty} \log(\frac{n}{2}) = \infty$
  $ \land \displaystyle \lim_{n \rightarrow \infty} \log(n) = \infty$ gilt, \\
  kann die Regel vom Krankenhaus angewendet werden :°) \\
  $\displaystyle \lim_{n \rightarrow \infty} \frac{\log(\frac{n}{2})}{log (n)} $
  $ = \displaystyle \lim_{n \rightarrow \infty} \frac{ \frac{\mathrm{d} }{\mathrm{d}n} \log(\frac{n}{2})}{ \frac{\mathrm{d} }{\mathrm{d}n}log (n)}$\\
  $\displaystyle \lim_{n \rightarrow \infty} \frac{\frac{1}{n}}{\frac{1}{n}} = \displaystyle \lim_{n \rightarrow \infty} 1 = 1 $\\
  $\Rightarrow c = 1 \lightning c < 1  $ \\
  $\Rightarrow$ Die Rekursionsgleichung kann nicht mit dem allgemeinen Master-Theorem analysiert werden
\end{aufgabe}
\begin{aufgabe}
  \begin{teilaufgabe}
    \begin{algorithm}[H]
      \caption{\textsc{3Sort}($A, i, j$)}
      \begin{algorithmic}[1]
        \If{$j\leq i+1$}
        \If{$A[i] > A[j]$}
        \State $A[i] \leftrightarrow A[j]$
        \EndIf
        \State return
        \EndIf
        \State $k \gets \left\lfloor\frac{j-i+1}{3}\right\rfloor$
        \State 3Sort($A, i, j-k$) \label{3Sort:1}
        \State 3Sort($A, i+k, j$) \label{3Sort:2}
        \State 3Sort($A, i, j-k$) \label{3Sort:3}
      \end{algorithmic}
    \end{algorithm}

  \end{teilaufgabe}
  \begin{teilaufgabe}
    Sei $n:=len(A)$.
    Beweis über Induktion über $n$:\\
    \textbf{Basisfall:}\\
    $n = 0$: Trivial\\
    $n = 1$: Trivial\\
    $n = 2$:

    Mit $a>b$:

    $A\gets [a,b];\mathrm{3Sort}(A,0,1)\implies A=[b,a]$\\
    $A\gets [b,a];\mathrm{3Sort}(A,0,1)\implies A=[b,a]$

    Nun Induktion:

    Sei $n\in\mathbb{N}$

    Angenommen $3Sort(A,i,j)$ sortiert $A$ wenn $j-i+1\leq n$

    Zu zeigen ist: $3Sort(A,i,j)$ sortiert $A$ wenn $j-i+1=n+1$



    Wir betrachten eine Ausführung von 3Sort mit $j-i+1=n+1$:

    3Sort in Zeile \ref{3Sort:1} sortiert $A[i:j-k]$, da
    $$((j-k)-i+1)=\left(\left(j-\left\lfloor\frac{j-i+1}{3}\right\rfloor\right)-i+1\right)=\left\lceil\frac{2}{3}\cdot(j-i+1)\right\rceil \leq n$$

    $A=(a|b|c)$ wobei $a,b,c$ die drittel von $A$ sind.

    Da $A[i:j-k]$ nun sortiert ist, ist bekannt, dass das zweite $\frac13$ von $A$ ($b$) größer als das erste $\frac13$ von $A$ ($a$) ist.
    Da jedes Element aus $a$ nun mindestens $\frac13\cdot(n+1)$ Elemente über sich hat mit wissen wir, dass kein Element der obersten $\frac13$ von $A$ ($c$) im ersten $\frac13$ von $A$ ($a$) ist.
    Alle Elemente, die am Ende in $c$ sein sollen sind nun in $(b|c)$.

    Nach der Ausführung von 3Sort in Zeile \ref{3Sort:2} ist damit $c$ korrekt populiert.

    Alle Elemente von $a,b$ befinden sich in $(a|b)$ und $c$ hat keine Elemente von $a$ und $b$.
    Nach der Ausführung von 3Sort in Zeile \ref{3Sort:3} ist damit $a,b$ korrekt populiert.
    $c$ wurde nicht geändert und ist immernoch korrekt populiert. Nun ist $A=(a|b|c)$ sortiert.
  \end{teilaufgabe}

  \begin{teilaufgabe}

    $$T(n)=\begin{cases}
        1                        & \text{falls } n=1     \\
        3\cdot T(\frac{2}{3}n)+1 & \text{falls } n\geq 2 \\
      \end{cases}$$
    Nach Mastertheorem:

    $a=3;b=\frac{3}{2};c=1$

    Da $a>b$: $T(n)\in\Theta(n^{\log_{}\frac{3}{2}(3)})$

  \end{teilaufgabe}
\end{aufgabe}

\end{document}

