  \documentclass{article}

  \usepackage[utf8]{inputenc}
  \usepackage[ngerman]{babel}
  \usepackage{amssymb}
  \usepackage{amsmath}
  \usepackage{graphics}
  % Pseudocode
  \usepackage{algorithm}
  \usepackage[noend]{algpseudocode}
  \usepackage{graphicx}
  \usepackage{pdfpages}

  \usepackage{tikz}

  \setlength{\parindent}{0in}

  \newcommand{\zettelNummer}{10}
  \newcommand{\studierenderEins}{Eli Kogan-Wang (7251030)}
  \newcommand{\studierenderZwei}{David Noah Stamm (7249709)}
  \newcommand{\studierenderDrei}{Daniel Heins (7213874)}
  \newcommand{\studierenderVier}{Tim Wolf (7269381)}

  \newcounter{AufgabenCounter}
  \setcounter{AufgabenCounter}{1}
  \newcounter{TeilaufgabenCounter}
  \newenvironment{aufgabe}{\section*{Aufgabe \theAufgabenCounter}\setcounter{TeilaufgabenCounter}{1}}{\stepcounter{AufgabenCounter}}
  \newenvironment{teilaufgabe}{\paragraph*{\alph{TeilaufgabenCounter})}}{\stepcounter{TeilaufgabenCounter}}

  \renewcommand{\to}{\textnormal{ to }}
  \newcommand{\bigO}{\mathcal{O}}

  \newcommand{\qed}{\hfill$\square$}

\begin{document}

\title{Datenstrukturen und Algorithmen \\ Heimübung \zettelNummer{}}
\author{\studierenderEins{} \\
  \studierenderZwei{} \\
  \studierenderDrei{} \\
  \studierenderVier{}}

\maketitle

% Aufgabe 1
\begin{aufgabe}
  Für die Adjazenzlistendarstellung:

  Sei $G_{new}$ ein neuer leerer Graph in Adjazenzlistendarstellung. Er wird im Laufe des Algorithmus
  mit den Einträgen des Transponierten Graphen $G^T$ erweitert.

  Für jeden Knoten $v$ aus $V(G)$: ($|V|$ mal)

  Wir fügen $v$ zu $G_{new}$ hinzu. ($O(1)$)

  Für jeden Knoten $v$ aus $V(G)$: ($|V|$ mal)

  Wir iterieren über die ursprüngliche Adjazenzliste
  $A(v)$ mit $v_{adj}$ adjazent zu $v$: ($|A(v)|$ mal)

  Wir fügen zur Adjazenzliste von $v_{adj}$ den Knoten $v$ hinzu. ($O(1)$)

  Nun ist $G_{new}$ eine Adjazenzlistendarstellung von $G^T$.

  Bemerkung: $|V|\cdot|A(v)|$ = $O(|E|)$

  Das heißt wir üben $O(|V|+|E|)$ Elementare Operationen aus.

  Der Algorithmus ist korrekt, weil er alle und nur diese Kanten aus $G^T$ dem
  Graphen $G_{new}$ hinzufügt.

  \rule{\textwidth}{0.5pt}

  Für die Adjazenzmatrixdarstellung:

  Sei $G_{new}$ ein neuer kantenloser Graph in Adjazenzmatrixdarstellung
  über dieselben Knoten $V$ aus $G$.

  Nun sei $A$ die Adjazenzmatrix von $G$ und $A_{new}$ die Adjazenzmatrix von $G_{new}$.

  Nun für $1\leq i\leq |V|$: ($O(|V|)$) Und $1\leq j\leq |V|$: ($O(|V|)$)

  $A_{new}(j,i) = A(i,j)$ ($O(1)$)

  Nun ist $A_{new}=A^T$ und $G_{new}$ eine Adjazenzmatrixdarstellung von $G^T$.

  Der Algorithmus übt $O(|V|^2)$ Elementare Operationen aus.

  Die Korrektheit des Algorithmus lässt sich aus der Vertauschung der Indizes $i$ und $j$
  begründen.
\end{aufgabe}

% Aufgabe 2
\begin{aufgabe}

  Der Algorithmus \textsc{Bipartite-Check-Breadth-First-Search} nimmt einen Graphen $G$ und einen Startknoten $s\in V$.

  \begin{algorithm}[H]
    \caption{\textsc{Bipartite-Check-Breadth-First-Search}($G, s$)}
    \begin{algorithmic}[1]
      \For{jeden Knoten $u$ in $V\backslash\{s\}$}
      \State $color[u]\gets WHITE$
      \State $\pi[u]\gets NIL$
      \EndFor
      \State $color[s]\gets GRAY$
      \State $\pi[s]\gets NIL$
      \State $Q\gets\{\}$
      \State Enqueue($Q$, $s$)
      \While{$Q\neq\emptyset$}
      \State $u\gets$ Dequeue($Q$)
      \If{$color[u]=GRAY$}
      \State otherColor $\gets BLACK$
      \Else
      \State otherColor $\gets GRAY$
      \EndIf
      \For{jeden Knoten $v$ in $A(u)$} \Comment{$A(u)$: Nachbarn von $u$}
      \If{$color[v]=WHITE$}
      \State $color[v]\gets $ otherColor
      \State $\pi[v]\gets u$
      \State Enqueue($Q$, $v$)
      \ElsIf{$color[v]\neq$ otherColor}
      \State \textbf{return} \textsc{false}
      \EndIf
      \EndFor
      \EndWhile
      \State \textbf{Return} \textsc{true}
    \end{algorithmic}
  \end{algorithm}

  Wir reduzieren das Problem der Bipartitheitsprüfung auf den der $2$-Färbbarkeit.

  Mithilfe der Breitensuche könenn wir die $2$-Färbung auf $G$ versuchen und bei einem Konflikt
  abbrechen.

  Ein Graph ist genau dann $2$-Färbbar, wenn er bipartit ist.

  Unser Algorithmus ist damit korrekt, weil true rückgibt, genau dann wenn der Graph $2$-Färbbar ist.
  Und weil er false rückgibt, wenn der Graph nicht $2$-Färbbar ist.

  Die Laufzeit einer Breitensuche ist aus der Vorlesung mit $O(|V|+|E|)$ Elementaren Operationen bekannt und wurde hierbei  nur durch Elementare Operationen Erweitert, weswegen dieser weiterhin in O(|V|+|E|) liegt.
\end{aufgabe}

\begin{aufgabe}

  Der vorgeschlagene Algorithmus besteht aus 3 Phasen:

  1. Phase:

  Aufteilen der Kanten mit Gewichtung $w\neq 1$ in $w$-Kanten mit Gewichtung $1$.

  Man fügt zusätzliche Knoten hinzu und ballert dadurch zusätzliche Kanten rein.

  Laufzeit: $O(|E|\cdot k)$ (da $w\in [1,k]$)

  2. Phase:

  Breitensuche

  Laufzeit: $O((|V|+|E|)\cdot k)$ da wir Knoten und Kanten hinzugefügt haben.

  3. Phase:

  Vergessen der neu hinzugefügten Knoten+Kanten.

  Wir vergessen die neu hinzugefügten Knoten und Kanten in den gefundenen kürzesten Pfaden.

  Laufzeit: $O(|V|\cdot k)$ (da wir nur Ergebnisse für jeden Knoten speichern und maximal $k$ Knoten+Kanten pro Knoten vergessen)

  Die Gesamtlaufzeit ist damit $O(|V|+|E|\cdot k)$
\end{aufgabe}
\end{document}
