\documentclass[a4paper,12pt]{article}
\usepackage[utf8]{inputenc}
\usepackage[ngerman]{babel}
\usepackage[top=1in, bottom=1.25in, left=1.25in, right=1.25in]{geometry}
\usepackage{fancyhdr}
\usepackage{titling}
\usepackage{amssymb}
\usepackage{mathtools}
\usepackage{enumerate}
\usepackage[shortlabels]{enumitem}
\usepackage{makecell}
\usepackage{minted}


\renewcommand\theadfont{}

\renewcommand{\footrulewidth}{0.4pt}

\setlength\headheight{15pt}
\setlength{\parskip}{1em}

\title{Datenbanksysteme Blatt 1}
\author{
  Eli Kogan-Wang, elikoga, 7251030
  \and
  Niklas Bäumker, niba, 7165553
}
\date{\today}

\pagestyle{fancy}
\fancyhf{}
\lhead{\thetitle}
\rhead{\thedate}
\lfoot{\theauthor}
\rfoot{Seite \thepage}

\newcounter{AufgabenCounter}
\setcounter{AufgabenCounter}{1}
\newcounter{TeilaufgabenCounter}
\newenvironment{aufgabe}{\section*{Aufgabe \theAufgabenCounter}\setcounter{TeilaufgabenCounter}{1}}{\stepcounter{AufgabenCounter}}
% \newenvironment{teilaufgabe}{\paragraph*{\alph{TeilaufgabenCounter})}}{\stepcounter{TeilaufgabenCounter}}
\newenvironment{teilaufgabe}{\paragraph*{\theTeilaufgabenCounter.}}{\stepcounter{TeilaufgabenCounter}}


\begin{document}
\maketitle
\thispagestyle{fancy}

\begin{aufgabe}
  \begin{teilaufgabe}
    \begin{minted}{SQL}
SELECT N, WS
  FROM Held;
    \end{minted}
  \end{teilaufgabe}
  \begin{teilaufgabe}
    \begin{minted}{SQL}
SELECT DISTINCT N, WS
  FROM Held
  WHERE N LIKE "Aqua%" OR N LIKE "%man";
    \end{minted}
  \end{teilaufgabe}
  \begin{teilaufgabe}
    \begin{enumerate}[(a)]
      \item \begin{minted}{SQL}
SELECT HID, BID
  FROM
    Held JOIN Beschuetzer
    ON Held.N = Beschuetzer.N;
        \end{minted}
      \item \begin{minted}{SQL}
SELECT HeldID, BesID
  FROM
    Held NATURAL JOIN Beschuetzer;
        \end{minted}
    \end{enumerate}
  \end{teilaufgabe}
  \begin{teilaufgabe}
    \begin{minted}{SQL}
SELECT N2
  FROM
    ((SELECT N AS N2, HID, WS, WP
      FROM H)
    JOIN
    (SELECT *
      FROM V
      WHERE SN>50)
    ON WP = VID)
    JOIN
    (SELECT *
      FROM B
      WHERE VS>=25
      AND VS<=75)
    ON BS = VID;
    \end{minted}
  \end{teilaufgabe}
  \begin{teilaufgabe}
    \begin{minted}{SQL}
SELECT VID, O
  FROM
    (SELECT * FROM V
      EXCEPT
    SELECT VID, O, SN, NE
      FROM
        V JOIN B
        ON VID = BS);
    \end{minted}
  \end{teilaufgabe}
  \begin{teilaufgabe}
    \begin{minted}{SQL}
SELECT N, S, T
  FROM
    (SELECT N, WS AS S, "Held:in" as T
      FROM H
    UNION
    SELECT N, VS AS S, "Beschuetzer:in" as T
      FROM B
    );
    \end{minted}
  \end{teilaufgabe}
  \begin{teilaufgabe}
    \begin{minted}{SQL}
SELECT VID
  FROM B, V
  WHERE BS = VID
  GROUP BY VID
  HAVING COUNT(BS) = 3;
    \end{minted}
  \end{teilaufgabe}
  \begin{teilaufgabe}
    \begin{minted}{SQL}
SELECT VID, O, COUNT(BID), SUM(VS)
  FROM V
  JOIN B ON BS = VID
  GROUP BY VID
  HAVING SUM(VS) < 15
  LIMIT 2;
    \end{minted}
  \end{teilaufgabe}
  \begin{teilaufgabe}
    \begin{minted}{SQL}
SELECT CAST(summe AS REAL) / anzahl
  FROM (SELECT sum(max_vs) as summe
  FROM (
    SELECT MAX(VS) AS max_vs
      FROM V
      JOIN B ON BS = VID
      GROUP BY VID
    )), (SELECT count(VID) as anzahl
      FROM V);
    \end{minted}
    Indem man den Durchschnitt ``Zu Fuß'' (mit summe/anzahl) rechnet, anstatt mit
    \mintinline{SQL}{avg}, rechnet man die Verliese ohne Beschützer:innen als 0 mit.
  \end{teilaufgabe}
\end{aufgabe}
\begin{aufgabe}
  \begin{teilaufgabe}
    Die Namen von Helden, die mehr als 35 Waffenstärke haben.

    \begin{tabular}{|l|}
      \hline
      \thead{Name} \\
      \hline
      Merlin       \\
      \hline
      Percy        \\
      \hline
    \end{tabular}
  \end{teilaufgabe}
  \begin{teilaufgabe}
    Die Verlies ID und die Anzahl der Beschützer:innen, die die Verliese beschützen,
    die echt mehr als 2 Beschützer:innen haben.

    \begin{tabular}{|l|l|}
      \hline
      \thead{VerID} & \thead{COUNT(Beschuetzer.BesID)} \\
      \hline
    \end{tabular}
  \end{teilaufgabe}
  \begin{teilaufgabe}
    Die Orte von Verliesen, die die zweitstärksten Beschützer:innen beschützen.

    \begin{tabular}{|l|}
      \hline
      \thead{Ort} \\
      \hline
      Alcatraz    \\
      \hline
    \end{tabular}
  \end{teilaufgabe}
\end{aufgabe}
\end{document}
