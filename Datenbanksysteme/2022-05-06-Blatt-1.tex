\documentclass[a4paper,12pt]{article}
\usepackage[utf8]{inputenc}
\usepackage[ngerman]{babel}
\usepackage[top=1in, bottom=1.25in, left=1.25in, right=1.25in]{geometry}
\usepackage{fancyhdr}
\usepackage{titling}
\usepackage{amssymb}
\usepackage{mathtools}


\renewcommand{\footrulewidth}{0.4pt}

\setlength\headheight{15pt}
\setlength{\parskip}{1em}

\title{Datenbanksysteme Blatt 1}
\author{
  Eli Kogan-Wang, elikoga, 7251030
  \and
  Niklas Bäumker, niba, 7165553
}
\date{\today}

\pagestyle{fancy}
\fancyhf{}
\lhead{\thetitle}
\rhead{\thedate}
\lfoot{\theauthor}
\rfoot{Seite \thepage}

\newcounter{AufgabenCounter}
\setcounter{AufgabenCounter}{1}
\newcounter{TeilaufgabenCounter}
\newenvironment{aufgabe}{\section*{Aufgabe \theAufgabenCounter}\setcounter{TeilaufgabenCounter}{1}}{\stepcounter{AufgabenCounter}}
% \newenvironment{teilaufgabe}{\paragraph*{\alph{TeilaufgabenCounter})}}{\stepcounter{TeilaufgabenCounter}}
\newenvironment{teilaufgabe}{\paragraph*{\theTeilaufgabenCounter.}}{\stepcounter{TeilaufgabenCounter}}


\begin{document}
\maketitle
\thispagestyle{fancy}

\begin{aufgabe}
  \begin{teilaufgabe}
    Geben Sie die Namen aller Held:innen aus.

    $\pi_{\mathrm{Name}}(\mathrm{Held})$
  \end{teilaufgabe}
  \begin{teilaufgabe}
    Geben Sie die IDs aller Beschützer:innen aus, die eine Verteidigungsstärke von weniger
    als 100 haben.

    $\pi_{\mathrm{BesID}}(\sigma_{\mathrm{VertStaerke}<100}(\mathrm{Beschuetzer}))$
  \end{teilaufgabe}
  \begin{teilaufgabe}
    Geben Sie die IDs aller Beschützer:innen aus, die ein Verlies mit einem Schatzniveau
    von weniger als 10 beschützen.

    $\pi_{\mathrm{BesID}}(
      \sigma_{\mathrm{SchatzNiveau}<10}(
      \mathrm{Beschuetzer}\bowtie_{
        \mathrm{Beschuetzt}=\mathrm{VerID}
      }\mathrm{Verlies}
      )
      )$
  \end{teilaufgabe}
  \begin{teilaufgabe}
    Geben Sie die Namen aller Held:innen aus, die keine besondere Stärke gegen irgendwel-
    che Beschützer:innen haben.\\
    \textbf{Hinweis}: Es darf ausgenutzt werden, dass die Namen der Held:innen sowie der Be-
    schützer:innen eindeutig sind.

    $\pi_{\mathrm{Name}}(\mathrm{Held}\bowtie_{
        \mathrm{Beschuetzt}=\mathrm{VerID}
      }\mathrm{Verlies}
      )$

  \end{teilaufgabe}
\end{aufgabe}

\end{document}
