\documentclass[a4paper,12pt]{article}
\usepackage[utf8]{inputenc}
\usepackage[ngerman]{babel}
\usepackage[top=1in, bottom=1.25in, left=1.25in, right=1.25in]{geometry}
\usepackage{fancyhdr}
\usepackage{titling}
\usepackage{amssymb}
\usepackage{mathtools}
\usepackage{enumerate}
\usepackage[shortlabels]{enumitem}
\usepackage{makecell}

\renewcommand\theadfont{}

\renewcommand{\footrulewidth}{0.4pt}

\setlength\headheight{15pt}
\setlength{\parskip}{1em}

\title{Datenbanksysteme Blatt 1}
\author{
  Eli Kogan-Wang, elikoga, 7251030
  \and
  Niklas Bäumker, niba, 7165553
}
\date{\today}

\pagestyle{fancy}
\fancyhf{}
\lhead{\thetitle}
\rhead{\thedate}
\lfoot{\theauthor}
\rfoot{Seite \thepage}

\newcounter{AufgabenCounter}
\setcounter{AufgabenCounter}{1}
\newcounter{TeilaufgabenCounter}
\newenvironment{aufgabe}{\section*{Aufgabe \theAufgabenCounter}\setcounter{TeilaufgabenCounter}{1}}{\stepcounter{AufgabenCounter}}
% \newenvironment{teilaufgabe}{\paragraph*{\alph{TeilaufgabenCounter})}}{\stepcounter{TeilaufgabenCounter}}
\newenvironment{teilaufgabe}{\paragraph*{\theTeilaufgabenCounter.}}{\stepcounter{TeilaufgabenCounter}}


\begin{document}
\maketitle
\thispagestyle{fancy}

\begin{aufgabe}
  \begin{teilaufgabe}
    Geben Sie die Namen aller Held:innen aus.

    $\pi_{\mathrm{Name}}(\mathrm{Held})$
  \end{teilaufgabe}
  \begin{teilaufgabe}
    Geben Sie die IDs aller Beschützer:innen aus, die eine Verteidigungsstärke von weniger
    als 100 haben.

    $\pi_{\mathrm{BesID}}(\sigma_{\mathrm{VertStaerke}<100}(\mathrm{Beschuetzer}))$
  \end{teilaufgabe}
  \begin{teilaufgabe}
    Geben Sie die IDs aller Beschützer:innen aus, die ein Verlies mit einem Schatzniveau
    von weniger als 10 beschützen.

    $\pi_{\mathrm{BesID}}(
      \sigma_{\mathrm{Schatzniveau}<10}(
      \mathrm{Beschuetzer}\bowtie_{
        \mathrm{Beschuetzt}=\mathrm{VerID}
      }\mathrm{Verlies}
      )
      )$
  \end{teilaufgabe}
  \begin{teilaufgabe}
    Geben Sie die Namen aller Held:innen aus, die keine besondere Stärke gegen irgendwelche
    Beschützer:innen haben.\\
    \textbf{Hinweis}: Es darf ausgenutzt werden, dass die Namen der Held:innen sowie der Be-
    schützer:innen eindeutig sind.

    Im Text steht: ``Sie enthält auch
    Held:innen die noch kein Verlies plündern wollen und/oder keine besondere Stärke gegen eine
    Beschützer:in haben.'' Das wird interpretiert als: `WS' und `SG' können NULL enthalten. Selbiges
    gilt auch für das Feld `BS'.

    Es werden Anfragen geschrieben, die vermeiden `NULL' zu verwenden, da in den Beispieldaten der Wert `-' angegeben ist.

    $$\begin{aligned}
         & \pi_{\mathrm{HeldName}}(\beta_{[\mathrm{HeldName}\gets \mathrm{Name}]}(\mathrm{Held})) \\
         & -\pi_{\mathrm{HeldName}}(
        (
        \beta_{[\mathrm{HeldName}\gets \mathrm{Name}]}(\mathrm{Held})
        )\bowtie_{\mathrm{StarkGegen}=\mathrm{BesID}}(
        \mathrm{Beschuetzer}
        )
        )
      \end{aligned}$$
  \end{teilaufgabe}
  \begin{teilaufgabe}
    Geben Sie die IDs aller Verliese aus,
    die von mindestens zwei Held:innen geplündert
    werden.

    $$\begin{aligned}
         & \pi_{\mathrm{VerID}}(                                                                                                                                                 \\
         & \quad(\beta_{[\mathrm{HeldIDA}\gets\mathrm{HeldID},\mathrm{WillPluendernA}\gets\mathrm{WillPluendern}]}(\pi_{\mathrm{HeldID},\mathrm{WillPluendern}}(\mathrm{Held}))) \\
         & \quad\quad\bowtie_{\mathrm{HeldIDA}\neq\mathrm{HeldID}\land\mathrm{WillPluendernA}=\mathrm{WillPluendern}}                                                            \\
         & \quad(\pi_{\mathrm{HeldID},\mathrm{WillPluendern}}(\mathrm{Held}))                                                                                                    \\
         & )                                                                                                                                                                     \\
      \end{aligned}$$
  \end{teilaufgabe}
  \begin{teilaufgabe}
    Geben Sie für alle Held:innen, bei denen mindestens eine der folgenden Bedingungen
    erfüllt ist, die Namen und deren Waffenstärke aus.

    \begin{enumerate}[(a)]
      \item Die Waffenstärke ist geringer als 20.
            $\pi_{\mathrm{Name},\mathrm{WaffenStaerke}}(\sigma_{\mathrm{WaffenStaerke}<20}(\mathrm{Held}))$
      \item Wollen ein Verlies plündern, welches ein Schatzniveau von mehr als 80 besitzt.
            $$\begin{aligned}
                 & \pi_{\mathrm{Name},\mathrm{WaffenStaerke}}(                                            \\
                 & \quad(\mathrm{Held})                                                                   \\
                 & \quad\quad\bowtie_{\mathrm{SchatzNiveau}>80\land\mathrm{WillPluendern}=\mathrm{VerID}} \\
                 & \quad(\mathrm{Verlies})                                                                \\
                 & )                                                                                      \\
              \end{aligned}$$
    \end{enumerate}

    a) $\cup$ b):
    $$
      \begin{aligned}
         & \pi_{\mathrm{Name},\mathrm{WaffenStaerke}}(                                                          \\
         & \quad(\mathrm{Held})                                                                                 \\
         & \quad\quad\bowtie_{\mathrm{SchatzNiveau}>80\land\mathrm{WillPluendern}=\mathrm{VerID}}               \\
         & \quad(\mathrm{Verlies})                                                                              \\
         & ) \cup \pi_{\mathrm{Name},\mathrm{WaffenStaerke}}(\sigma_{\mathrm{WaffenStaerke}<20}(\mathrm{Held})) \\
      \end{aligned}$$
  \end{teilaufgabe}
\end{aufgabe}

\begin{aufgabe}
  \begin{teilaufgabe}
    $$R\cup (S\cap T)=(R\cup S)\cap (R\cup T)$$

    Sei $u$ ein Tupel, dass $r$-mal in $R$ vorkommt, $s$-mal in $S$ vorkommt und $t$-mal in $T$ vorkommt.

    Das Tupel kommt $\min(s,t)$-mal in $S\cap T$ vor. Damit kommt das Tupel $r+\min(s,t)$-mal in $R\cup (S\cap T)$ vor.

    Das Tupel kommt $r+s$-mal in $R\cup S$ vor. Das Tupel kommt $r+t$-mal in $R\cup T$ vor. Damit kommt das Tupel $\min(r+s,r+t)$-mal in $(R\cup S)\cap (R\cup T)$ vor.

    \textbf{Bemerkung}: $\min(r+s,r+t)=r+\min(s,t)$

    Das algebraische Gesetz gilt auch für Bags.
  \end{teilaufgabe}
  \begin{teilaufgabe}
    $$(R\cup S)-(R\cap S)=(R-S)\cup (S-R)$$

    Sei $t$ ein Tupel, dass $r$-mal in $R$ vorkommt, $s$-mal in $S$ vorkommt.

    Also $t^r\in R$ und $t^s\in S$.

    Nun: $t^{r+s}\in (R\cup S)$ und $t^{\min(r,s)}\in (R\cap S)$.

    Daraus: $t^{\max(0,(r+s)-\min(r,s))}\in (R\cup S)-(R\cap S)$.

    Auch: $t^{\max(0,r-s)}\in (R-S)$ und $t^{\max(0,s-r)}\in (S-R)$.

    Daraus: $t^{\max(0,r-s)+\max(0,s-r)}\in (R-S)\cup (S-R)$.

    \OE: Ohne Einschränkung der Allgemeinheit (da r,s symetrie) wird angenommen, dass $r\geq s$.

    Nun: $\max(0,(r+s)-\min(r,s))=\max(0,(r+s)-s)=\max(0,r)=r$.

    Und: $\max(0,r-s)+\max(0,s-r)=r-s+0=r-s$.

    \textbf{Bemerkung}: $r\neq r-s$

    Das algebraische Gesetz gilt demnach nicht für Bags.

    \textbf{Gegenbeispiel}:

    R:
    \begin{tabular}{|l|}
      \hline
      \thead{A} \\
      \hline
      1         \\
      \hline
    \end{tabular}

    S:
    \begin{tabular}{|l|}
      \hline
      \thead{A} \\
      \hline
      1         \\
      \hline
    \end{tabular}

    (R$\cup$S):
    \begin{tabular}{|l|}
      \hline
      \thead{A} \\
      \hline
      1         \\
      \hline
      1         \\
      \hline
    \end{tabular}

    (R$\cap$S):
    \begin{tabular}{|l|}
      \hline
      \thead{A} \\
      \hline
      1         \\
      \hline
    \end{tabular}

    (R$\cup$S)-(R$\cap$S):
    \begin{tabular}{|l|}
      \hline
      \thead{A} \\
      \hline
      1         \\
      \hline
    \end{tabular}

    (R-S):
    \begin{tabular}{|l|}
      \hline
      \thead{A} \\
      \hline
    \end{tabular}

    (S-R):
    \begin{tabular}{|l|}
      \hline
      \thead{A} \\
      \hline
    \end{tabular}

    (R-S)$\cup$(S-R):
    \begin{tabular}{|l|}
      \hline
      \thead{A} \\
      \hline
    \end{tabular}

    \begin{tabular}{|l|}
      \hline
      \thead{A} \\
      \hline
    \end{tabular} $\neq$ \begin{tabular}{|l|}
      \hline
      \thead{A} \\
      \hline
      1         \\
      \hline
    \end{tabular}

  \end{teilaufgabe}

\end{aufgabe}

\begin{aufgabe}
  \begin{teilaufgabe}
    Held (H):\\
    \begin{tabular}{|l|l|l|l|l|}
      \hline
      % HeldID (HID) & WillPluendern (WP) & StarkGegen (SG) & Name (N) & Waffenstärke (WS) \\
      \thead{HeldID                   \\(HID)} & \thead{WillPluendern\\(WP)} & \thead{StarkGegen\\(SG)} & \thead{Name\\(N)} & \thead{Waffenstärke\\(WS)} \\
      \hline
      1 & 1 & 1 & \textbf{Held 1} & 1 \\
      \hline
    \end{tabular}

    Beschuetzer (B):\\
    % BesID Beschuetzt Name VertStaerke
    % (BID) (B) (N) (VS)
    \begin{tabular}{|l|l|l|l|l|}
      \hline
      \thead{BesID                 \\(BID)} & \thead{Beschuetzt\\(BS)} & \thead{Name\\(N)} & \thead{VertStaerke\\(VS)} \\
      \hline
      1 & 1 & \textbf{B...r 1} & 1 \\
      \hline
    \end{tabular}
    % \begin{enumerate}[a)]
    \begin{enumerate}[wide, label=\alph*)]
      \item Held $\times$ Beschuetzer\\
            \tiny
            \begin{tabular}{|l|l|l|l|l|l|l|l|l|l|}
              \hline
              \thead{HeldID                                                  \\(HID)} & \thead{WillPluendern\\(WP)} & \thead{StarkGegen\\(SG)} & \thead{H.Name\\(H.N)} & \thead{Waffenstärke\\(WS)}& \thead{BesID \\(BID)} & \thead{Beschuetzt\\(BS)} & \thead{B.Name\\(B.N)} & \thead{VertStaerke\\(VS)} \\
              \hline
              1 & 1 & 1 & \textbf{Held 1} & 1 & 1 & \textbf{B...r 1} & 1 & 1 \\
              \hline
            \end{tabular}
            \normalsize
      \item Held $\bowtie$ Beschuetzer\\
            \tiny
            \begin{tabular}{|l|l|l|l|l|l|l|l|l|l|}
              \hline
              \thead{HeldID \\(HID)} & \thead{WillPluendern\\(WP)} & \thead{StarkGegen\\(SG)} & \thead{Name\\(N)} & \thead{Waffenstärke\\(WS)}& \thead{BesID \\(BID)} & \thead{Beschuetzt\\(BS)} & \thead{VertStaerke\\(VS)} \\
              \hline
              \hline
            \end{tabular}
    \end{enumerate}

    \textbf{Bemerkung}: Definition verallgemeinert, damit Algebra-Aufgaben einfacher lösbar werden
  \end{teilaufgabe}
  \begin{teilaufgabe}
    Beide Anfragen geben alle `Verlies', `Held' Kombinationen zurück.
  \end{teilaufgabe}
  \begin{teilaufgabe}
    %   Verlies (V):
    % VerID NaechsteEbene Ort SchatzNiveau
    % (VID) (NE) (O) (SN)
    Verlies (V):\\
    \begin{tabular}{|l|l|l|l|}
      \hline
      \thead{VerID   \\(VID)} & \thead{NaechsteEbene\\(NE)} & \thead{Ort\\(O)} & \thead{SchatzNiveau\\(SN)} \\
      \hline
      1 & 1 & 1 & 1  \\
      \hline
      2 & 1 & 1 & 13 \\
      \hline
    \end{tabular}

    \begin{enumerate}[a)]
      \item Verlies $\ltimes$ ($\sigma_{\text{Schatzniveau>12}}$(Verlies))\\
            \begin{tabular}{|l|l|l|l|}
              \hline
              \thead{VerID   \\(VID)} & \thead{NaechsteEbene\\(NE)} & \thead{Ort\\(O)} & \thead{SchatzNiveau\\(SN)} \\
              \hline
              1 & 1 & 1 & 1  \\
              \hline
              2 & 1 & 1 & 13 \\
              \hline
            \end{tabular}
      \item $\sigma_{\text{Schatzniveau>12}}$(Verlies)\\
            \begin{tabular}{|l|l|l|l|}
              \hline
              \thead{VerID   \\(VID)} & \thead{NaechsteEbene\\(NE)} & \thead{Ort\\(O)} & \thead{SchatzNiveau\\(SN)} \\
              \hline
              2 & 1 & 1 & 13 \\
              \hline
            \end{tabular}
    \end{enumerate}
  \end{teilaufgabe}
  \begin{teilaufgabe}
    Alle Beschützer-Verlies Paare, bei denen der Beschützer das Verlies beschützt, und das Verlies eine Verteidigungsstärke kleiner als 20 hat.
  \end{teilaufgabe}
\end{aufgabe}
\begin{aufgabe}
  \begin{teilaufgabe}
    \begin{tabular}{|l|}
      \hline
      \thead{Name \\(N)} \\
      \hline
      Peter       \\
      \hline
    \end{tabular}

  \end{teilaufgabe}
  \begin{teilaufgabe}
    \begin{tabular}{|l|}
      \hline
      ID \\
      \hline
      23 \\
      \hline
    \end{tabular}
  \end{teilaufgabe}
  \begin{teilaufgabe}
    Es findet ein Vergleich zwischen `VertStaerke'-Inhalten (Integer) und `fuenfzig' (Text) statt.
    Das Ergebnis ist für uns undefiniert.
  \end{teilaufgabe}
  \begin{teilaufgabe}
    \begin{tabular}{|l|l|}
      \hline
      HeldID & WillPluendern \\
      \hline
      \hline
    \end{tabular}
  \end{teilaufgabe}
\end{aufgabe}

\end{document}
