\documentclass[a4paper,12pt]{article}
\usepackage[utf8]{inputenc}
\usepackage[ngerman]{babel}
\usepackage[top=1in, bottom=1.25in, left=1.25in, right=1.25in]{geometry}
\usepackage{minted}
\usepackage{blindtext}
\usepackage{fancyhdr}
\usepackage{titling}
\usepackage{amssymb}
\usepackage{mathtools}
\usepackage{systeme}


\renewcommand{\footrulewidth}{0.4pt}

\setlength\headheight{15pt}
\setlength{\parskip}{1em}

\title{Document Template}
\author{Eli Kogan-Wang}
\date{\today}

\pagestyle{fancy}
\fancyhf{}
\lhead{\thetitle}
\rhead{\thedate}
\lfoot{\theauthor}
\rfoot{Page \thepage}


\begin{document}
% \maketitle
% \thispagestyle{fancy}
\renewcommand{\abstractname}{Abstract}
\begin{abstract}
  This is partial notes for linal 2021-10-12.
\end{abstract}
\section{Lineare Gleichungssysteme}

\paragraph*{Beispiel}

\[
  \systeme*{2x+3y=3,4x-y=1}
  \tag{*}
\]

\[
  \systeme*{2x+3y=3,4x-y=1}
  \tag{**}
\]

(1-te) Gleichung + $3\cdot$(2-te Gleichung) liefert

$(x,y)$ ist eine Lösung von $(*)$ genau dann wenn $(x,y)$ eine Lösung von $(**)$ ist.

$(**)$ hat eine einzige Lösung:

$$x=\frac{6}{14}=\frac{3}{7}$$
$$y=4\cdot\frac{3}{7}-1=\frac{12-7}{7}=\frac{5}{7}$$

\paragraph*{Fragen}

\begin{itemize}
  \item Ist jedes solches Gleichungssystem lösbar?
  \item Sind Lösungen immer eindeutig?
  \item Wie können wir alle Lösugen beschreiben?
  \item Was passiert wenn wir mehr Gleichungen als Unbekannte haben?
\end{itemize}

Lineare Gleichungssysteme


\[
  \systeme*{a_{1;1}x=b_1}
\]


\end{document}
