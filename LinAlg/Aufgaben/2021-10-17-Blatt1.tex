\documentclass[a4paper,12pt]{article}
\usepackage[utf8]{inputenc}
\usepackage[ngerman]{babel}
\usepackage[top=1in, bottom=1.25in, left=1.25in, right=1.25in]{geometry}
\usepackage{minted}
\usepackage{blindtext}
\usepackage{fancyhdr}
\usepackage{titling}
\usepackage{amssymb}
\usepackage{mathtools}
\usepackage{systeme}
\usepackage{tikz}
\usetikzlibrary{calc,tikzmark}


\renewcommand{\footrulewidth}{0.4pt}

\setlength\headheight{15pt}
\setlength{\parskip}{1em}

\title{Lineare Algebra 1, Blatt 1}
\author{
    Ben Krogmann
    \and
    Eli Kogan-Wang
}
\date{\today}

\pagestyle{fancy}
\fancyhf{}
\lhead{\thetitle}
\rhead{\thedate}
\lfoot{\theauthor}
\rfoot{Seite \thepage}


\begin{document}
\maketitle
\thispagestyle{fancy}

\paragraph{Aufgabe 1}
Nur $A\vec c$ ist wohldefiniert, da die anderen beiden Vektoren zu wenige Elemente haben, um ein Multiplikation ausführen zu können.

\[
    A\vec c=\begin{pmatrix}
        1  & -2 & 3 & 5  \\
        -2 & 1  & 0 & 7  \\
        0  & 2  & 4 & -1
    \end{pmatrix}\begin{pmatrix}
        1 \\2\\-1\\3
    \end{pmatrix}=\begin{pmatrix}
        1-4-3+15  \\
        -2+2+0+21 \\
        0+4-4-3
    \end{pmatrix}=\begin{pmatrix}
        9 \\21\\-3
    \end{pmatrix}
\]

\noindent\rule{\textwidth}{0.4pt}

\paragraph{Aufgabe 2} Zu zeigen:
$\forall A\in Mat_{m\times n}(\mathbb R),\vec x\in\mathbb R^n\text{ und }\lambda\in\mathbb R:A(\lambda\vec x)=\lambda(A\vec x)$

\begin{align*}
    A(\lambda\vec x) & = A\begin{pmatrix}\lambda x_1\\\dots\\\lambda x_n\end{pmatrix}                                                                           \\
                     & = \begin{pmatrix}a_{11}&\dots&a_{1n}\\\dots\\a_{m1}&\dots&a_{mn}\end{pmatrix}\begin{pmatrix}\lambda x_1\\\dots\\\lambda x_n\end{pmatrix} \\
                     & = \begin{pmatrix}a_{11}(\lambda x_1)&+\dots+&a_{1n}(\lambda x_n)\\\dots\\a_{m1}(\lambda x_1)&+\dots+&a_{mn}(\lambda x_n)\end{pmatrix}    \\
                     & = \begin{pmatrix}(a_{11}x_1)\lambda&+\dots+&(a_{1n}x_n)\lambda\\\dots\\(a_{m1}x_1)\lambda&+\dots+&(a_{mn}x_n)\lambda\end{pmatrix}        \\
                     & = \lambda\begin{pmatrix}a_{11}x_1&+\dots+&a_{1n}x_n\\\dots\\a_{m1}x_1&+\dots+&a_{mn}x_n\end{pmatrix}                                     \\
                     & = \lambda(A\vec x) \tag*{$\Box$}
\end{align*}

\clearpage

\paragraph{Aufgabe 3}

\subparagraph{(1)}
\[
    \begin{aligned}
                                       & \begin{pmatrix}1&-2&3&-4&|&4\\0&1&-1&1&|&-3\\1&3&0&-3&|&1\\0&-7&3&1&|&-3\end{pmatrix}  \\
        \overset{III-I,IV+7II}\leadsto & \begin{pmatrix}1&-2&3&-4&|&4\\0&1&-1&1&|&-3\\0&5&-3&1&|&-3\\0&0&-4&8&|&18\end{pmatrix} \\
        \overset{III-5II}\leadsto      & \begin{pmatrix}1&-2&3&-4&|&4\\0&1&-1&1&|&-3\\0&0&2&-4&|&12\\0&0&-4&8&|&18\end{pmatrix} \\
        \overset{IV+2III}\leadsto      & \begin{pmatrix}1&-2&3&-4&|&4\\0&1&-1&1&|&-3\\0&0&2&-4&|&12\\0&0&0&0&|&42\end{pmatrix}
    \end{aligned}
\]

Da $0x_1+0x_2+0x_3+0x_4=0\not=42$, ist die Lösungsmenge des Gleichungssystems (1) die leere Menge.

\noindent\rule{\textwidth}{0.4pt}

\subparagraph{(2)}

\[
    \begin{aligned}
                                            & \begin{pmatrix}1&-2&1&1&|&1\\1&-2&1&-1&|&-1\\1&-2&1&5&|&5\end{pmatrix} \\
        \overset{II-I,III-I}\leadsto        & \begin{pmatrix}1&-2&1&1&|&1\\0&0&0&-2&|&-2\\0&0&0&4&|&4\end{pmatrix}   \\
        \overset{II\cdot(-\frac12)}\leadsto & \begin{pmatrix}1&-2&1&1&|&1\\0&0&0&1&|&1\\0&0&0&4&|&4\end{pmatrix}     \\
        \overset{I-II,III-4II}\leadsto      & \begin{pmatrix}1&-2&1&0&|&0\\0&0&0&1&|&1\\0&0&0&0&|&0\end{pmatrix}
    \end{aligned}
\]

\[
    \begin{pmatrix}
        x_1 \\x_2\\x_3\\x_4
    \end{pmatrix}=\begin{pmatrix}
        2x_2-x_3 \\\frac12x_1+\frac12x_3\\-x_1+2x_2\\1
    \end{pmatrix}=\begin{pmatrix}
        0 \\0\\0\\1
    \end{pmatrix}+x_1\begin{pmatrix}
        0 \\\frac12\\-1\\0
    \end{pmatrix}+x_2\begin{pmatrix}
        2 \\0\\2\\0
    \end{pmatrix}+x_3\begin{pmatrix}
        -1 \\\frac12\\0\\0
    \end{pmatrix}
\]

Die Lösungsmenge von $(2)$ ist \[
    \begin{pmatrix}
        0 \\0\\0\\1
    \end{pmatrix}+\alpha\begin{pmatrix}
        0 \\\frac12\\-1\\0
    \end{pmatrix}+\beta\begin{pmatrix}
        2 \\0\\2\\0
    \end{pmatrix}+\gamma\begin{pmatrix}
        -1 \\\frac12\\0\\0
    \end{pmatrix}\]

mit $\alpha,\beta,\gamma\in\mathbb R$.

\emph{Test: }$\alpha=0,\beta=0,\gamma=0$

\[
    \begin{pmatrix}
        x_1 \\x_2\\x_3\\x_4
    \end{pmatrix}=\begin{pmatrix}
        0 \\0\\0\\1
    \end{pmatrix}\implies\systeme{
        0+0+0+1=1,
        0+0+0-1=-1,
        0+0+0+5=5
    }
\]

\noindent\rule{\textwidth}{0.4pt}

\subparagraph{(3)}

\[
    \begin{aligned}
                                                                                 & \begin{pmatrix}1&3&5&7&|&12\\3&5&7&1&|&0\\5&7&1&3&|&4\\7&1&3&5&|&16\end{pmatrix}                       \\
        \overset{II-3I,III-5I,IV-7I}\leadsto                                     & \begin{pmatrix}1&3&5&7&|&12\\0&-4&-8&-20&|&-36\\0&-8&-24&-32&|&-56\\0&-20&-32&-44&|&-68\end{pmatrix}   \\
        \overset{II\cdot(-\frac14),III\cdot(-\frac18),IV\cdot(-\frac14)}\leadsto & \begin{pmatrix}1&3&5&7&|&12\\0&1&2&5&|&9\\0&1&3&4&|&7\\0&5&8&11&|&17\end{pmatrix}                      \\
        \overset{III-II,IV-5II}\leadsto                                          & \begin{pmatrix}1&3&5&7&|&12\\0&1&2&5&|&9\\0&0&1&-1&|&-2\\0&0&-7&-9&|&-18\end{pmatrix}                  \\
        \overset{IV+7III}\leadsto                                                & \begin{pmatrix}1&3&5&7&|&12\\0&1&2&5&|&9\\0&0&1&-1&|&-2\\0&0&0&-16&|&-32\end{pmatrix}                  \\
        \overset{IV\cdot(-\frac1{16})}\leadsto                                   & \begin{pmatrix}1&3&5&7&|&12\\0&1&2&5&|&9\\0&0&1&-1&|&-2\\0&0&0&1&|&2\end{pmatrix}                      \\
        \overset{I-7IV,II-5IV,III+IV}\leadsto                                    & \begin{pmatrix}1&3&5&0&|&-2\\0&1&2&0&|&-1\\0&0&1&0&|&0\\0&0&0&1&|&2\end{pmatrix}                       \\
        \overset{I-5III,II-2III}\leadsto                                         & \begin{pmatrix}1&3&0&0&|&-2\\0&1&0&0&|&-1\\0&0&1&0&|&0\\0&0&0&1&|&2\end{pmatrix}                       \\
        \overset{I-3II}\leadsto                                                  & \begin{pmatrix}1&0&0&0&|&1\\0&1&0&0&|&-1\\0&0&1&0&|&0\\0&0&0&1&|&2\end{pmatrix}                      &
    \end{aligned}
\]

\[
    \vec{x}=\begin{pmatrix}
        x_1 \\x_2\\x_3\\x_4
    \end{pmatrix}=\begin{pmatrix}
        1 \\-1\\0\\2
    \end{pmatrix}
\]

Die Lösungsmenge von $(3)$ ist $\left\{\vec{x}\right\}$

\emph{Test: }

\[
    \systeme{
        1-3+0+14=12,
        3-5+0+2=0,
        5-7+0+6=4,
        7-1+0+10=16
    }
\]

\clearpage

\subparagraph{(4)}
\[
    \begin{aligned}
                                                               & \begin{pmatrix}-6&8&-5&-1&|&9\\-2&4&7&3&|&1\\-3&5&4&2&|&3\\-3&7&17&7&|&\lambda\end{pmatrix}                                 \\
        \overset{II\cdot(-3),III\cdot(-2),IV\cdot(-2)}\leadsto & \begin{pmatrix}-6&8&-5&-1&|&9\\6&-12&-21&-9&|&-3\\6&-10&-8&-4&|&-6\\6&-14&-34&-14&|&-2\lambda\end{pmatrix}                  \\
        \overset{II+I,III+I,IV+I}\leadsto                      & \begin{pmatrix}-6&8&-5&-1&|&9\\0&-4&-26&-10&|&6\\0&-2&-13&-5&|&3\\0&-6&-39&-15&|&-2\lambda+9\end{pmatrix}                   \\
        \overset{II\leftrightarrow III}\leadsto                & \begin{pmatrix}-6&8&-5&-1&|&9\\0&-2&-13&-5&|&3\\0&-4&-26&-10&|&6\\0&-6&-39&-15&|&-2\lambda+9\end{pmatrix}                   \\
        \overset{III-2II,IV-3II}\leadsto                       & \begin{pmatrix}-6&8&-5&-1&|&9\\0&-2&-13&-5&|&3\\0&0&0&0&|&0\\0&0&0&0&|&-2\lambda\end{pmatrix}                               \\
        \overset{I\cdot(-\frac16),II\cdot(-\frac12)}\leadsto   & \begin{pmatrix}1&-\frac86&\frac56&\frac16&|&-\frac96\\0&1&\frac{13}2&\frac52&|&-\frac32\\0&0&0&0&|&-2\lambda\end{pmatrix}   \\
        \overset{I+\frac43II}\leadsto                          & \begin{pmatrix}1&0&\frac{57}6&\frac{21}6&|&-\frac{21}6\\0&1&\frac{13}2&\frac52&|&-\frac32\\0&0&0&0&|&-2\lambda\end{pmatrix}
    \end{aligned}
\]


\[
    \begin{pmatrix}
        x_1 \\x_2\\x_3\\x_4
    \end{pmatrix}=\begin{pmatrix}
        -\frac72-\frac{19}2x_3-\frac72x_4 \\-\frac32-\frac{13}2x_3-\frac52x_4\\x_3\\x_4
    \end{pmatrix}=\begin{pmatrix}
        -\frac72 \\-\frac32\\0\\0
    \end{pmatrix}+x_3\begin{pmatrix}
        -\frac{19}2 \\-\frac{13}2\\1\\0
    \end{pmatrix}+x_4\begin{pmatrix}
        -\frac72 \\-\frac52\\0\\1
    \end{pmatrix}
\]

Für $0=-2\lambda$ also $\lambda=0$ hat das System die Lösung:

\[
    \begin{pmatrix}
        x_1 \\x_2\\x_3\\x_4
    \end{pmatrix}=\begin{pmatrix}
        -\frac72 \\-\frac32\\0\\0
    \end{pmatrix}+\alpha\begin{pmatrix}
        -\frac{19}2 \\-\frac{13}2\\1\\0
    \end{pmatrix}+\beta\begin{pmatrix}
        -\frac72 \\-\frac52\\0\\1
    \end{pmatrix}\]
mit $\alpha,\beta\in\mathbb R$.

Für $\lambda\not=0$ hat das System keine Lösung.

\clearpage

\paragraph{Aufgabe 4}
$f(x)=ax^3+bx^2+cx+d$

$\begin{aligned}
                                                    & \begin{cases}a(-1)^3+b(-1)^2+c(-1)+d=0\\a0^3+b0^2+c0+d=1\\a1^3+b1^2+c1+d=2\\a2^3+b2^2+c2+d=4\end{cases} \\
        \leadsto                                    & \begin{cases}-a+b-c+d=0\\d=1\\a+b+c+d=2\\8a+4b+2c+d=4\end{cases}                                        \\
        \overset{I-II,III-II,IV-II}\leadsto         & \begin{cases}-a+b-c=-1\\a+b+c=1\\8a+4b+2c=3\\d=1\end{cases}                                             \\
        \overset{II+I,III+8I}\leadsto               & \begin{cases}-a+b-c=-1\\2b=0\\12b-6c=-5\\d=1\end{cases}                                                 \\
        \overset{I\cdot(-1),II\cdot\frac12}\leadsto & \begin{cases}a-b+c=1\\b=0\\12b-6c=-5\\d=1\end{cases}                                                    \\
        \overset{III-12II,I+II}\leadsto             & \begin{cases}a+c=1\\b=0\\-6c=-5\\d=1\end{cases}                                                         \\
        \overset{III\cdot(-\frac16)}\leadsto        & \begin{cases}a+c=1\\b=0\\c=\frac56\\d=1\end{cases}                                                      \\
        \overset{I-III}\leadsto                     & \begin{cases}a=\frac16\\b=0\\c=\frac56\\d=1\end{cases}
    \end{aligned}$

$\implies f(x)=\frac16x^3+\frac56x+1$

\clearpage

\paragraph{Aufgabe 5}

Durch Anwendung des Hinweises erhält man:

\[
    \left\{
    \begin{aligned}
        \log(x_1)  & + & 2\log(x_2) & + & 3\log(x_3) & = & \log(2) \\
        2\log(x_1) & + & 3\log(x_2) & + & 4\log(x_3) & = & \log(4) \\
        2\log(x_1) & + & \log(x_2)  & + & \log(x_3)  & = & \log(2) \\
    \end{aligned}
    \right.
\]

Dies ist ein lineares Gleichungssystem wobei die Unbekannten die Logarithmen der eigentlich
gesuchten Unbekannten sind.

Die vollständige Koeffizientenmatrix lautet wie folgt:

\[
    \begin{aligned}
                                              & \begin{pmatrix}
                                                    1 & 2 & 3 & | & \log(2) \\
                                                    2 & 3 & 4 & | & \log(4) \\
                                                    2 & 1 & 1 & | & \log(2) \\
                                                \end{pmatrix}              \\
        \overset{II-2I,II-2I}\leadsto         & \begin{pmatrix}
                                                    1 & 2  & 3  & | & \log(2)            \\
                                                    0 & -1 & -2 & | & \log(4) - 2\log(2) \\
                                                    0 & -3 & -5 & | & \log(2) - 2\log(2) \\
                                                \end{pmatrix} \\
        \leadsto                              & \begin{pmatrix}
                                                    1 & 2  & 3  & | & \log(2)   \\
                                                    0 & -1 & -2 & | & 0         \\
                                                    0 & -3 & -5 & | & - \log(2) \\
                                                \end{pmatrix}          \\
        \overset{II\cdot(-1),III-3II}\leadsto & \begin{pmatrix}
                                                    1 & 2 & 3 & | & \log(2)  \\
                                                    0 & 1 & 2 & | & 0        \\
                                                    0 & 0 & 1 & | & -\log(2) \\
                                                \end{pmatrix}             \\
        \overset{I-3III,II-2III}\leadsto      & \begin{pmatrix}
                                                    1 & 2 & 0 & | & \log(2)+3\log(2) \\
                                                    0 & 1 & 0 & | & 2\log(2)         \\
                                                    0 & 0 & 1 & | & -\log(2)         \\
                                                \end{pmatrix}     \\
        \overset{I-2I}\leadsto                & \begin{pmatrix}
                                                    1 & 0 & 0 & | & 0        \\
                                                    0 & 1 & 0 & | & 2\log(2) \\
                                                    0 & 0 & 1 & | & -\log(2) \\
                                                \end{pmatrix}             \\
    \end{aligned}
\]

Aus der letzten vollständigen Koeffizientenmatrix kann man so bestimmen:

\[
    \begin{aligned}
              & \log(x_1) & = & 0                & \iff & x_1=1           \\
        \land & \log(x_2) & = & 2\log(2)=\log(4) & \iff & x_2=4           \\
        \land & \log(x_3) & = & -\log(2)         & \iff & x_3=\frac{1}{2}
    \end{aligned}
\]

Damit sind $x_1=1$, $x_2=4$, $x_3=\frac{1}{2}$ die einzigen positiven reelen Zahlen für die
Angegebenes gilt.

\paragraph{Aufgabe 6}
\[
    A\vec a=\begin{pmatrix}
        1      & +2     & +3     & +4     & +\dots & +n     \\
        0      & +2     & +3     & +4     & +\dots & +n     \\
        0      & +0     & +3     & +4     & +\dots & +n     \\
        0      & +0     & +0     & +4     & +\dots & +n     \\
        \vdots & \vdots & \vdots & \vdots & \ddots & \vdots \\
        0      & +0     & +0     & +0     & +\dots & +n
    \end{pmatrix}=\begin{pmatrix}
        \sum_{k=1}^nk \\\sum_{k=2}^nk\\\sum_{k=3}^nk\\\sum_{k=4}^nk\\\vdots\\\sum_{k=1}^nk
    \end{pmatrix}
\]

\[
    A\vec b=\begin{pmatrix}
        1      & +2^3   & +3^3   & +4^3   & +\dots & +n^3   \\
        0      & +2^3   & +3^3   & +4^3   & +\dots & +n^3   \\
        0      & +0     & +3^3   & +4^3   & +\dots & +n^3   \\
        0      & +0     & +0     & +4^3   & +\dots & +n^3   \\
        \vdots & \vdots & \vdots & \vdots & \ddots & \vdots \\
        0      & +0     & +0     & +0     & +\dots & +n^3
    \end{pmatrix}=\begin{pmatrix}
        \sum_{k=1}^nk^3\\\sum_{k=2}^nk^3\\\sum_{k=3}^nk^3\\\sum_{k=4}^nk^3\\\vdots\\\sum_{k=1}^nk^3\end{pmatrix}
\]

\end{document}
