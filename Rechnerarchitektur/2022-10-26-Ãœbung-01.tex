\documentclass[a4paper,12pt]{article}
\usepackage[utf8]{inputenc}
\usepackage[T1]{fontenc}
\usepackage{lmodern}
\usepackage[ngerman]{babel}
\usepackage[top=1in, bottom=1.25in, left=1.25in, right=1.25in]{geometry}
\usepackage{minted}
\usepackage{blindtext}
\usepackage{fancyhdr}
\usepackage{titling}
\usepackage{amssymb}
\usepackage{mathtools}


\renewcommand{\footrulewidth}{0.4pt}

\setlength\headheight{15pt}
\setlength{\parskip}{1em}

\title{Rechnerarchitektur Übung 01}
\author{Eli Kogan-Wang}
\date{\today}

\pagestyle{fancy}
\fancyhf{}
\lhead{\thetitle}
\rhead{\thedate}
\lfoot{\theauthor}
\rfoot{Page \thepage}

\newcommand{\Aufgabe}[1]{
  {
  \vspace*{0.5cm}
  \textsf{\textbf{Aufgabe #1}}
  \vspace*{0.2cm}

  }
}

\setlength{\parindent}{0in}

\begin{document}
% \maketitle
% \thispagestyle{fancy}

\Aufgabe{1}

\newcommand{\kpd}{\text{Kosten-Pro-Die}\,}
\newcommand{\kpw}{\text{Kosten-Pro-Wafer}\,}
\newcommand{\dpw}{\text{Dies-Pro-Wafer}\,}
\newcommand{\Ausbeute}{\rm{Ausbeute}}
\newcommand{\wf}{\rm{Waferfläche}}
\newcommand{\df}{\rm{Diefläche}}
\newcommand{\dd}{\rm{Defektdichte}}


\tiny
$$\begin{aligned}
    \kpd
    =                     & \frac{\kpw}{\dpw\times \Ausbeute}                                                                                              \\
    =                     & \frac{\kpw}{\lparen\frac{\wf}{\df}\rparen\times \Ausbeute}                                                                     \\
    =                     & \frac{\kpw\times\df}{\wf\times \Ausbeute}                                                                                      \\
    =                     & \frac{\kpw\times\df}{\wf\times\lparen\frac{1}{\lparen1+\lparen\dd\times\df\times\frac{1}{\alpha}\rparen\rparen^\alpha}\rparen} \\
    \overset{\alpha=2}{=} & \frac{\kpw\times\df}{\wf\times\lparen\frac{1}{\lparen1+\lparen\dd\times\df\times\frac{1}{2}\rparen\rparen^2}\rparen}           \\
    =                     & \frac{\kpw\times\df\times \lparen1+\lparen\dd\times\df\times\frac{1}{2}\rparen\rparen^2}{\wf}                                  \\
    =                     & \frac{\kpw\times\df\times \lparen1+\lparen\dd\times\df\times\frac{1}{2}\rparen\rparen^2}{\wf}                                  \\
    =                     & \frac{\kpw\times\df\times \lparen1+2\times\dd\times\df\times\frac{1}{2}+\frac{1}{2^2}\times\dd^2\times\df^2\rparen}{\wf}       \\
    =                     & \frac{\kpw\times\df\times \lparen1+\dd\times\df+\frac{1}{4}\times\dd^2\times\df^2\rparen}{\wf}                                 \\
  \end{aligned}$$

\normalsize

Wir sehen $\kpd\in\mathcal{O}(\df^3)$.

(2):

Wir dürfen einen Faktor $\frac{\df}{\wf}$ mit $\frac{1}{165}$ ersetzen um \\
die Annäherung $\dpw=165$ umzukehren.

Nun: \\
$\kpw=\$\,1000$, \\
$\dd=1((10^{-2}m)^2)^{-1}$, \\
$\df=250(10^{-3}m)^2$

Also:

$$\begin{aligned}
    \frac{\$\,1000\times (1+1((10^{-2}m)^2)^{-1}\times250(10^{-3}m)^2+\frac{1}{4}\times(1((10^{-2}m)^2)^{-1})^2\times(250(10^{-3}m)^2)^2)}{165}
  \end{aligned}$$


$$\begin{aligned}
    1((10^{-2}m)^2)^{-1}\times250(10^{-3}m)^2 & =250((10^{-2}m)^{-2})(10^{-3}m)^2 \\
                                              & =250((10^4m^{-2})(10^{-6}m^2)     \\
                                              & =250(10^{-2})                     \\
                                              & =2,50                             \\
  \end{aligned}$$

$$\begin{aligned}
    (1((10^{-2}m)^2)^{-1}\times250(10^{-3}m)^2)^2 & =2,50^2 \\
                                                  & =6,25   \\
  \end{aligned}$$

Also:

$$\begin{aligned}
    \frac{\$\,1000\times (1+2,50+6,25)}{165} & =\$\,1000\times\frac{1+2,50+\frac{1}{4}\times 6,25}{165} \\
                                             & =\$\,1000\times\frac{1+4,0625}{165}                      \\
                                             & =\$\,1000\times\frac{5,0625}{165}                        \\
                                             & =\$\,30,6\overline{81}                                   \\
  \end{aligned}$$


\end{document}
