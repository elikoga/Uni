\documentclass[answers]{submit}
\homework

%TODO: Hier Gruppennummer, Namen, sowie Matrikelnummern einfügen.
\exercisegroup{27}{Eli Kogan-Wang (7251030) \\David Noah Stamm (7249709) \\ Bogdan Rerich (7248483) \\Jan Schreiber(7253698)}

% Folgendes braucht nicht geändert werden
\exercisetl{Universität Paderborn \\ Prof. Dr. Johannes Blömer}
\exerciselecture{Berechenbarkeit und Komplexität -- WS 2022/2023}
\exercisenumber{12}
\exercisehandin{30. Januar 2023 -- 13:00 Uhr}

%% Generic %%

% Following commands were often defined in documentclass
\newcommand{\Q}{\mathbb{Q}}
\newcommand{\N}{\mathbb{N}}
\newcommand{\Z}{\mathbb{Z}}
\newcommand{\R}{\mathbb{R}}
\newcommand{\cO}{\mathcal{O}}
\newcommand{\abs}[1]{\left\vert #1 \right\vert}
\newcommand{\norm}[1]{\left\Vert #1 \right\Vert}

\newcommand{\C}{\mathbb{C}}
%\DeclareMathOperator*{\argmax}{arg\,max}
%\DeclareMathOperator*{\argmin}{arg\,min}

\newcommand{\set}[2]{\left\{ #1 \ \middle| \ #2 \right\}}

\newcommand{\la}{\lambda}
\newcommand{\p}{{\textup P}}
\newcommand{\np}{{\textup NP}\xspace}
\newcommand{\conp}{{\textup coNP}}
\newcommand{\cst}{{\tt{cost}}}
\newcommand{\val}{{\tt{value}}}
\newcommand{\Bin}{{\{0,1\}}}


%%% BuK %%%
\newcommand{\qa}{q_{\mathrm{accept}}}
\newcommand{\qr}{q_{\mathrm{reject}}}
\newcommand{\dtm}[1][M]{\langle #1 \rangle}
\newcommand{\lep}{\le_{p}}

%% PSEUDOCODE %%%%
\newcommand{\alg}[1]{\texttt{#1}}

\newenvironment{pseudocode}[1]%
{\vspace*{-0.5\baselineskip} \upshape \begin{tabbing}
    99 \= bla \= bla \= bla \= bla \= bla \= bla \= bla \= bla \kill
    #1 \+ \\}%
    {\end{tabbing}}
\newcommand{\keyword}[1]{\texttt{#1}}
\newcommand{\IF}{\keyword{IF} }
\newcommand{\THEN}{\keyword{THEN} \+ \\}
\newcommand{\ELSE}{\< \keyword{ELSE} \\}
\newcommand{\END}{\< \keyword{END} \- \\ }
\newcommand{\OR}{\keyword{OR} }
\newcommand{\AND}{\keyword{AND} }
\newcommand{\RETURN}{\keyword{RETURN} }
\newcommand{\FOR}{\keyword{FOR} }
\newcommand{\WHILE}{\keyword{WHILE} }
\newcommand{\TO}{\keyword{TO} }
\newcommand{\DO}{\keyword{DO} \+ \\ }
\newcommand{\REPEAT}{\keyword{REPEAT} \+ \\ }
\newcommand{\UNTIL}{\< \keyword{UNTIL} \- }


\newcommand{\bestl}{{\tt Beste-L"osung}}
\newcommand{\lb}{{\tt untere-Schranke}}
\newcommand{\mcut}{{\textsc{MaxCut}}}
\newcommand{\opt}{{\textup opt}}
\newcommand{\dt}{{\textup{DTIME}}}
\newcommand{\sat}{{\textup{SAT}}}
\newcommand{\satd}{{\textup{3SAT}}}
\newcommand{\satz}{{\textup{2SAT}}}
\newcommand{\clique}{{\textup{Clique}}}
\newcommand{\ggt}{{\textrm{ggT}}}
\newcommand{\ol}{\overline}
\newcommand{\tm}{{\textrm T}_M}
\newcommand{\msb}{\textrm{MSB}}
\newcommand{\rsopt}{\textrm{RS}_{\textup opt}}
\newcommand{\rsent}{\textrm{RS}_{\textup ent}}
\newcommand{\rsval}{\textrm{RS}_{\textup val}}
\newcommand{\tspopt}{\textrm{TSP}_{\textup opt}}
\newcommand{\tspent}{\textrm{TSP}_{\textup ent}}
\newcommand{\etspent}{\textrm{ETSP}_{\textup ent}}

\newcommand{\tspval}{\textrm{TSP}_{\textup val}}
\newcommand{\etsp}{\textrm{ETSP}}
\newcommand{\nt}{{\textup{NTIME}}}
\newcommand{\tn}{{\textrm{T}_N}}
\newcommand{\ein}{{\textrm{Eintrag}}}
\newcommand{\phis}{\phi_{\textup Start}}
\newcommand{\phia}{\phi_{\textup{accept}}}
\newcommand{\phie}{\phi_{\textup Eintrag}}
\newcommand{\phim}{\phi_{\textup move}}
\newcommand{\susu}{{\textup SubsetSum}}
\newcommand{\ghkreis}{{\textup gH-Kreis}}
\newcommand{\uhkreis}{{\textup uH-Kreis}}
\newcommand{\eue}{{\textup Exakte-"Uberdeckung}}
\newcommand{\kue}{{\textrm{Knoten"uberdeckung}}}
\newcommand{\clopt}{{\textup Clique}_{\textup opt}}
\newcommand{\unabopt}{{\textup Unabh"angig}_{\textup opt}}
\newcommand{\zq}{\Z_q}
\newcommand{\god}{\textrm{G"odel}}
\newcommand{\mr}{M^{\textrm{reject}}}
\newcommand{\diag}{\textrm{Diag}}
\newcommand{\exo}{\textrm{XOR}}
\newcommand{\use}{\textrm{Useful}}
\newcommand{\pot}{{\cal P}(Q)}

\newcommand{\remu}{R(1,3)}
\newcommand{\remum}{R(1,n)}
\newcommand{\rsq}{RS(q,m,n)}

\newcommand{\ee}{{\textup e}}

\newcommand{\bigO}{{\cal O}}
\newcommand{\bit}{\{0,1\}}
\newcommand{\bits}{\{0,1\}^*}

\newcommand{\rd}{\rhd}
\newcommand{\gehtu}[1]{%
  \stackrel{#1}{\smash{\rule{0.15mm}{1ex}}\rule[0.5ex]{2em}{0.15mm}}}
%\newcommand{\Beweis}{{\textbf Beweis.} }
\newcommand{\bla}{\sqcup}
\newcommand{\ent}{\stackrel{\wedge}{=}}

\newcommand{\panda}{PANDA}\hyphenation{PANDA}
 % Diverse praktische Kommandos aus Vorlesungsskript und Übungen
\usepackage{enumerate}
\usepackage{algorithm2e}

\begin{document}
\exerciseheader
\exercisetitle
\newcommand{\cI}{\mathcal{I}}
\newcommand{\cP}{\mathcal{P}}
\newcommand{\cT}{\mathcal{T}}
\newcommand{\RO}{\mathfrak{O}_{\textsc{RS}}}
\newcommand{\RS}{\mathrm{RS_{opt}}}



\begin{exercise}[6]{Approximationsalgorithmen}
  Wir betrachten die Optimierungsvariante einer Einschränkung von \textsc{HittingSet} (siehe Heimübung~9 Aufgabe~2).
  Die Menge der Instanzen $\cI$ ist die Menge aller Paare $(X,S)$, wobei $X$ eine endliche Menge ist und $S\subseteq\cP(X)$, sodass für alle $T\in S: \abs{T}\leq 3$.
  Die Menge der gültigen Lösungen ist
  \[ F(I) = \{ H \mid H\in \cP(X)\mbox{ und für alle $T\in S: H\cap T \neq \emptyset$} \} \ . \]
  Für $H\in F(I)$ ist die Zielfunktion $w(H) = \abs{H}$.
  Es handelt sich um ein Minimierungsproblem.
  Betrachten Sie folgenden Approximationsalgorithmus.
  \begin{algorithm}
    $H \gets \emptyset$\;
    \For{$T\in S$}{
      \If{$T\cap H = \emptyset$}{
        $H \gets H\cup T$\;
      }
    }
    \Return{$H$}\;
    \caption{\textsc{kleinerSchnitt} $(X,S)$}
  \end{algorithm}

  Zeigen Sie, dass \textsc{kleinerSchnitt} eine Approximationsgüte von höchstens $3$ hat.

  \answer{ \\

    Da nach Aufgabenstellung bereits gilt, dass KleinerSchnitt (X,S) ein Approximationsalgorithmus ist,
    ist nur noch die Approximationsgüte zu Zeigen: \\

    Sei H die Ausgabe von kleinerSchnitt bei Eingabe einer Beliebigen, aber festen Instanz (X,S). \\

    \textbf{Behauptung:} w(H)= $\abs{H} \leq 3 \cdot \abs{O}$, wobei O die Optimale Lösungsmenge ist. \\

    Da vor jedem hinzufügen eines T $\in S$ geprüft wird, ob H $\cap$ T = $\emptyset$ gilt: \\

    H = $\overset{.}{\bigcup_{T \in S}}$ T  eine disjunkte Vereinigung von allen T $\in S$. \\

    Außerdem gilt:  $O \in F(I)$ gilt für alle T $\in$ S, dass $O \cap T \neq \emptyset $. \\

    Folglich gilt, dass (mindestens) ein Element aus jedem der T, welches zu H hinzugefügt werden, in O liegen muss.  \\

    $ \implies \frac{1}{3} \abs{H} \leq \abs {O}$, da für jede Teilmenge T $\in$ S gilt $\abs{T}$ = 3 gilt\\

    Also gilt Insgesamt: H $\leq$ 3 $\cdot \abs{O} \iff \frac{w(H)}{opt(H)} \leq 3 $ \\

    $\implies$ Kleiner Schnitt hat einen Approximationsfaktor von 3 $\qed$.
  }
\end{exercise}


\begin{exercise}[12]{Such- und Optimierungsvarianten}
  Betrachten Sie folgende, aus der Vorlesung bekannte, Sprache.
  \[ \textsc{Rucksack} = \left\{ \langle G,W,g,w\rangle \middle| \begin{array}{c} \mbox{$G = \{g_1,\dots,g_n\}$, $W = \{w_1,\dots,w_n\}$, $g,w\in\N$,} \\ \mbox{für $1\leq i\leq n$ ist $g_i,w_i\in\N$, und es} \\ \mbox{gibt $S\subseteq\{1,\dots,n\}$ mit $\sum_{i\in S} g_i \leq g$ und $\sum_{i\in S} w_i\geq w$.} \end{array} \right\} \]
  Gehen Sie im folgenden davon aus, dass Sie Zugriff auf ein \emph{Orakel} $\RO$ haben.
  Gegeben eine Instanz $I = \langle G,W,g,w\rangle$ kann Ihnen dieses Orakel \emph{in einem Zeitschritt} beantworten, ob $I$ in \textsc{Rucksack} liegt, oder nicht.

  Zeigen Sie:
  Eine DTM, die $\RO$ verwenden darf, kann
  \begin{enumerate}
    \item gegeben $\langle G,W,g,w\rangle\in\textsc{Rucksack}$ in polynomieller Zeit ein $S\subseteq \{1,\dots,n\}$ berechnen, sodass $\sum_{i\in S} g_i \leq g$ und $\sum_{i\in S} w_i\geq w$.
    \item gegeben $\langle G,W,g\rangle$ in \emph{polynomieller} Zeit ein $w\in \N$ berechnen, sodass $\langle G,W,g,w\rangle\in \textsc{Rucksack}$, aber $\langle G,W,g,(w+1)\rangle\not\in \textsc{Rucksack}$.
          Hierbei können Sie den Algorithmus aus Teilaufgabe 1 benutzen, müssen sich aber klar machen, was polynomiell in Bezug auf die Eingaben bedeutet.
  \end{enumerate}

  \answer{

    \begin{enumerate}
      \item Wir zeigen eine DTM $M$, die unter verwendung von $M$ ein $S\subseteq \{1,\dots,n\}$ berechnet, sodass $\sum_{i\in S} g_i \leq g$ und $\sum_{i\in S} w_i\geq w$.\\

            Bei Eingabe $\langle G,W,g,w\rangle\in\textsc{Rucksack}$ geht die DTM wie folgt vor:\\

            \begin{enumerate}
              \item $idx \gets 1$, $S \gets \{1,\dots,n\}$
              \item Solange $idx \leq n$: \label{while}
                    \begin{enumerate}
                      \item Prüfe, ob $\langle \{g_i|g\in S \setminus \{idx\}\}, \{w_i|w\in S \setminus \{idx\}\}, g, w\rangle \in \textsc{Rucksack}$ mit $\RO$:
                            \begin{enumerate}
                              \item Falls ja, dann $S \gets S \setminus \{idx\}$
                            \end{enumerate}
                      \item $idx \gets idx + 1$
                    \end{enumerate}
              \item Schreibe $S$ auf das Band
            \end{enumerate}

            Schleifeninvariante: Nach jeden Durchlauf von \ref{while} gilt:

            Für jedes $S'$, mit $S'=S\setminus \{n\}$ mit $n\in \{1,\dots,idx - 1\}$ gilt:

            $$\langle \{g_i|g\in S'\}, \{w_i|w\in S'\}, g, w\rangle \notin \textsc{Rucksack}$$

            Und $$\langle \{g_i|g\in S\}, \{w_i|w\in S\}, g, w\rangle \in \textsc{Rucksack}$$

            Nachdem $idx = n + 1$ gilt, ist $S$ eine solche gesuchte Menge, da es keine strikten
            Untermengen von $S$ gibt, die in $\textsc{Rucksack}$ lägen, aber $S$ selbst in $\textsc{Rucksack}$ liegt.

            Der Algorithmus geht $n$ mal durch die Schleife, und für jeden Durchlauf wird $\RO$ genau einmal aufgerufen.

            Der Algorithmus ist offensichtlich polynomiell in der Eingabe.

      \item Wir beschreiben eine DTM, die unter Verwendung von $M$ ein $w\in \N$ berechnet,
            sodass $\langle G,W,g,w\rangle\in \textsc{Rucksack}$,
            aber $\langle G,W,g,(w+1)\rangle\not\in \textsc{Rucksack}$.

            Bei Eingabe von $\langle G,W,g\rangle$ geht die DTM wie folgt vor:

            \begin{enumerate}
              \item $max_w \gets \sum_{i=1}^n w_i$
              \item Begehe binäre Suche auf $w\in\{1,\dots,max_w\}$, nach einem $w$ mit der gesuchten Eigenschaft
            \end{enumerate}

            Für die binäre Suche verwenden wir die DTM $M$ aus Teilaufgabe 1, mit 2 Aufrufen:

            Einmal mit $w$ und einmal mit $w+1$.

            Der Suchbereich ist $max_w$ groß, und $max_w\in O(2^n)$, da die Summe der Gewichte binär kodiert werden kann.

            Die binäre Suche aber benötigt $O(\log(max_w))=O(n)$ Schritte, sodass der Algorithmus polynomiell ist.
    \end{enumerate}
  }
\end{exercise}

\begin{exercise}[6]{Rekursive Aufzählbarkeit und Entscheidbarkeit}

  \begin{enumerate}[a)]
    \item \emph{Beweisen} Sie, ob folgende Sprache rekursiv aufzählbar oder, durch geeignete Reduktion, nicht rekursiv aufzählbar ist
          \[
            L_{\theenumi} = \set{\dtm[M]}{\begin{array}{l} \text{$M$ ist DTM und es gibt $w,w'\in\Bin^*$, sodass DTM $M$} \\
                \text{bei Eingabe $w$ hält und für die Konkatenation $ww'$ nicht hält.}
              \end{array}}
          \]
    \item \emph{Beweisen} Sie, ob folgende Sprache entscheidbar oder, durch geeignete Reduktion, nicht entscheidbar ist
          \[
            L_{\theenumi} = \set{\dtm[k,x,M_1,M_2,\dots,M_k]}{
              \begin{array}{l}
                \text{$k\in\N,x\in \Bin^*$, $M_i$ ist DTM für alle $1\le i \le k$}    \\
                \text{und es gibt $I\subseteq\{1,2,\dots,k\},\abs{I}\ge k/2$, sodass} \\
                \text{für alle $i\in I$ DTM $M_i$  bei Eingabe $x$ hält. }
              \end{array}}
          \]
  \end{enumerate}

  \answer{

    a) \\

    \textbf{Behauptung:} $H^c \leq L_a $

    Betrachte folgende Reduktionsfunktion:

    $$f(w)=\begin{cases}
        \langle M^{(x)} \rangle      & \text{wenn w} = \langle M \rangle \text{x, wobei }  \langle M \rangle \\
                                     & \text{die Kodierung einer DTM ist und x} \in \{0,1\}^*                \\
        \langle M^{\epsilon} \rangle & \text{ sonst}
      \end{cases}$$ \\

    Wobei $\langle M^{(x)} \rangle$ eine Turingmaschine ist, welche wie folgt definiert ist: \\

    $\langle M^{(x)}\rangle$ bei Eingabe z $\in \{0,1\}^*$
    \begin{enumerate}
      \item Berechne |z|. Falls |z|=0, so akzeptiere.
      \item Simuliere M bei Eingabe x für |z| Schritte.
      \item Falls M in |z| Schritten hält, so akzeptiere z.
      \item Sonst gehe in eine Endlosschleife.
    \end{enumerate}

    und $\langle M^{\epsilon} \rangle$ die Turingmaschine, welche nur das leere Wort akzeptiert und sonst in eine endlosschleife geht. \\

    Da die Einzige Änderung von $f(w)$ bezüglich der Turingmaschine in Satz 2.10.1 ist,
    dass hinzufügen einer Bedingung in (1), welche trivialerweise berechbar ist. Folglich ist f(w) auch berechbar. \\

    Angenommen w $\in H^c$ \\

    Entweder w $ \neq \langle M\rangle$x \\

    $\implies f(w)= \langle M^{\epsilon} \rangle$, welche in $L(L_a)$ liegt,da für $w = \epsilon$ und $w' \in \{0,1\}^+$, die DTM bei Eingabe w hält, aber nicht für $ww'$.

    oder w = $\langle M\rangle$x, wobei M eine DTM ist, welche bei Eingabe x nicht hält. \\

    $\implies$ f(w)= $\langle M^{(x)} \rangle$ \\

    Wähle w= $\epsilon$ (= das leere Wort) und ${w^*}'$ = x \\

    $\implies ww' = w' = x$. \\

    $\implies$ f(w) $\in L_a$, da nach (1) w von M $\in$ L(M) und nach (4) M bei Eingabe $w'$ in eine Endlosschleife geht. \\

    Angenommen w $\notin H^c$

    $\implies$ w = $\langle M\rangle$x, wobei M eine DTM ist, welche bei Eingabe x hält. \\

    $\implies$ M hält nach



    b) \\
    \textbf{Behauptung:} $H \leq L_b $

    Betrachte folgende Reduktionsfunktion:

    $$f(w)=\begin{cases}
        \langle 2,x,M,M\rangle           & \text{wenn w} = \langle M \rangle \text{x, wobei }  \langle M \rangle \\
                                         & \text{ die Kodierung einer DTM ist und x} \in \{0,1\}^*               \\
        \epsilon \text{(das leere Wort)} & \text{,wenn w} \neq \langle M \rangle \text{x sonst}
      \end{cases}$$ \\

    f ist trivialerweise berechenbar, da f nur die Reihenfolge der Eingabe verändert
    und eine Kodierung einer 2 und M auf das bannt kopiert, welches alles Operationen sind, von denen wir wissen,
    dass sie berechenbar sind. \\

    \textbf{Behauptung:} w $ \in H \iff$  f(w) $ \in L_b$ \\

    Angenommen w $\in$ H.

    $\implies$ w = $\langle M \rangle$x und M hält bei Eingabe x.

    $\implies$ f(w)= $\langle 2,x,M,M\rangle$. Da nach Voraussetzung beide Turingmaschinen bei Eingabe x halten.\\
    \\
    Folglich halten mehr als $\frac{2}{2}$=1 Turingmaschine, womit gilt: f(w) $\in L_b$ \\

    Angenommen w $\notin$ H. \\

    $\implies$ Entweder w $ \neq \langle M \rangle$x, wird damit auf $\epsilon$ abgebildet, welches nicht in $L_b$ liegt \\

    oder  w = $\langle M \rangle$x und M hält bei Eingabe x nicht.\\

    $\implies$ f(w)= $\langle 2,x,M,M\rangle$, aber da M nicht bei Eingabe x hält, hält keine der Turingmaschinen. \\

    $\implies$ f(w) $\notin L_b$

    $\implies $ w $ \in H \iff$  f(w) $ \in L_b$ \\

    $\implies$ w $ \in H \iff$  f(w) $ \in L_b \qed $ \\

    Da nach 2.9.2 gilt, dass H nicht entscheidbar ist, folgt mit 2.9.1, dass $L_b$ nicht entscheidbar ist.

  }

\end{exercise}


\end{document}
