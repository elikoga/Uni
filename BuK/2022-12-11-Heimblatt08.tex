\documentclass[answers]{submit}
\homework

%TODO: Hier Gruppennummer, Namen, sowie Matrikelnummern einfügen.
\exercisegroup{27}{Eli Kogan-Wang (7251030) \\David Noah Stamm (7249709) \\ Bogdan Rerich (7248483) \\Jan Schreiber(7253698)}

% Folgendes braucht nicht geändert werden
\exercisetl{Universität Paderborn \\ Prof. Dr. Johannes Blömer}
\exerciselecture{Berechenbarkeit und Komplexität -- WS 2022/2023}
\exercisenumber{8}
\exercisehandin{12. Dezember 2022 -- 13:00 Uhr}

%% Generic %%
\newcommand{\Q}{\mathbb{Q}}
\newcommand{\N}{\mathbb{N}}
\newcommand{\C}{\mathbb{C}}
\newcommand{\Z}{\mathbb{Z}}
\newcommand{\R}{\mathbb{R}}
\newcommand{\cO}{\mathcal{O}}
\newcommand{\abs}[1]{\left\vert #1 \right\vert}
\newcommand{\norm}[1]{\left\Vert #1 \right\Vert}
\DeclareMathOperator*{\argmax}{arg\,max}
\DeclareMathOperator*{\argmin}{arg\,min}

\newcommand{\la}{\lambda}
\newcommand{\p}{{\textup P}}
\newcommand{\np}{{\textup NP}}
\newcommand{\cst}{{\tt{cost}}}
\newcommand{\val}{{\tt{value}}}


%% PSEUDOCODE %%%%

\newenvironment{pseudocode}[1]%
{\vspace*{-0.5\baselineskip} \upshape \begin{tabbing}
    99 \= bla \= bla \= bla \= bla \= bla \= bla \= bla \= bla \kill
    #1 \+ \\}%
    {\end{tabbing}}
\newcommand{\keyword}[1]{\texttt{#1}}
\newcommand{\IF}{\keyword{IF} }
\newcommand{\THEN}{\keyword{THEN} \+ \\}
\newcommand{\ELSE}{\< \keyword{ELSE} \\}
\newcommand{\END}{\< \keyword{END} \- \\ }
\newcommand{\OR}{\keyword{OR} }
\newcommand{\AND}{\keyword{AND} }
\newcommand{\RETURN}{\keyword{RETURN} }
\newcommand{\FOR}{\keyword{FOR} }
\newcommand{\WHILE}{\keyword{WHILE} }
\newcommand{\TO}{\keyword{TO} }
\newcommand{\DO}{\keyword{DO} \+ \\ }
\newcommand{\REPEAT}{\keyword{REPEAT} \+ \\ }
\newcommand{\UNTIL}{\< \keyword{UNTIL} \- }


\newcommand{\bestl}{{\tt Beste-L"osung}}
\newcommand{\lb}{{\tt untere-Schranke}}
\newcommand{\mcut}{{\textsc{MaxCut}}}
\newcommand{\opt}{{\textup opt}}
\newcommand{\dt}{{\textup{DTIME}}}
\newcommand{\sat}{{\textup{SAT}}}
\newcommand{\satd}{{\textup{3SAT}}}
\newcommand{\satz}{{\textup{2SAT}}}
\newcommand{\clique}{{\textup{Clique}}}
\newcommand{\ggt}{{\textrm{ggT}}}
\newcommand{\ol}{\overline}
\newcommand{\tm}{{\textrm T}_M}
\newcommand{\msb}{\textrm{MSB}}
\newcommand{\rsopt}{\textrm{RS}_{\textup opt}}
\newcommand{\rsent}{\textrm{RS}_{\textup ent}}
\newcommand{\rsval}{\textrm{RS}_{\textup val}}
\newcommand{\tspopt}{\textrm{TSP}_{\textup opt}}
\newcommand{\tspent}{\textrm{TSP}_{\textup ent}}
\newcommand{\etspent}{\textrm{ETSP}_{\textup ent}}

\newcommand{\tspval}{\textrm{TSP}_{\textup val}}
\newcommand{\etsp}{\textrm{ETSP}}
\newcommand{\nt}{{\textup{NTIME}}}
\newcommand{\tn}{{\textrm{T}_N}}
\newcommand{\ein}{{\textrm{Eintrag}}}
\newcommand{\phis}{\phi_{\textup Start}}
\newcommand{\phia}{\phi_{\textup{accept}}}
\newcommand{\phie}{\phi_{\textup Eintrag}}
\newcommand{\phim}{\phi_{\textup move}}
\newcommand{\susu}{{\textup SubsetSum}}
\newcommand{\ghkreis}{{\textup gH-Kreis}}
\newcommand{\uhkreis}{{\textup uH-Kreis}}
\newcommand{\eue}{{\textup Exakte-"Uberdeckung}}
\newcommand{\kue}{{\textrm{Knoten"uberdeckung}}}
\newcommand{\clopt}{{\textup Clique}_{\textup opt}}
\newcommand{\unabopt}{{\textup Unabh"angig}_{\textup opt}}
\newcommand{\zq}{\Z_q}
\newcommand{\god}{\textrm{G"odel}}
\newcommand{\mr}{M^{\textrm{reject}}}
\newcommand{\diag}{\textrm{Diag}}
\newcommand{\exo}{\textrm{XOR}}
\newcommand{\use}{\textrm{Useful}}
\newcommand{\pot}{{\cal P}(Q)}

\newcommand{\remu}{R(1,3)}
\newcommand{\remum}{R(1,n)}
\newcommand{\rsq}{RS(q,m,n)}

\newcommand{\ee}{{\textup e}}

\newcommand{\bigO}{{\cal O}}
\newcommand{\bit}{\{0,1\}}
\newcommand{\bits}{\{0,1\}^*}

\newcommand{\rd}{\rhd}
\newcommand{\qa}{q_{{\textup accept}}}
\newcommand{\qr}{q_{{\textup reject}}}
\newcommand{\gehtu}[1]{%
  \stackrel{#1}{\smash{\rule{0.15mm}{1ex}}\rule[0.5ex]{2em}{0.15mm}}}
%\newcommand{\Beweis}{{\textbf Beweis.} }
\newcommand{\bla}{\sqcup}
\newcommand{\ent}{\stackrel{\wedge}{=}}

\newcommand{\panda}{PANDA}\hyphenation{PANDA}
 % Diverse praktische Kommandos aus Vorlesungsskript und Übungen
\usepackage{enumerate}


\begin{document}
\exerciseheader
\exercisetitle


\begin{exercise}[6]{$k$-Färbbarkeit in NP}
  Ein ungerichteter Graphen $G=(V,E)$ heißt \emph{k-färbbar}, wenn es eine Abbildung $c: V \rightarrow \{ 1,2,\dots,k\}$ gibt, sodass $c(u)\neq c(v)$ für alle $\{u,v\}\in E$.
  Zeigen Sie, dass die Sprache
  \[
    \textup{Color} = \{\langle G, k \rangle \; \vert \; G \text{ ist ein ungerichteter Graph, der eine $k$-Färbung besitzt} \}
  \]
  in NP liegt.
  Konstruieren Sie hierfür eine NTM, welche die Sprache in polynomieller Zeit entscheidet, und beweisen Sie deren Korrektheit.

  \answer{
    Sei im folgenden N eine NTM

    N bei Eingabe x $\in \{0,1\}*$
    \begin{enumerate}
      \item (Formcheck) Prüfe, ob x von der Form $\langle G, k \rangle$ ist, wobei G der Kodierung eines Graphen entspricht und k der einer natürlichen Zahl
      \item Generiere nicht deterministisch Abbildungen $c: V \rightarrow \{ 1,2,\dots,k\}$
      \item Prüfe $\forall$ $\{u,v\} \in E$, ob c(u) $\neq$ c(v), falls dies nicht der Fall ist verwerfe
      \item Akzeptiere
    \end{enumerate}
    Laufzeit von N

    \begin{enumerate}
      \item Überprüfen ist in P
      \item Die Abbildung kann in polynomieller Zeit generiert werden, da jedem Knoten systematisch ein Element aus $\{ 1,2,\dots,k\}$ zugeordnet wird.
      \item Dies dauert hat im Worst Case eine Laufzeit von  $\lvert V \rvert^2$ also ist diese Polynomiel
      \item Trivialerweise Polynomiel.
    \end{enumerate}

    Korrektheit: \\

    Behauptung: L(N) = Color. \\

    1) Color $\subseteq$ L(N) \\

    Angenommen x $\in$ Color

    $\rightarrow \text{ Es existert ein k } \in \mathbb{N} \text{ s.d. x=} \langle G,k \rangle $ und G besitzt eine k-Färbung.

    $\rightarrow$ Es existiert eine Abbildung $w: V \rightarrow \{ 1,2,\dots,k\}$ mit $w(u)\neq w(v)$ für alle $\{u,v\}\in E$.

    $\rightarrow$ x wird in 1 nicht abgelehnt. Die Abbildung w wird in 2 generiert und die Bedingungen in 3 treten nach Voraussetzung ein, folglich wird x in 4 akzeptiert.

    $\rightarrow$ Es existiert eine Berechnung in welcher x akzpetiert wird.

    $\Rightarrow$ x $\in$ L(N) \\

    2)  L(N) $\subseteq$ Color \\

    Abgenommen x $\notin$ Color

    $\rightarrow$ x $ \neq \langle G,k \rangle $, dann wird x in (1) abgelehnt oder x= $\langle G,k \rangle $, aber G ist nicht k färbbar.

    $\rightarrow$ Es kann in 2 keine Abbildung $c: V \rightarrow \{ 1,2,\dots,k\}$ generiert werden für die die Bedingungen aus 3) erfüllt werden können.

    $\rightarrow$ x wird in jeder Berechnung abgelehnt.

    $\Rightarrow$ x $\notin$ L(N).

    $\Rightarrow$ Color=L(N)

    $\Rightarrow$ Color $\in$ NP $\qed$
  }
\end{exercise}


\begin{exercise}[6]{P und Implikation}
  Eine aussagenlogische Formel $\phi$ ist in \textit{disjunktiver Normalform} (DNF), wenn sie von der Form $\phi=\bigvee_{i=1}^k\bigwedge_{j=1}^{m_i}x_{ij}$ ist, wobei $x_{ij}$ Literale sind.
  Betrachten Sie die folgende Sprache
  \begin{align*}
    \textsc{DisSat} & =\{\langle\phi\rangle\mid\phi\mbox{ ist eine erfüllbare Formel in disjunktiver Normalform}\} \ .
  \end{align*}
  \begin{enumerate}
    \item Zeigen Sie, dass \textsc{DisSat} in P liegt.
    \item Jede 3KNF-Formel, das heißt KNF-Formel deren Klauseln aus 3 Literalen bestehen, kann in eine logisch äquivalente DNF-Formel umgeformt werden (wie Sie wahrscheinlich aus der Vorlesung Modellierung wissen).
          Warum implizieren dieser Fakt und a) \emph{nicht} notwendigerweise, dass auch \textsc{3SAT} in P liegt?
  \end{enumerate}

  \answer{
    \begin{enumerate}
      \item Wir zeigen eine Turing Maschine $M$, die die Sprache \textsc{DisSat} in polynomieller Zeit entscheidet.

            $M$ bei eingabe $w$:

            \begin{enumerate}
              \item Wenn $w$ nicht von der Form $\langle \phi \rangle$ für eine Formel $\phi$ in DNF ist, wird $w$ abgelehnt.
              \item $w=\langle \phi \rangle$ für eine Formel $\phi=\bigvee_{i=1}^kK_i$ in DNF.
                    Wobei $K_i=\bigwedge_{j=1}^{m_i}x_{ij}$ ein Konjunktionsterm ist.
                    Für jeden Konjunktionsterm $K_i$ wird folgendes getan:
                    \begin{enumerate}
                      \item Prüfe für alle Literale $x_{ij}$ in $K_i$, ob $\neg x_{ij}$ nicht in $\phi$ vorkommt.
                            Wenn es nicht vorkommt, wird $w$ akzeptiert.
                      \item Wenn $x_{ij}$ und $\neg x_{ij}$ in $\phi$ vorkommt, wird der nächste Konjunktionsterm $K_{i+1}$ betrachtet.
                    \end{enumerate}
              \item Wenn alle Konjunktionsterme betrachtet wurden, wird $w$ abgelehnt.
            \end{enumerate}

            Die Laufzeit von $M$ ist polynomiel, da für jede Formel $\phi$ in DNF nur die Konjunktionsterme in $\phi$ betrachtet werden müssen.

            Die Betrachtung der Konjunktionsterme ist polynomiel, da für jeden Konjunktionsterm $K_i$ nur die Literale in $K_i$ betrachtet werden müssen,
            und naiverweise nur jedes Paar von Literalen in $K_i$ betrachtet werden muss (quadratische Vergleiche).

            Wenn $M$ die Formel $\phi$ in DNF akzeptiert, dann ist $\phi$ erfüllbar, mit einer Interpretation $I$ die alle Literale des $K_i$ erfüllt,
            in dem $M$ akzeptiert.

            Wenn $M$ nicht akzeptiert, dann ist entweder:
            \begin{enumerate}
              \item Die Eingabe $w$ nicht von der Form $\langle \phi \rangle$ für eine Formel $\phi$ in DNF.
              \item $\phi$ nicht erfüllbar, da alle Konjunktionsterme wiederspruchsvoll sind.
            \end{enumerate}

            $\Rightarrow$ \textsc{DisSat} ist in P.

      \item Dieser Fakt impliziert nicht notwendigerweise, dass auch \textsc{3SAT} in P liegt, da die uns bekannte Konversion
            nicht polynomiel ist.

            Insbesondere wandelt der uns bekannte Algorithmus (Distributivgesetz anwenden) eine 3KNF-Formel $\phi$ in eine DNF-Formel $\phi'$ um,
            die $\phi$ logisch äquivalent ist, aber $O(3^n)$ (mit $n$ der Anzahl der Klauseln in $\phi$) Terme hat. Bei einer Formel $\phi$ mit $n$ Klauseln
            ist eine der Terme dann $O(n)$ lang und somit ist die Laufzeit der Reduktion auf $\phi'$ nicht polynomiel.

            Damit ist keine polynomielle Reduktion von \textsc{3SAT} auf \textsc{DisSat} gegeben, und somit ist \textsc{3SAT} nicht notwendigerweise in P.
    \end{enumerate}
  }
\end{exercise}

\begin{exercise}[6]{\textup{SetCover}}
  Zeigen Sie, dass die folgende Sprache in NP liegt:
  \[
    \textup{SetCover} = \left\{\langle A,S_1,\dots,S_n, k \rangle \;\middle|\;
    \begin{aligned}
       & S_1,\dots,S_n\subseteq A \land k\in \N \land \exists I\subseteq \{1,\dots,n\} \\
       & \text{mit } \lvert I \rvert \le k \text{ und } \bigcup_{i\in I} S_i = A
    \end{aligned}  \right\}
  \]

  \answer{

    o.B.d.A alle hier betrachten Mengen sind endlich. \\

    Sei V im Folgenden eine DTM \\

    V bei Eingabe $\in \{0,1\}^*$
    \begin{enumerate}
      \item (Formcheck) Überprüfe, ob x = $\langle A,S_1,\dots,S_n, k, I \rangle$ mit
            $ S_1,\dots,S_n\subseteq A \land k\in \N \land  I\subseteq \{1,\dots,n\}
              \text{mit } \lvert I \rvert \le k$
      \item Prüfe, ob $\bigcup_{i\in I} S_i = A$, falls nicht verwerfe
      \item Akzeptiere
    \end{enumerate}

    Behauptung: V ist ein polynomieller Verifzierer der Sprache SetCover. \\

    Korrektheit: \\

    Angenommen x $\in$ SetCover.

    Also ist x = $\langle A,S_1,\dots,S_n, k \rangle$ mit $S_1,\dots,S_n\subseteq A \land k\in \N$ und es existiert eine Indexmenge I s.d. $\lvert I \rvert \le k$ und $\bigcup_{i\in I} S_i = A$

    Mit I als Zertifikat wird dann dem Entsprechend V in Schritt 1 und 2 die Eingabe nicht verwerfen und somit in 3 akzeptiert.\\

    Angenommen x $\notin$ SetCover.

    Also ist x entweder nicht von der richtigen Form, wird also 1) abgelehnt oder es existiert keine Indexmenge, welche die geforderten Bedingungen nicht erfüllt.

    Also lehnt auch V bei jeder Paarung von x und einer Indexmenge nach 1 und 2 ab, da entweder Bedingungen aus dem Formcheck nicht erfüllt sind oder  $\bigcup_{i\in I} S_i \neq A$, da x sonst $\in$ SetCover.

    $\Rightarrow$ V ist ein Verfizierer von SetCover

    Laufzeit von V:
    \begin{enumerate}
      \item Hat polynomielle Laufzeit, da das überprüfen der Mengen Inklusionen in $\cO(\lvert A \rvert)$ möglich ist. Ob k in $\mathbf{N}$ liegt ist auch trivialerweise in polynmieller Zeit möglich, genauso wie
      \item Dies ist in polynomieller Laufzeit umsetztbar, da die Vereinigung von Mengen in polynomieller Zeit abhängig von der große der Mengen umgesetzt werden kann und der Vergleich in $\cO(\lvert A \rvert)$ berechnet werden kann.
      \item trivialerweise polynomiell
    \end{enumerate}
    Polynomialität von I:\\
    Da I $\subseteq \{1, \dots, n \}$. Kann dieses maximal n Elemente enthalten und kann somit in $\cO(\lvert A \rvert)$ berechnet werden.

    $\Rightarrow$ V ist polynomieller Verifizierer von SetCover $\Rightarrow \text{SetCover} \in NP$
  }
\end{exercise}

\begin{exercise}[6]{Implikationen}
  Seien $A,B,C,D$ beliebige Sprachen.
  Nehmen Sie zusätzlich folgende Beziehungen an
  \begin{itemize}
    \item Es gibt eine Reduktion von $A$ zu $B$, also $A \le B$.
    \item Es gibt eine Reduktion von $B$ zu $C$, also $B \le C$.
    \item Es gibt eine Reduktion von $D$ zu $C$, also $D \le C$.
  \end{itemize}

  Entscheiden und Begründen Sie für jede der folgenden logischen Aussagen kurz, ob sie
  \begin{enumerate}[(i)]
    \item immer wahr ist, also für alle Wahlen von Sprachen erfüllt ist.
    \item möglicherweise wahr ist, also sowohl eine erfüllende als auch eine widersprechende Wahl von Sprachen existiert.
    \item immer falsch ist, also für keine Wahl von Sprachen erfüllt ist.
  \end{enumerate}

  Die Aussagen lauten:
  \begin{enumerate}[a)]
    \item $A$ is rekursiv aufzählbar aber nicht entscheidbar, und $C$ ist entscheidbar.
    \item $A$ ist nicht entscheidbar und $D$ is nicht rekursiv aufzählbar.
    \item Wenn $C$ entscheidbar ist, dann ist das Komplement von $D$ entscheidbar.
    \item Wenn $C$ rekursiv aufzählbar ist, dann ist $B\cap D$ rekursiv aufzählbar.
  \end{enumerate}

  \answer{
    \begin{enumerate}[a)]
      \item (iii), da Reduktionen Transitiv sind $\Rightarrow$ A $\leq$ C und C entscheidbar $\xrightarrow{2.8.2}$ A entscheidbar.
      \item A nicht entscheidbar und A $\leq$ C $\xrightarrow{2.9.1}$ C ist nicht entscheidbar $\iff$ C oder $C^c$ ist nicht rekursiv aufzählbar. \\
            D nicht rekursiv aufzählbar $\xrightarrow{2.9.1}$ C ist nicht rekursiv aufzählbar.
            $\Rightarrow$ (ii) ist richtig.
      \item (i), da D $\leq$ C und C entscheidbar $\xrightarrow{2.8.2}$ D ist entscheidbar $\xrightarrow{2.5.1}$ $D^c$ ist entscheidbar.
      \item (i), da D $\leq$ C, B $\leq$ C und C rekursiv aufzählbar $\xrightarrow{2.8.2}$ D,B sind rekursiv aufzählbar $\xrightarrow{2.5.2}$ $ B \cap D $ ist rekursiv aufzählbar.
    \end{enumerate}
  }
\end{exercise}
Angenommen C ist rekursiv aufzählbar + D $\leq$ C $\xrightarrow{}$ D ist aufzählbar.
Angenommen C ist nicht rekursiv aufzählbar + D $\leq$

\end{document}
