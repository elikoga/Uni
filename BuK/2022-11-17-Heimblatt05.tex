\documentclass[a4paper,12pt]{article}
\usepackage[utf8]{inputenc}
\usepackage[ngerman]{babel}
\usepackage[top=1in, bottom=1.25in, left=1.25in, right=1.25in]{geometry}
\usepackage{minted}
\usepackage{blindtext}
\usepackage{fancyhdr}
\usepackage{titling}
\usepackage{amssymb}
\usepackage{mathtools}

\setlength{\parindent}{0in}
\newcommand{\qed}{\hfill$\square$}



\renewcommand{\footrulewidth}{0.4pt}

\setlength\headheight{15pt}
\setlength{\parskip}{1em}

\title{Document Template}
\author{Eli Kogan-Wang}
\date{\today}

\pagestyle{fancy}
\fancyhf{}
\lhead{\thetitle}
\rhead{\thedate}
\lfoot{\theauthor}
\rfoot{Page \thepage}


\begin{document}

Sei
\[ \textsc{2erPotenz}=\left\{\langle M\rangle \middle| \begin{array}{c}M\text{ akzeptiert genau dann, wenn} \\ \text{die Eingabe die Form }10^i, i\geq 0,\text{ hat.} \end{array}\right\} \]
Zeigen Sie, dass \textsc{2erPotenz} nicht entscheidbar ist.

\vspace{5cm}

Mit $\overline{H}=\{\langle M\rangle x\mid M\text{ hält nicht bei Eingabe }x\}$ ist das Komplement des Halteproblems gemeint.

Wir zeigen ein $f$, sodass

$$w\in \overline{H}\iff f(w)\in \textsc{2erPotenz}$$

Wir definieren:

$$f(w)=\begin{cases}
    \langle M_{reject}\rangle & \text{wenn }w\text{ nicht der Form }\langle M\rangle x \\
    \langle M^{(x)}\rangle    & \text{wenn }w\text{ der Form }\langle M\rangle x       \\
  \end{cases}$$

Wobei $M^{(x)}$ wie folgt bei Eingabe von $y$ vorgeht:

\begin{enumerate}
  \item Form-Check: Wenn $y$ nicht der Form $10^i$ mit $i\geq 0$ ist, dann lehne ab.
  \item Nun: $y=10^i$ mit $i\geq 0$. Simuliere $M$ mit $x$ als Eingabe für $i$ Schritte.
  \item Wenn $M$ nach $i$ Schritten hält, dann lehne ab, ansonsten akzeptiere.
\end{enumerate}

$f$ ist eine berechenbare Funktion.

Nun, angenommen $w\in\overline{H}$

$$\begin{aligned}
    w\in \overline{H} & \implies w=\langle M\rangle x \text{ und } M\text{ hält nicht bei Eingabe }x \\
                      & \implies f(w)=\langle M^{(x)}\rangle\end{aligned}$$

Und $M^{(x)}$ akzeptiert genau dann, wenn $x$ die Form $10^i$ mit $i\geq 0$ hat und $M$ bei Eingabe von $x$ nicht für $i$ Schritte hält.

Da $\langle M\rangle x\in \overline{H}$ genau dann, wenn $M$ nicht bei Eingabe von $x$ hält, wird $M^{(x)}$ für alle Eingaben der Form $10^i$ mit $i\geq 0$ akzeptieren.

Damit ist $f(w)\in \textsc{2erPotenz}$.

Damit $w\in \overline{H}\implies f(w)\in \textsc{2erPotenz}$.

Nun angenommen $w\notin \overline{H}$

Damit ist entweder $w$ nicht der Form $\langle M\rangle x$ $\implies$ $M^{(x)}$ lehnt ab $\implies f(w)\in \textsc{2erPotenz}$

Oder $w=\langle M\rangle x$ und $M$ hält bei Eingabe von $x$.

Damit gibt es ein $i$ sodass $M$ bei Eingabe von $x$ nach $i$ Schritten hält. Damit akzeptiert $M^{(x)}$ das Wort $10^i$ nicht.

Damit ist $f(w)\notin \textsc{2erPotenz}$.

\vspace{1cm}

Wir haben gezeigt, dass $w\in \overline{H}\iff f(w)\in \textsc{2erPotenz}$. Und damit $\overline{H}\leq \textsc{2erPotenz}$.

Damit ist $\textsc{2erPotenz}$ nicht entscheidbar.

\end{document}
