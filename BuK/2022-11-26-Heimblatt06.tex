\documentclass[answers]{submit}
\homework
\usepackage{stmaryrd}
\usepackage{enumitem}
%TODO: Hier Gruppennummer, Namen, sowie Matrikelnummern einfügen.
\exercisegroup{27}{Eli Kogan-Wang (7251030) \\David Noah Stamm (7249709) \\ Bogdan Rerich (7248483) \\Jan Schreiber(7253698)}

% Folgendes braucht nicht geändert werden
\exercisetl{Universität Paderborn \\ Prof. Dr. Johannes Blömer}
\exerciselecture{Berechenbarkeit und Komplexität -- WS 2022/2023}
\exercisenumber{6}
\exercisehandin{28. November 2022 -- 13:00 Uhr}

%% Generic %%
\newcommand{\Q}{\mathbb{Q}}
\newcommand{\N}{\mathbb{N}}
\newcommand{\C}{\mathbb{C}}
\newcommand{\Z}{\mathbb{Z}}
\newcommand{\R}{\mathbb{R}}
\newcommand{\cO}{\mathcal{O}}
\newcommand{\abs}[1]{\left\vert #1 \right\vert}
\newcommand{\norm}[1]{\left\Vert #1 \right\Vert}
\DeclareMathOperator*{\argmax}{arg\,max}
\DeclareMathOperator*{\argmin}{arg\,min}

\newcommand{\la}{\lambda}
\newcommand{\p}{{\textup P}}
\newcommand{\np}{{\textup NP}}
\newcommand{\cst}{{\tt{cost}}}
\newcommand{\val}{{\tt{value}}}


%% PSEUDOCODE %%%%

\newenvironment{pseudocode}[1]%
{\vspace*{-0.5\baselineskip} \upshape \begin{tabbing}
    99 \= bla \= bla \= bla \= bla \= bla \= bla \= bla \= bla \kill
    #1 \+ \\}%
    {\end{tabbing}}
\newcommand{\keyword}[1]{\texttt{#1}}
\newcommand{\IF}{\keyword{IF} }
\newcommand{\THEN}{\keyword{THEN} \+ \\}
\newcommand{\ELSE}{\< \keyword{ELSE} \\}
\newcommand{\END}{\< \keyword{END} \- \\ }
\newcommand{\OR}{\keyword{OR} }
\newcommand{\AND}{\keyword{AND} }
\newcommand{\RETURN}{\keyword{RETURN} }
\newcommand{\FOR}{\keyword{FOR} }
\newcommand{\WHILE}{\keyword{WHILE} }
\newcommand{\TO}{\keyword{TO} }
\newcommand{\DO}{\keyword{DO} \+ \\ }
\newcommand{\REPEAT}{\keyword{REPEAT} \+ \\ }
\newcommand{\UNTIL}{\< \keyword{UNTIL} \- }


\newcommand{\bestl}{{\tt Beste-L"osung}}
\newcommand{\lb}{{\tt untere-Schranke}}
\newcommand{\mcut}{{\textsc{MaxCut}}}
\newcommand{\opt}{{\textup opt}}
\newcommand{\dt}{{\textup{DTIME}}}
\newcommand{\sat}{{\textup{SAT}}}
\newcommand{\satd}{{\textup{3SAT}}}
\newcommand{\satz}{{\textup{2SAT}}}
\newcommand{\clique}{{\textup{Clique}}}
\newcommand{\ggt}{{\textrm{ggT}}}
\newcommand{\ol}{\overline}
\newcommand{\tm}{{\textrm T}_M}
\newcommand{\msb}{\textrm{MSB}}
\newcommand{\rsopt}{\textrm{RS}_{\textup opt}}
\newcommand{\rsent}{\textrm{RS}_{\textup ent}}
\newcommand{\rsval}{\textrm{RS}_{\textup val}}
\newcommand{\tspopt}{\textrm{TSP}_{\textup opt}}
\newcommand{\tspent}{\textrm{TSP}_{\textup ent}}
\newcommand{\etspent}{\textrm{ETSP}_{\textup ent}}

\newcommand{\tspval}{\textrm{TSP}_{\textup val}}
\newcommand{\etsp}{\textrm{ETSP}}
\newcommand{\nt}{{\textup{NTIME}}}
\newcommand{\tn}{{\textrm{T}_N}}
\newcommand{\ein}{{\textrm{Eintrag}}}
\newcommand{\phis}{\phi_{\textup Start}}
\newcommand{\phia}{\phi_{\textup{accept}}}
\newcommand{\phie}{\phi_{\textup Eintrag}}
\newcommand{\phim}{\phi_{\textup move}}
\newcommand{\susu}{{\textup SubsetSum}}
\newcommand{\ghkreis}{{\textup gH-Kreis}}
\newcommand{\uhkreis}{{\textup uH-Kreis}}
\newcommand{\eue}{{\textup Exakte-"Uberdeckung}}
\newcommand{\kue}{{\textrm{Knoten"uberdeckung}}}
\newcommand{\clopt}{{\textup Clique}_{\textup opt}}
\newcommand{\unabopt}{{\textup Unabh"angig}_{\textup opt}}
\newcommand{\zq}{\Z_q}
\newcommand{\god}{\textrm{G"odel}}
\newcommand{\mr}{M^{\textrm{reject}}}
\newcommand{\diag}{\textrm{Diag}}
\newcommand{\exo}{\textrm{XOR}}
\newcommand{\use}{\textrm{Useful}}
\newcommand{\pot}{{\cal P}(Q)}

\newcommand{\remu}{R(1,3)}
\newcommand{\remum}{R(1,n)}
\newcommand{\rsq}{RS(q,m,n)}

\newcommand{\ee}{{\textup e}}

\newcommand{\bigO}{{\cal O}}
\newcommand{\bit}{\{0,1\}}
\newcommand{\bits}{\{0,1\}^*}

\newcommand{\rd}{\rhd}
\newcommand{\qa}{q_{{\textup accept}}}
\newcommand{\qr}{q_{{\textup reject}}}
\newcommand{\gehtu}[1]{%
  \stackrel{#1}{\smash{\rule{0.15mm}{1ex}}\rule[0.5ex]{2em}{0.15mm}}}
%\newcommand{\Beweis}{{\textbf Beweis.} }
\newcommand{\bla}{\sqcup}
\newcommand{\ent}{\stackrel{\wedge}{=}}

\newcommand{\panda}{PANDA}\hyphenation{PANDA}
 % Diverse praktische Kommandos aus Vorlesungsskript und Übungen


\begin{document}
\exerciseheader
\exercisetitle

\begin{exercise}[6]{$\cO$-Kalkül}
  Welche der folgenden Aussagen sind korrekt? Beweisen Sie Ihre Antwort!
  \begin{enumerate}
    \item[a)] $10^{-1000}n^{1.01} \in \Theta(n)$
    \item[b)] $n^2 \in \cO(n\log(n))$
    \item[c)] $n \log(n) \in o(n^2)$
    \item[d)] $n \in \Omega\left(2^{\sqrt{\log(n)}}\right)$
    \item[e)] $n! \in 2^{\cO(n)}$, das heißt, es existiert $c>0$, sodass $n!\le 2^{cn}$.
  \end{enumerate}
  \answer{

    \begin{enumerate}
      \item[a)] $\lim \limits_{n \to \infty} \frac{10^{-1000}n^{1.01}}{n} = \lim \limits_{n \to \infty} 10^{-1000}n^{0.01} = \infty \Rightarrow  10^{-1000}n^{1.01} \notin \Theta(n)$
      \item[b)] $\lim \limits_{n \to \infty} \frac{n^2}{nlog(n)} = \lim \limits_{n \to \infty} \frac{n}{log(n)} \xrightarrow[Krankenhaus]{\text{Regel vom}} \lim \limits_{n \to \infty} \frac{\frac{d}{dx} n}{\frac{d}{dx} log(n)} = \lim \limits_{n \to \infty} \frac{1}{\frac{1}{n}} = \lim \limits_{n \to \infty} n = \infty \\ \Rightarrow n^2 \notin \cO(n\log(n)) $
      \item[c)] $\lim \limits_{n \to \infty} \frac{nlog(n)}{n^2} = \lim \limits_{n \to \infty} \frac{log(n)}{n} \xrightarrow[Krankenhaus]{\text{Regel vom}} \lim \limits_{n \to \infty} \frac{\frac{d}{dx} log(n)}{\frac{d}{dx} n} = \lim \limits_{n \to \infty} \frac{\frac{1}{n}}{1} = \lim \limits_{n \to \infty} \frac{1}{n} = 0  \\ \Rightarrow nlog(n) \in o(n^2)$
      \item[d)] Wir zeigen $\forall n\geq2: n\geq2^{\sqrt{\log(n)}}$.
        $$n\geq 2\implies \exists r\in\mathbb{R}:r\geq 1\land 2^r=n$$
        Solch ein $r$ nennt man auch $r:=\log_2(n)$.
        Nun:
        $$2^{\sqrt{\log(n)}}=2^{\sqrt{\log(2^r)}}=2^\sqrt{r}$$
        Und da:
        $$\sqrt{r}\le r\quad\forall r\geq1$$
        Gilt nun:
        $$2^{\sqrt{\log(n)}}=2^{\sqrt{r}}\le 2^r=n$$
        Damit gilt $n\in \Omega\left(2^{\sqrt{\log(n)}}\right)$
      \item[e)] %%Angenommen es existiert $c>0$, sodass $n!\le 2^{cn} \forall n \geq n_0 \\
        %%\text{Wähle n} \geq 2^c. \Rightarrow log(n!) = log(n)+log(n-1)+...+log(1) \leq log(n)+ log  $
        Angenommen: $ n! \in 2^{\cO(n)} \iff log(n!) \in \cO(n) \iff n log(n) \in \cO(n), \text{aber }  n log(n) \notin \cO(n) \lightning  \Rightarrow n! \notin 2^{\cO(n)} $
    \end{enumerate}
  }
\end{exercise}

\begin{exercise}[6]{Sprachen in P}
  Sei
  \begin{align*}
    \textsc{Connected} & = \{\left<G\right> \mid \text{$G$ ist ein ungerichteter zusammenhängender Graph}\} \text{ und} \\
    \textsc{Triangle}  & = \{\left<G\right> \mid \text{$G$ ist ein ungerichteter Graph mit einer 3-Clique}\} \ .
  \end{align*}
  Zeigen Sie, dass beide Sprachen in P liegen.
  \answer{
    Für $\textsc{Connected}$:

    Wir zeigen eine Turing Maschine $M_{\textsc{Connected}}$, wobei $\textsc{Connected} = L(M_{\textsc{Connected}})$.

    $M_{\textsc{Connected}}$ lehnt alle Eingaben nicht der Form $\langle G\rangle$ mit $G$ ist ein ungerichteter Graph ab (Form-Check).

    $M_{\textsc{Connected}}$ bei Eingabe $\langle G\rangle$ mit $G = (V, E)$ ist ein ungerichteter Graph:
    \begin{enumerate}[itemsep=0em]
      \item Markiere den ersten Knoten $s_1$ \label{start}
      \item Wiederhole die Folgenden Schritte bis keine Knoten mehr markiert werden können: \label{loop}
            \begin{enumerate}
              \item Durchlaufe alle Kanten $\{a, b\}$ des Graphen $G$.
              \item Ist $a$ markiert, aber $b$ nicht so markiere $b$.
              \item Ist $b$ markiert, aber $a$ nicht so markiere $a$.
            \end{enumerate}
      \item Überprüfe ob alle Knoten markiert sind. Wenn ja, akzeptiere, sonst lehne ab. \label{end}
    \end{enumerate}

    Mit der Länge eines Weges bezeichnen wir die Anzahl der Knoten auf einem Weg. Ein
    ungerichteter zusammenhängender Graph ist ein Graph, in dem es von jeden Knoten
    zu jedem anderen einen Weg gibt.

    Insbesondere genügt es zu zeigen, dass es von einem bestimmten Knoten $s$ zu jedem
    anderen einen Weg gibt, da daraus alle anderen Wege über $s$ abgeleitet werden können.
    Die Schritte in \ref{loop} markieren im $k$-ten durchlauf genau die Knoten, die in einem Weg
    von $k$, aber nicht früher erreicht werden können.

    Nach den Ausführungen von \ref{loop} sind damit alle Knoten markiert, die von s aus erreicht
    werden können.

    Die Ausgabe in Schritt \ref{end} ist damit korrekt und die DTM $M_{\textsc{Connected}}$ entscheided
    Connected.

    \vspace{0.2cm}

    Weiter analysieren wir die Laufzeit von $M_{\textsc{Connected}}$.

    Schritt \ref{start} und \ref{end} werden nur einmal ausgeführt. Bei jedem Durchlauf bis auf den letzten
    der Schritte in \ref{loop} wird mindestenz ein Knoten markiert. Maximal $\abs{V} + 1$ Durchläufe
    können demnach durchgeführt werden.

    Schritt \ref{start} ist polynomiell ausführbar. Genau so wie mit \ref{end}. Auch \ref{loop} ist polynomiell, da
    es maximal $\abs{E}\in\mathcal{O}(\abs{V}^2)$ Schritten, $\abs{V} + 1$ mal ausführt.

    Damit ist die Laufzeit von $M_{\textsc{Connected}}$ polynomiell und Connected in P.

    \qed

    \rule{\textwidth}{0.4pt}

    Für $\textsc{Triangle}$.

    Wir zeigen eine Turing Maschine $M_{\textsc{Triangle}}$, wobei $\textsc{Triangle} = L(M_{\textsc{Triangle}})$.

    $M_{\textsc{Triangle}}$ lehnt alle Eingaben nicht der Form $\langle G\rangle$ mit $G$ ist ein ungerichteter Graph ab (Form-Check).

    $M_{\textsc{Connected}}$ bei Eingabe $\langle G\rangle$ mit $G = (V, E)$ ist ein ungerichteter Graph:

    \begin{enumerate}[itemsep=0em]
      \item Wiederhole die folgenden Schritte für jedes $3$-Tupel von Knoten aus $V$: $(s_a,s_b,s_c)\in V\times V\times V$ \label{tri:loop}
            \begin{enumerate}[itemsep=0em]
              \item Wenn nicht $s_a\neq s_b$, $s_b\neq s_c$, $s_c\neq s_a$, gehe zum nächsten Durchlauf. \label{tri:check}
              \item Prüfe, ob $\{s_a, s_b\}$, $\{s_b, s_c\}$, $\{s_c, s_a\}$ in $E$. Falls alle in $E$, akzeptiere, sonst gehe zum nächsten Durchlauf. \label{tri:search}
            \end{enumerate}
      \item Falls wir nicht schon akzeptiert haben, hat der Graph $G$ keine $3$-Clique und wir lehnen ab. \label{tri:reject}
    \end{enumerate}

    Wir betrachten alle $\abs{V}^3$, $3$-Tupel von Knoten in $V$.
    Für die Tupel mit 3 paarweise ungleichen Elementen, prüfen wir, ob sie eine $3$-Clique in $G$ bilden.

    Besitzt $G$ eine solche $3$-Clique so finden wir diese in einem Durchlauf von $\ref{tri:loop}$ und akzeptieren in $\ref{tri:search}$.

    Besitzt $G$ keine $3$-Clique, so erreichen wir Schritt \ref{tri:reject} und lehnen ab.

    Damit ist die Ausgabe in Schritt \ref{tri:search} oder \ref{tri:reject} korrekt und $M_{\textsc{Triangle}}$ entscheidet $\textsc{Triangle}$.

    \vspace{0.2cm}

    Weiter analysieren wir die Laufzeit von $M_{\textsc{Triangle}}$:

    Die Schritte in \ref{tri:loop} werden $\abs{V}^3$ mal ausgeführt und sowohl \ref{tri:check} als auch \ref{tri:search} sind polynomiell über $\abs{V}$.

    Der letzte Schritt \ref{tri:reject} auch polynomiell.

    Damit ist die Laufzeit von $M_{\textsc{Triangle}}$ polynomiell und $\textsc{Triangle}$ in P.

    \qed
  }
\end{exercise}

\begin{exercise}[6]{Klasse P}
  Sei $L\in$ P und $L^*=\{\epsilon\}\cup\bigcup_{k\in\N} L^k$.
  Die formale Definition von $L^k$, der Sprache aller Verkettungen von $k$ Worten aus $L$, finden Sie auf Präsenzübung~3 Aufgabe~2.
  Zeigen Sie, dass $L^*\in$ P.

  \answer{

    %%Da L $\in P$, gibt es eine DTM $\langle M_{L} \rangle$ welche die Sprache L in Laufzeit $\cO(n^m)$ entscheidet. \\

    Nach Präsenzübung 3 Aufgabe 2 entscheidet die folgende Turingmaschine $\langle M_{L^k} \rangle$ die Sprache $L^k$:

    $\langle M_{L^k} \rangle$ bei Eingabe w $\in \{0,1\}$

    \begin{enumerate}
      \item Für alle Aufteilungen $v_1, ..., v_k \in L $ von w mache
            \begin{enumerate}
              \item Für i = 1...k wiederhole Simulation von $\langle M_{L} \rangle$ mit Eingabe $v_i$
              \item Falls $\langle M_{L} \rangle$ in jedem Durchgang akzeptiert hat, akzeptiere
            \end{enumerate}
      \item Verwerfe.
    \end{enumerate}

    Laufzeitanalyse von $\langle M_{L^k} \rangle$: \\
    Da L $\in P$ ist die Laufzeit von $\langle M_{L^k} \rangle$ bei einer Eingabe u höchstens $\cO(|u|^m)$ für ein festes m $\in \mathbf{N}$. Da jedes $v_i$ höchstens so lang wie w, ist die Laufzeit von jeder Simulation in b) höchstens $\cO(|w|^m)$. \\

    Da es insgesamt k Simulation in a) durchgeführt werden und k eine von w unabhängige Konstante kann a) durch $\cO(|w|^m)$ abgeschätzt werden. \\

    b) kann trivialerweise in $\cO(1)$ entschieden werden. \\

    Insgesamt führen wir in 1) maximal $\binom{|w|}{k}$ Schleifendurchläufe durch und 2) kann durch $\cO(1)$ abgeschätzt werden. \\

    Insgesamt hat $\langle M_{L^k} \rangle$ also eine Laufzeit von $\binom{|w|}{k}*\cO(|w|^k)+\cO(1) \in \cO(|w|^k)$ und ist damit also in polynomoieller Zeit entscheidbar, da $\binom{|w|}{k} = \frac{|w|!}{k!*(|w|-k)!}$ eine Konstante bezüglich $\cO(|w|^k)$ ist.  \\

    Behauptung: Falls $L_1 \text{ und } L_2 \in P$, so gilt: $L_1 \cup L_2 \in P$ \\

    Beweis: Nach Aufgabe 3 aus Präsenzübung 6 gilt, dass da $L_1 \text{ und } L_2 \in P$ auch $(L_1)^C \text{und} (L_2)^C \in P$ liegen. \\

    Also nach selbiger Aufgabe auch $(L_1)^C \cap (L_2)^C = L_1 \cup L_2 \in P \qed$ \\

    Dies gilt dann auch dem entsprechend für abzählbare Vereinigungen von Sprachen aus P. \\

    Folglich liegt $L^* \in P$, da $L^*=\{\epsilon\}\cup\bigcup_{k\in\N} L^k$ und damit eine abzählbare Vereinigung von Sprachen aus P ist, da
    $\{\epsilon\}$ durch die Turingmaschiene: $\langle M_{\{\epsilon\}} \rangle$, welche nur $\{\epsilon\}$ akzeptiert, trivialerweise in polynomieller Zeit entschieden werden kann.

    Folglich ist $L^*\in$ P $\qed$


  }
\end{exercise}

\begin{exercise}[6]{Reduktion mit $\overline{H}$}
  Beweisen Sie, dass die im Folgenden definierte Sprache des Äquivalenzproblems nicht rekursiv aufzählbar ist.
  Nutzen Sie in ihrer Reduktion das Komplement des Halteproblems.


  \[ L = \{(\langle M \rangle, \langle M' \rangle)\mid L(M)=L(M')\} \]

  \answer{
    Behauptung $\overline{H} \leq L$
    Beweis:

    Wir definieren die Reduktionsfunktion $f: \{0,1\}^* \rightarrow \{0,1\}^*$: \\

    $$f(w)=\begin{cases}
        \langle M^{(reject)} \rangle                    & \text{wenn w} \neq \langle M \rangle x \\
        (\langle M^{(x)}\rangle,\langle M_{(x)}\rangle) & \text{wenn w} = \langle M \rangle x
      \end{cases}$$ \\

    Sei $\langle M^{(reject)} \rangle$ die Turingmaschine die jede Eingabe ablehnt. \\

    $\langle M^{(x)}\rangle$ sei eine leicht abgeänderte Turingmaschine aus dem Satz 2.10.1, welche wie folgt definiert ist: \\

    $\langle M^{(x)}\rangle$ bei Eingabe z $\in \{0,1\}^*$
    \begin{enumerate}
      \item Berechne |z|.
      \item Simuliere M bei Eingabe x für |z| Schritte.
      \item Falls M in |z| Schritten hält, so akzeptiere z.
      \item Sonst akzeptiere z
    \end{enumerate}
    Folglich ist $f(w)$ nach diesem Satz 2.10.1 auch berechbar. \\

    Darüber hinaus gilt die Implikation 2.2: \\

    M hält bei Eingabe x nicht $\Rightarrow M^{(x)}$ hält bei jeder Eingabe z $\{0,1\}^*$ (1) \\


    $\langle M_{(x)}\rangle$ sei eine leicht abgeänderte Turingmaschine aus dem Satz 2.10.1, welche wie folgt definiert ist: \\

    $\langle M_{(x)}\rangle$ bei Eingabe z $\in \{0,1\}^*$
    \begin{enumerate}
      \item Berechne |z|.
      \item Simuliere M bei Eingabe x für |z| Schritte.
      \item Falls M in |z| Schritten hält, so lehne z ab.
      \item Sonst akzeptiere z
    \end{enumerate}
    Folglich ist $f(w)$ nach diesem Satz 2.10.1 auch berechbar. \\
    Darüber hinaus gilt die Implikation 2.2:

    M hält bei Eingabe x nicht $\Rightarrow M_{(x)}$ hält bei jeder Eingabe z $\{0,1\}^*$ (2) \\

    Z.z. $(w \in \overline{H} \iff f(w) \in L)$ \\

    \setlength{\parskip}{1em}

    Richtung $\implies$: Angenommen $w \in \overline{H}$.

    Das heißt, $w=\langle M\rangle x$ mit $M$ hält bei Eingabe $x$ nicht.

    Also wird w unter f nach $\langle M^{(x)}\rangle,\langle M_{(x)}\rangle$ abgebildet.

    Da M bei keiner Eingabe hält, gilt nach (1) und (2) (alternativ folgt dies aus Zeile 3,4 der jeweiligen DTMS), dass sowohl $\langle M^{(x)}\rangle \text{als auch}\langle M_{(x)}\rangle$ jede Eingabe akzeptiert.

    Dementsprechend ist $L((\langle M^{(x)}\rangle)= L(\langle M_{(x)}\rangle)) \text{und es gilt w}\in f(w)$

    \vspace{1cm}

    Richtung $\Leftarrow$: Angenommen $w \notin \overline{H}$.

    Also ist entweder $w$ von falscher Form oder $M$ hält bei Eingabe $x$.

    Ist $w$ von falscher Form, dann ist $f(w)=\langle M^{(reject)} \rangle \notin L$, da $\langle M^{(reject)} \rangle$ nicht von der Form $(\langle M \rangle, \langle M' \rangle)$

    Ist $w=\langle M\rangle x$ und $M$ hält bei Eingabe $x$, dann:

    Es existiert ein $n$ mit $M$ hält bei Eingabe $x$ nach $n$ Schritten.

    Und für alle $n'>n$ gilt: $M$ hält bei Eingabe $x$ nach $n'$ Schritten.

    Damit werden $\langle M^{(x)}\rangle$ und $\langle M_{(x)}\rangle$ bei jeder Eingabe der Länge $n'>n$ halten und die jeweilige Eingabe akzeptieren, bzw ablehnen.

    Also gilt $L(\langle M^{(x)}\rangle)$ = $\{ z | z \in \{0,1\}^* \text{mit} |n| \leq |z| \} $

    und $L(\langle M_{(x)}\rangle)$ = $\{ z | z \notin \{0,1\}^* \text{mit} |n| \leq |z| \} $

    Folglich gilt: $L(\langle M^{(x)}\rangle) \neq L(\langle M_{(x)}\rangle)$

    Damit ist $f(w)=\langle M^{(x)}\rangle,\langle M_{(x)}\rangle \notin L$.

    $\Rightarrow \overline{H} \leq L$ $\qed$
  }
\end{exercise}





\end{document}
