\documentclass[answers]{submit}
\homework

%TODO: Hier Gruppennummer, Namen, sowie Matrikelnummern einfügen.
\exercisegroup{27}{  Eli Kogan-Wang (7251030) \\David Noah Stamm (7249709) \\ Bogdan Rerich (7248483) \\Jan Schreiber(7253698)}

% Folgendes braucht nicht geändert werden
\exercisetl{Universität Paderborn \\ Prof. Dr. Johannes Blömer}
\exerciselecture{Berechenbarkeit und Komplexität -- WS 2022/2023}
\exercisenumber{2}
\exercisehandin{31. Oktober 2022 -- 13:00 Uhr}

%% Generic %%
\newcommand{\Q}{\mathbb{Q}}
\newcommand{\N}{\mathbb{N}}
\newcommand{\C}{\mathbb{C}}
\newcommand{\Z}{\mathbb{Z}}
\newcommand{\R}{\mathbb{R}}
\newcommand{\cO}{\mathcal{O}}
\newcommand{\abs}[1]{\left\vert #1 \right\vert}
\newcommand{\norm}[1]{\left\Vert #1 \right\Vert}
\DeclareMathOperator*{\argmax}{arg\,max}
\DeclareMathOperator*{\argmin}{arg\,min}

\newcommand{\la}{\lambda}
\newcommand{\p}{{\textup P}}
\newcommand{\np}{{\textup NP}}
\newcommand{\cst}{{\tt{cost}}}
\newcommand{\val}{{\tt{value}}}


%% PSEUDOCODE %%%%

\newenvironment{pseudocode}[1]%
{\vspace*{-0.5\baselineskip} \upshape \begin{tabbing}
    99 \= bla \= bla \= bla \= bla \= bla \= bla \= bla \= bla \kill
    #1 \+ \\}%
    {\end{tabbing}}
\newcommand{\keyword}[1]{\texttt{#1}}
\newcommand{\IF}{\keyword{IF} }
\newcommand{\THEN}{\keyword{THEN} \+ \\}
\newcommand{\ELSE}{\< \keyword{ELSE} \\}
\newcommand{\END}{\< \keyword{END} \- \\ }
\newcommand{\OR}{\keyword{OR} }
\newcommand{\AND}{\keyword{AND} }
\newcommand{\RETURN}{\keyword{RETURN} }
\newcommand{\FOR}{\keyword{FOR} }
\newcommand{\WHILE}{\keyword{WHILE} }
\newcommand{\TO}{\keyword{TO} }
\newcommand{\DO}{\keyword{DO} \+ \\ }
\newcommand{\REPEAT}{\keyword{REPEAT} \+ \\ }
\newcommand{\UNTIL}{\< \keyword{UNTIL} \- }


\newcommand{\bestl}{{\tt Beste-L"osung}}
\newcommand{\lb}{{\tt untere-Schranke}}
\newcommand{\mcut}{{\textsc{MaxCut}}}
\newcommand{\opt}{{\textup opt}}
\newcommand{\dt}{{\textup{DTIME}}}
\newcommand{\sat}{{\textup{SAT}}}
\newcommand{\satd}{{\textup{3SAT}}}
\newcommand{\satz}{{\textup{2SAT}}}
\newcommand{\clique}{{\textup{Clique}}}
\newcommand{\ggt}{{\textrm{ggT}}}
\newcommand{\ol}{\overline}
\newcommand{\tm}{{\textrm T}_M}
\newcommand{\msb}{\textrm{MSB}}
\newcommand{\rsopt}{\textrm{RS}_{\textup opt}}
\newcommand{\rsent}{\textrm{RS}_{\textup ent}}
\newcommand{\rsval}{\textrm{RS}_{\textup val}}
\newcommand{\tspopt}{\textrm{TSP}_{\textup opt}}
\newcommand{\tspent}{\textrm{TSP}_{\textup ent}}
\newcommand{\etspent}{\textrm{ETSP}_{\textup ent}}

\newcommand{\tspval}{\textrm{TSP}_{\textup val}}
\newcommand{\etsp}{\textrm{ETSP}}
\newcommand{\nt}{{\textup{NTIME}}}
\newcommand{\tn}{{\textrm{T}_N}}
\newcommand{\ein}{{\textrm{Eintrag}}}
\newcommand{\phis}{\phi_{\textup Start}}
\newcommand{\phia}{\phi_{\textup{accept}}}
\newcommand{\phie}{\phi_{\textup Eintrag}}
\newcommand{\phim}{\phi_{\textup move}}
\newcommand{\susu}{{\textup SubsetSum}}
\newcommand{\ghkreis}{{\textup gH-Kreis}}
\newcommand{\uhkreis}{{\textup uH-Kreis}}
\newcommand{\eue}{{\textup Exakte-"Uberdeckung}}
\newcommand{\kue}{{\textrm{Knoten"uberdeckung}}}
\newcommand{\clopt}{{\textup Clique}_{\textup opt}}
\newcommand{\unabopt}{{\textup Unabh"angig}_{\textup opt}}
\newcommand{\zq}{\Z_q}
\newcommand{\god}{\textrm{G"odel}}
\newcommand{\mr}{M^{\textrm{reject}}}
\newcommand{\diag}{\textrm{Diag}}
\newcommand{\exo}{\textrm{XOR}}
\newcommand{\use}{\textrm{Useful}}
\newcommand{\pot}{{\cal P}(Q)}

\newcommand{\remu}{R(1,3)}
\newcommand{\remum}{R(1,n)}
\newcommand{\rsq}{RS(q,m,n)}

\newcommand{\ee}{{\textup e}}

\newcommand{\bigO}{{\cal O}}
\newcommand{\bit}{\{0,1\}}
\newcommand{\bits}{\{0,1\}^*}

\newcommand{\rd}{\rhd}
\newcommand{\qa}{q_{{\textup accept}}}
\newcommand{\qr}{q_{{\textup reject}}}
\newcommand{\gehtu}[1]{%
  \stackrel{#1}{\smash{\rule{0.15mm}{1ex}}\rule[0.5ex]{2em}{0.15mm}}}
%\newcommand{\Beweis}{{\textbf Beweis.} }
\newcommand{\bla}{\sqcup}
\newcommand{\ent}{\stackrel{\wedge}{=}}

\newcommand{\panda}{PANDA}\hyphenation{PANDA}
 % Diverse praktische Kommandos aus Vorlesungsskript und Übungen


\begin{document}
\exerciseheader
\exercisetitle

\begin{exercise}[6]{Akzeptieren und Entscheiden von Sprachen}
  Betrachten Sie die folgende DTM $M$ mit Eingabealphabet $\Sigma=\{0,1\}$.

  \begin{center}
    \begin{tabular}{c|c|c|c|c}
      $\delta$ & $0$         & $1$              & $\sqcup$         & $\rhd$         \\\hline
      $s$      & $(q_0,0,R)$ & $(q_2,\sqcup,R)$ & $(\qa,\sqcup,L)$ & $(s,\rhd,R)$   \\\hline
      $q_0$    & $(\qr,1,R)$ & $(q_1,1,R)$      & $(\qr,1,R)$      & $(\qr,\rhd,R)$ \\\hline
      $q_1$    & $(\qr,0,R)$ & $(s,1,R)$        & $(\qr,\sqcup,R)$ & $(\qr,\rhd,R)$ \\\hline
      $q_2$    & $(q_2,0,R)$ & $(q_2,\sqcup,R)$ & $(q_2,1,R)$      & $(\qr,\rhd,R)$
    \end{tabular}
  \end{center}

  \begin{enumerate}
    \item Was ist die von $M$ akzeptierte Sprache $L$?
          Erläutern Sie, warum $M$ die Sprache $L$ nicht entscheidet.
          Geben Sie ein Wort $x\not\in L$ an, welches $M$ nicht ablehnt.
    \item Ändern Sie $M$ so ab, dass die Sprache $L$ von Ihrer geänderten DTM entschieden wird.
  \end{enumerate}
  \answer{
    \\ 1. Die von der Turingmaschine akzeptierte Sprache ist: \\
    L(M) = $\{ w \in \{0,1\}^* | w =(011)^n $ mit n $ \in \N_{0} \}$ \\
    M entscheidet die Sprache nicht, da z.B. 1 $\notin $ L(M) nicht ablehnt. \\ \\
    2. \\
    \begin{tabular}{c|c|c|c|c}
      $\delta$ & $0$         & $1$              & $\sqcup$         & $\rhd$         \\\hline
      $s$      & $(q_0,0,R)$ & $(\qr,\sqcup,R)$ & $(\qa,\sqcup,L)$ & $(s,\rhd,R)$   \\\hline
      $q_0$    & $(\qr,1,R)$ & $(q_1,1,R)$      & $(\qr,1,R)$      & $(\qr,\rhd,R)$ \\\hline
      $q_1$    & $(\qr,0,R)$ & $(s,1,R)$        & $(\qr,\sqcup,R)$ & $(\qr,\rhd,R)$ \\\hline
    \end{tabular}
  }
\end{exercise}

\begin{exercise}[6]{Definiton Turingmaschine}
  Betrachten Sie die folgende alternative Definition einer Turingmaschine.
  Das Band der DTM ist in beide Richtungen unendlich, und nicht durch $\rhd$ zu einer Seite hin begrenzt (es gilt also auch $\rhd\not\in\Gamma$).
  Die Eingabe der DTM steht auf dem Band an beliebiger Position und das Band enthält links und rechts der Eingabe nur $\sqcup$.
  Es gilt immer noch $\sqcup\not\in\Sigma$.
  Die Startkonfiguration ist nun die Konfiguration, bei der die DTM sich im Startzustand befindet und der Kopf auf dem ersten Symbol der Eingabe steht.
  Beweisen Sie, dass es zu jeder DTM mit dieser Definition des Bands eine DTM gibt, die die Definition des Vorlesungsskripts erfüllt, sodass beide Maschinen dieselbe Sprache akzeptieren bzw. entscheiden.

  \answer{
    Sei DTM $M$ gegeben als $4$-Tupel $(Q,\Sigma,\Gamma,\delta)$, welche die Sprache L(M) akzeptiert auf einem beidseitigen unendlichen Band \\
    Wir simulieren die DTM M mit einer $\tilde{M}$, welche der Band Definition aus der Vorlesung, wie folgt definiert ist

    \includegraphics[width=\textwidth]{2022_10_27_19_45_Office_Lens.jpg}
  }
\end{exercise}

\begin{exercise}[6]{$2$-Band-DTM Konstruieren}
  Beschreiben Sie eine $2$-Band Turingmaschine, die bei Eingabe der Binärdarstellung zweier Zahlen $a,b\in\N$ die Binärdarstellung der Zahl $a+b$ berechnet.
  Gehen Sie davon aus, dass $a$ bereits auf dem ersten Band und $b$ auf dem zweiten Band steht und bauen Sie Ihre DTM so, dass das Ergebnis $a+b$ am Ende auf Band 1 steht.
  Es genügt, wenn Sie Ihre Turingmaschine informell beschreiben.
  Begründen Sie, warum Ihre DTM das Gewünschte leistet.
  Analysieren Sie die Laufzeit im $\cO$-Kalkül, unter der Annahme, dass weder $a$ noch $b$ mit führenden Nullen angegeben sind.
  \answer{
  O.B.d.A. die Binärzahlen kommen in LSB-First Reihenfolge auf das Band. \\

  Wir geben den DTM $M=(Q=\{q_0,q_1,q_{\text{accept}},q_{\text{reject}}\},\Sigma=\{0,1\},\Gamma=\{\rhd,0,1,\sqcup\},\delta)$.

  Wobei $\delta$ gegeben ist durch:

  \begin{center}
    \begin{tabular}{c|c|c|c|c|c|c}
      Zustand inp & IB1      & IB2      & OB1      & OB2      & Zustand out         & Bewegung \\\hline
      $q_0$       & $\rhd$   & $\rhd$   & $\rhd$   & $\rhd$   & $q_0$               & $R$      \\\hline
      $q_0$       & $0$      & $0$      & $0$      & $0$      & $q_0$               & $R$      \\\hline
      $q_0$       & $0$      & $1$      & $1$      & $0$      & $q_0$               & $R$      \\\hline
      $q_0$       & $1$      & $0$      & $1$      & $0$      & $q_0$               & $R$      \\\hline
      $q_0$       & $1$      & $1$      & $0$      & $0$      & $q_1$               & $R$      \\\hline
      $q_0$       & $\sqcup$ & $\sqcup$ & $\sqcup$ & $\sqcup$ & $q_{\text{accept}}$ & $R$      \\\hline
      $q_0$       & $\sqcup$ & $0$      & $0$      & $\sqcup$ & $q_0$               & $R$      \\\hline
      $q_0$       & $\sqcup$ & $1$      & $1$      & $\sqcup$ & $q_0$               & $R$      \\\hline
      $q_0$       & $0$      & $\sqcup$ & $0$      & $\sqcup$ & $q_0$               & $R$      \\\hline
      $q_0$       & $1$      & $\sqcup$ & $1$      & $\sqcup$ & $q_0$               & $R$      \\\hline
      $q_1$       & $0$      & $0$      & $1$      & $0$      & $q_0$               & $R$      \\\hline
      $q_1$       & $0$      & $1$      & $0$      & $0$      & $q_1$               & $R$      \\\hline
      $q_1$       & $1$      & $0$      & $0$      & $0$      & $q_1$               & $R$      \\\hline
      $q_1$       & $1$      & $1$      & $1$      & $0$      & $q_1$               & $R$      \\\hline
      $q_1$       & $\sqcup$ & $\sqcup$ & $1$      & $\sqcup$ & $q_{\text{accept}}$ & $R$      \\\hline
      $q_1$       & $\sqcup$ & $0$      & $1$      & $\sqcup$ & $q_0$               & $R$      \\\hline
      $q_1$       & $\sqcup$ & $1$      & $0$      & $\sqcup$ & $q_1$               & $R$      \\\hline
      $q_1$       & $0$      & $\sqcup$ & $1$      & $\sqcup$ & $q_{\text{accept}}$ & $R$      \\\hline
      $q_1$       & $1$      & $\sqcup$ & $0$      & $\sqcup$ & $q_1$               & $R$      \\\hline
    \end{tabular}
  \end{center}

  Wobei Bewegung: $R$ dafür steht, dass man auf beiden Bändern nach rechts geht.

  Wenn nicht angegeben: $q_{\text{reject}}$ und $R$.

  Wir gehen jeden Zeitschritt nach rechts und "Full-Adden"-die Bandelemente und speichern in Band 1.

  Den Carry speichern wir im Zustand.

  Wenn $b$ zuende ist, terminieren wir sobald der carry losgeworden ist.

  Damit ist die Worst Case Laufzeit im $\mathcal{O}$-Kalkül $\mathcal{O}(\max(\log a, \log b))$.

  Die Laufzeit ist nicht in $\mathcal{O}(\log b)$, da das hintere Ende von $a$ mit 1en gefüllt werden kann, die bis zum
  Ende zum carry führen.

  }
\end{exercise}


\begin{exercise}[6]{Symmetrische Differenz}
  Die symmetrische Differenz $L_1\Delta L_2$ zweier Sprachen $L_1,L_2\subseteq \Sigma^*$ ist die Menge aller Worte aus $\Sigma^*$, die in genau einer der beiden Sprachen liegen.

  Zeigen Sie: Sind $L_1,L_2$ entscheidbar, so ist auch $L_1\Delta L_2$ entscheidbar.
  Geben Sie dazu explizit eine DTM $M$ als $4$-Tupel $(Q,\Sigma,\Gamma,\delta)$ an, die $L_1\Delta L_2$ entscheidet.
  Gehen Sie dabei davon aus, dass für jedes $L_i$, $i\in\{1,2\}$, eine DTM $M_i = (Q_i, \Sigma,\Gamma_i,\delta_i)$ gegeben ist, die $L_i$ entscheidet.

  \answer{
    Offensichtlicherweise gilt: \\ \\ $L_1\Delta L_2$ = $(L_1 \textbackslash L_2) \cup (L_2 \textbackslash L_1) $ = $ (L_1 \cup L_2) \textbackslash (L_1 \cap L_2) $ (*)\\ \\
    Nach Satz 1.10 (ii) und (iii) existieren Turingmaschinen, welche die Sprache $L_1 \cup L_2$ und $ L_1 \cap L_2$ entscheiden, falls $L_1$ und $L_2$ entscheidbar sind. Das ist hier nach Annahme der Fall, also seien $M_\cup $und $M_\cap$ mit $M_i = (Q_i, \Sigma,\Gamma_i,\delta_i)$ für $i\in\{\cup,\cap\}$. Nun ist nur noch zu Zeigen, dass eine Turingmaschiene $M_{\cup\setminus\cap}$ existiert, welche die Differenz zweier Entscheidbarer Sprachen entscheidet. \\
    Wir konstruieren die 2-Band-DTM $M_{\cup\setminus\cap}$, die wie folgt arbeitet: \\ \\
    • Kopiere den Input auf Band 2. \\
    • Arbeite parallel wie $M_\cup$ auf Band 1 und wie $M_\cap $ auf Band 2. \\ \\
    Formaler ist $M_{\cup\setminus\cap}$ folgendermaßen definiert: Die Zustandsmenge Q von M enthält
    $Q_\cup$×$Q_\cap$. Weiter enthält die Menge Q Zustände, die für das Kopieren der Eingabe
    auf das zweite Band zuständig sind. Schließlich definieren wir für  $q1 \in Q_\cup$, $q2 \in Q_\cap$, alle a_1 $\in \Sigma_1$, a_2 $\in \Sigma_2$ und D_i $\in {L, R, S}$ die Übergangsfunktion $ \delta $ von $M_{\cup\setminus\cap}$
    durch: $\delta ([q_1, q_2], a_1, a_2) = ([p_1, p_2], b_1, b_2, D_1, D_2)$, falls $\delta_1(q_1, a_1) = (p_1, b_1, D_1)$
    und $\delta_2(q_2, a_2) = (p_2, b_2, D_2)$. Weiter sind Regeln vorhanden, die dafür sorgen,
    dass, wenn $M_\cup$ bereits im Zustand $q_a$ gehalten hat, die Rechnung von $M_\cap$
    weitergefuhrt wird, und umgekehrt.\\
    $M_{\cup\setminus\cap}$ akzeptiert eine Eingabe, falls $M_\cup$ die Eingabe akzeptiert und  $M_\cap$ die Eingabe akzeptieren, d. h. der Zustand [$q_a, q_r$] erreicht wird. \\
    Folglich ist nun die gesuchte DTM M genau die DTM $M_{\cup\setminus\cap}$ nach (*). $\qed$
  }
\end{exercise}

\end{document}
