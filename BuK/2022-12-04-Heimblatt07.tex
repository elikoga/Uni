\documentclass[answers]{submit}
\homework

%TODO: Hier Gruppennummer, Namen, sowie Matrikelnummern einfügen.
\exercisegroup{27}{Eli Kogan-Wang (7251030) \\David Noah Stamm (7249709) \\ Bogdan Rerich (7248483) \\Jan Schreiber(7253698)}

% Folgendes braucht nicht geändert werden
\exercisetl{Universität Paderborn \\ Prof. Dr. Johannes Blömer}
\exerciselecture{Berechenbarkeit und Komplexität -- WS 2022/2023}
\exercisenumber{7}
\exercisehandin{5. Dezember 2022 -- 13:00 Uhr}

%% Generic %%
\newcommand{\Q}{\mathbb{Q}}
\newcommand{\N}{\mathbb{N}}
\newcommand{\C}{\mathbb{C}}
\newcommand{\Z}{\mathbb{Z}}
\newcommand{\R}{\mathbb{R}}
\newcommand{\cO}{\mathcal{O}}
\newcommand{\abs}[1]{\left\vert #1 \right\vert}
\newcommand{\norm}[1]{\left\Vert #1 \right\Vert}
\DeclareMathOperator*{\argmax}{arg\,max}
\DeclareMathOperator*{\argmin}{arg\,min}

\newcommand{\la}{\lambda}
\newcommand{\p}{{\textup P}}
\newcommand{\np}{{\textup NP}}
\newcommand{\cst}{{\tt{cost}}}
\newcommand{\val}{{\tt{value}}}


%% PSEUDOCODE %%%%

\newenvironment{pseudocode}[1]%
{\vspace*{-0.5\baselineskip} \upshape \begin{tabbing}
    99 \= bla \= bla \= bla \= bla \= bla \= bla \= bla \= bla \kill
    #1 \+ \\}%
    {\end{tabbing}}
\newcommand{\keyword}[1]{\texttt{#1}}
\newcommand{\IF}{\keyword{IF} }
\newcommand{\THEN}{\keyword{THEN} \+ \\}
\newcommand{\ELSE}{\< \keyword{ELSE} \\}
\newcommand{\END}{\< \keyword{END} \- \\ }
\newcommand{\OR}{\keyword{OR} }
\newcommand{\AND}{\keyword{AND} }
\newcommand{\RETURN}{\keyword{RETURN} }
\newcommand{\FOR}{\keyword{FOR} }
\newcommand{\WHILE}{\keyword{WHILE} }
\newcommand{\TO}{\keyword{TO} }
\newcommand{\DO}{\keyword{DO} \+ \\ }
\newcommand{\REPEAT}{\keyword{REPEAT} \+ \\ }
\newcommand{\UNTIL}{\< \keyword{UNTIL} \- }


\newcommand{\bestl}{{\tt Beste-L"osung}}
\newcommand{\lb}{{\tt untere-Schranke}}
\newcommand{\mcut}{{\textsc{MaxCut}}}
\newcommand{\opt}{{\textup opt}}
\newcommand{\dt}{{\textup{DTIME}}}
\newcommand{\sat}{{\textup{SAT}}}
\newcommand{\satd}{{\textup{3SAT}}}
\newcommand{\satz}{{\textup{2SAT}}}
\newcommand{\clique}{{\textup{Clique}}}
\newcommand{\ggt}{{\textrm{ggT}}}
\newcommand{\ol}{\overline}
\newcommand{\tm}{{\textrm T}_M}
\newcommand{\msb}{\textrm{MSB}}
\newcommand{\rsopt}{\textrm{RS}_{\textup opt}}
\newcommand{\rsent}{\textrm{RS}_{\textup ent}}
\newcommand{\rsval}{\textrm{RS}_{\textup val}}
\newcommand{\tspopt}{\textrm{TSP}_{\textup opt}}
\newcommand{\tspent}{\textrm{TSP}_{\textup ent}}
\newcommand{\etspent}{\textrm{ETSP}_{\textup ent}}

\newcommand{\tspval}{\textrm{TSP}_{\textup val}}
\newcommand{\etsp}{\textrm{ETSP}}
\newcommand{\nt}{{\textup{NTIME}}}
\newcommand{\tn}{{\textrm{T}_N}}
\newcommand{\ein}{{\textrm{Eintrag}}}
\newcommand{\phis}{\phi_{\textup Start}}
\newcommand{\phia}{\phi_{\textup{accept}}}
\newcommand{\phie}{\phi_{\textup Eintrag}}
\newcommand{\phim}{\phi_{\textup move}}
\newcommand{\susu}{{\textup SubsetSum}}
\newcommand{\ghkreis}{{\textup gH-Kreis}}
\newcommand{\uhkreis}{{\textup uH-Kreis}}
\newcommand{\eue}{{\textup Exakte-"Uberdeckung}}
\newcommand{\kue}{{\textrm{Knoten"uberdeckung}}}
\newcommand{\clopt}{{\textup Clique}_{\textup opt}}
\newcommand{\unabopt}{{\textup Unabh"angig}_{\textup opt}}
\newcommand{\zq}{\Z_q}
\newcommand{\god}{\textrm{G"odel}}
\newcommand{\mr}{M^{\textrm{reject}}}
\newcommand{\diag}{\textrm{Diag}}
\newcommand{\exo}{\textrm{XOR}}
\newcommand{\use}{\textrm{Useful}}
\newcommand{\pot}{{\cal P}(Q)}

\newcommand{\remu}{R(1,3)}
\newcommand{\remum}{R(1,n)}
\newcommand{\rsq}{RS(q,m,n)}

\newcommand{\ee}{{\textup e}}

\newcommand{\bigO}{{\cal O}}
\newcommand{\bit}{\{0,1\}}
\newcommand{\bits}{\{0,1\}^*}

\newcommand{\rd}{\rhd}
\newcommand{\qa}{q_{{\textup accept}}}
\newcommand{\qr}{q_{{\textup reject}}}
\newcommand{\gehtu}[1]{%
  \stackrel{#1}{\smash{\rule{0.15mm}{1ex}}\rule[0.5ex]{2em}{0.15mm}}}
%\newcommand{\Beweis}{{\textbf Beweis.} }
\newcommand{\bla}{\sqcup}
\newcommand{\ent}{\stackrel{\wedge}{=}}

\newcommand{\panda}{PANDA}\hyphenation{PANDA}
 % Diverse praktische Kommandos aus Vorlesungsskript und Übungen


\begin{document}
\exerciseheader
\exercisetitle

\begin{exercise}[6]{Sprachen in NP}
  Betrachten Sie die folgenden Sprachen
  \begin{align*}
    \textsc{2-Scheduling} & = \left\{\langle t_1,\dots,t_n,d\rangle \middle | \begin{array}{c}\mbox{Es existiert eine Indexmenge $I\subseteq\{1,\dots,n\}$,} \\ \mbox{sodass $\sum_{i\in I} t_i \leq d$ und $\sum_{i\not\in I}t_i \leq d$.}\end{array} \right\}                    \\
    \textsc{VertexCover}  & = \left\{ \langle G,k\rangle \middle| \begin{array}{c}\mbox{$G=(V,E)$ ist ein ungerichteter Graph,} \\ \mbox{sodass $C\subseteq V$ existiert mit $\lvert C\rvert \leq k$ und} \\ \mbox{für alle $\{u,v\}\in E$ ist $u\in C$ oder $v\in C$.}\end{array}\right\} \ .
  \end{align*}
  Zeigen Sie
  \begin{enumerate}
    \item \textsc{2-Scheduling} $\in$ NP.
    \item \textsc{VertexCover} $\in$ NP.
  \end{enumerate}

  \answer{
  Allgemein kann gezeigt werden, dass eine Sprache in NP liegt, indem man einen polynomiellen Verifizierer angebe. Sei im folgenden V ein Verifizierer. \\

  Zu 1. Sei V eine DTM welche sich bei Eingabe x $\in \{0,1\}^*$ sich wie folgt verhält.
  \begin{enumerate}
    \item Überprüfe, ob Eingabe der Form $\langle t_1,\dots,t_n,d, S\rangle$
          mit $t_1 \text{bis } t_n$ und d $\in \mathbb{R}$ und S $\subseteq \{ 1,...,n\}$.
    \item Teste, ob $\sum_{i\in S} t_i \leq d$ und $\sum_{i\not\in S}t_i \leq d$.
          Falls dies der Fall ist so akzeptiere, sonst lehne ab.
  \end{enumerate}\\
  Korrektheit:\\
  \langle $t_{1},...,t_{n},d$ \rangle \in \textsc{2-Scheduling} $\implies$ Es existiert eine Indexmenge S sodass  $\sum_{i\in S} t_i \leq d$ und $\sum_{i\not\in S}t_i \leq d$ und V akzepiert V $\langle t_1,\dots,t_n,d, S\rangle$\\ \\
  \langle $t_{1},...,t_{n},d$ \rangle \notin \textsc{2-Scheduling} $\implies$ Es existiert keine Indexmenge für die das Geforderte gilt und V lehnt $\langle t_1,\dots,t_n,d, S\rangle$ für alle S ab.\\

  Polynomialität von S:\\



  \\
  Laufzeit von V:\\
  Schritt 1: Der Formcheck lässt sich in polynomieller Zeit durchführen, da die Eingabe größen endlich sind.\\
  Schritt 2: Besteht aus dem Vergleich zweier Summen mit d. Dies lässt sich auch in linearer Zeit umsetzten, also auch explizit in polynomieller Zeit von $\lvert  \langle t_{1},...,t_{n},d \rangle \rvert$ durchführen. \\
  \\
  Zu 2. Sei V eine DTM, die sich bei Eingabe x $\in \{ 0, 1 \}^{*}$ wie folgt verhält.\\
  \begin{enumerate}
    \item Überprüfe ob x von von der Form $\langle G, k, C \rangle$, wobei G ein Graph, k $\in \mathbb{R}$ und C eine Menge sein soll.
    \item Teste für jedes u $\in C$ ob u $\in V$.
    \item Teste ob $\lvert C \rvert \leq k$
    \item Teste für alle $\{ u,v \} \in E$ ob $u \in C \lor v \in C$
  \end{enumerate}
  Korrektheit:\\
  w $\in \textsc{VertexCover} \implies$ Es existiert ein C $\subseteq V, \lvert C \rvert \leq k$ und für alle $\{ u,v \} \in E$ gilt $\{ u,v \} \in E$ ob $u \in C \lor v \in C$. Dieses C erfüllt dann alle 4 Schritte.\\
  w $\notin \textsc{VertexCover} \implies$ Wir können kein C $\subseteq V, \lvert C \rvert \leq k$ finden. Spätestens bei Schritt 3 lehnt V dann also $\langle G, k, C \rangle$ ab. \\

  Polynomialität von C:\\
  Es gilt C $\subseteq V$ für G=(V,E). Also $\lvert C \rvert \leq \lvert V \rvert$ dadurch auch $\lvert C \rvert \leq \lvert (V,E) \rvert$ und schließlich $\lvert C \rvert \leq \lvert \langle G, k \rangle \rvert^{n}$ für ein $n \in \mathbb{N}$\\

  Laufzeit von V:\\
  Schritt 1: Der Formcheck lässt sich in polynomieller Zeit durchführen.\\
  Schritt 2: Lässt sich in $\mathcal{O}(|V|)$, also in polynomieller Zeit von w durchführen.\\
  Schritt 3: Lässt sich in polynomieller Zeit durchführen.\\
  Schritt 4: Im worst-case gibt es für jeden Knoten in V eine Kante zu jedem anderen Knoten in V. Es gäbe also $|V| \cdot |V-1| \cdot \frac{1}{2}$ Kanten zu überprüfen. Das liegt polynomiell unter $|(V,E)|$
  }
\end{exercise}

\begin{exercise}[6]{Klasse NP}
  Sei $L\in$ NP und $L^*=\{\epsilon\}\cup\bigcup_{k\in\N} L^k$.
  Die formale Definition von $L^k$, der Sprache aller Verkettungen von $k$ Worten aus $L$, finden Sie auf Präsenzübung~3 Aufgabe~2.
  Zeigen Sie, dass $L^*\in$ NP.
  \answer{

    Nach Präsenzübung 3 Aufgabe 2 entscheidet die folgende Turingmaschine $\langle M_{L^k} \rangle$ die Sprache $L^k$:

    $\langle M_{L^k} \rangle$ bei Eingabe w $\in \{0,1\}$

    \begin{enumerate}
      \item Für alle Aufteilungen $v_1, ..., v_k \in L $ von w mache
            \begin{enumerate}
              \item Für i = 1...k wiederhole Simulation von $\langle M_{L} \rangle$ mit Eingabe $v_i$
              \item Falls $\langle M_{L} \rangle$ in jedem Durchgang akzeptiert hat, akzeptiere
            \end{enumerate}
      \item Verwerfe.
    \end{enumerate}

    Da L $\in $ NP, ist L polynomiell verifizierbar. Sei V ein polynomieller Verifizierer für L. \\ \\
    Zu zeigen: $L^{k}$ ist polyomiell verifizierbar, bzw. die Existenz eines polynomiellen Verifizierers $V_{k}$ für $L^{k}$. \\
    $V_{k}$ bei Eingabe $\langle w, c \rangle$:
    \begin{enumerate}
      \item Teste, ob $\langle M_{L} \rangle$ für jede Aufteilung $v_{1}, v_{2}, ... , v_{k}$ von $w$, alle $v_{i}$ als Eingabe akzeptiert. Falls ja, akzeptiere. Sonst, lehne ab.
      \item Teste, ob $|\langle c \rangle| \leq |\langle w \rangle|$. Falls ja, akzeptiere. Sonst, lehne ab.
    \end{enumerate}
    Laufzeit von $V_{k}$: \\
    $L \in NP$, daher läuft $\langle M_{L} \rangle$ in polynomieller Zeit. $\langle M_{L} \rangle$ wird $n*k$ Mal durchgeführt (wobei n die Anzahl Aufteilungen von w in k $v_{i}$ ist). \\
    Der zweite Test, kann in linearer Zeit durchgeführt werden. \\
    Insgesamt läuft $V_{k}$ also in polynomieller Zeit.\\
    \\
    $L^*=\{\epsilon\}\cup\bigcup_{k\in\N} L^k$ ist eine Vereinigung von polynomiell verifizierbarer Sprachen. \\
    Nach Präsenzübung 7 Aufgabe 3, gilt: $L_1 \text{ und } L_2 \in NP$, so gilt: $L_1 \cup L_2 \in NP$. Folglich gilt $L^{*} \in NP$.
  }
\end{exercise}

\clearpage

\begin{exercise}[12]{NP und coNP}
  Betrachten Sie folgendende Definitionen, ähnlich zu den Definitionen 3.7 und 3.8 aus der Vorlesung.
  \begin{definition}
    Sei $L$ eine Sprache.
    \begin{itemize}
      \item Eine DTM $F$ heißt \emph{Falsifizierer} für $L$, falls
            \[
              \bar L = \{w\mid \text{$\exists c:$ $F$ akzeptiert $\langle w,c\rangle$}\}.
            \]
      \item Ein Falsifizierer $F$ für $L$  heißt \emph{polynomieller Falsifizierer} für $L$, wenn
            es ein $k\in \N$ gibt, so dass
            \[
              \bar L = \{w\mid \text{$\exists c, |c|\le |w|^k:$ $F$ akzeptiert $\langle w,c\rangle$}\}.
            \]
            Weiter muss die Laufzeit von $F$ bei jeder Eingabe $\langle w,c\rangle$ polynomiell in der Länge $\abs{w}$ von $w$ sein.
      \item Existiert ein polynomieller Falsifizierer für eine Sprache $L$, so heißt $L$ \emph{polynomiell falsifizierbar}.
    \end{itemize}
  \end{definition}
  \begin{definition}
    {\normalfont coNP} ist die Klasse aller Sprachen, die polynomiell falsifizierbar sind.
  \end{definition}
  Beachten Sie, dass sich die Anforderungen an einen Falsifizierer immer auf das Komplement $\bar L$ der Sprache $L$ beziehen.
  Betrachten Sie außerdem folgende Sprache
  \[ \textsc{NonIso} = \left\{ \langle G,H\rangle \;\middle|\; \mbox{$G,H$ sind zwei ungerichtete, nicht isomorphe Graphen} \right\} \ . \]
  Die Definition von \emph{isomorph} finden Sie auf Präsenzübung~7 Aufgabe~2.

  Zeigen Sie
  \begin{enumerate}
    \item \textsc{NonIso} $\in$ coNP.
    \item $L\in$ coNP $\Leftrightarrow$ $\bar L\in$ NP.
    \item P $\subseteq$ NP $\cap$ coNP.
    \item NP $\neq$ coNP $\Rightarrow$ P $\neq$ NP.
  \end{enumerate}

  \answer{
    \begin{enumerate}
      \item \textsc{NonIso} $\in$ coNP.

            \textbf{Beweis:} \\

            Wir zeigen die Existenz eines polynomiellen Falsifizierers $F$ für \textsc{NonIso}.

            Wir beschreiben das verhalten von $F$ für eine beliebige Eingabe $w$.

            Wenn $w\neq \langle \langle G, H\rangle, c\rangle$, mit $G$ und $H$ zwei ungerichtete Graphen
            und $c$ eine Funktion von $V(G)$ zu $V(H)$ (Menge von Tupeln aus $V(G)\times V(H)$, wobei jedes $V(G)$ nur einmal vorkommt), dann lehnne $F$ $w$ ab (Form-Check).

            Wenn $w=\langle \langle G, H\rangle, c\rangle$, mit $G$ und $H$ zwei ungerichtete Graphen
            und $c$ eine Funktion von $V(G)$ zu $V(H)$ (Menge von Tupeln aus $V(G)\times V(H)$), dann führe folgendes aus:

            \begin{enumerate}
              \item Prüfe ob $c$ eine bijektion ist, falls nicht, lehne ab.
              \item Wende $c$ auf $G$ an und erhalte $G'$. Vergleiche $G'$ mit $H$. Wenn $G'\neq H$, lehne ab. Sonst, akzeptiere.
            \end{enumerate}

            Die Laufzeit von $F$ ist polynomiell in der Länge von $w$:

            \textbf{Begründung:} \\
            Der Form-Check prüft, ob $c$ eine Funktion ist. Dies ist polynomiell in $|V(G)|$, da für jeden Knoten geprüft wird, ob er genau einmal abgebildet wird. \\
            Das Hauptprogramm prüft, zuerst ob $c$ eine bijektion ist. Dies ist polynomiell in $|V(H)|$, da für jeden Knoten von $H$ geprüft wird, ob er genau einmal abgebildet wird. \\
            Das Hauptprogramm wendet $c$ auf $G$ an. Dies ist polynomiell in $|V(G)|$, da jeder Knoten von $G$, als auch jede Kante von $G$ genau einmal abgebildet wird. Es gibt maximal $|V(G)|^2$ Kanten in $G$. \\
            Das Hauptprogramm vergleicht $G'$ mit $H$. Dies ist polynomiell in $|V(G)|$, da jeder Knoten, als auch jede Kante auf gleichheit geprüft wird. Es gibt maximal $|V(G)|^2$ Kanten in $G$.

            \textbf{Korrektheit:} \\
            $F$ akzeptiert genau dann, wenn $G$ und $H$ isomorph sind und $c$ ein Graph-Isomorphismus zwischen $G$ und $H$ ist.

            $F$ ist damit ein Falsifizierer für \textsc{NonIso}, da, wenn $\langle G,H\rangle\in \textsc{NonIso}$
            sind $G$ und $H$ nicht isomorph und es existiert kein $c$ so dass $c$ ein Graph-Isomorphismus zwischen $G$ und $H$ ist.

            Ist hingegen $\langle G,H\rangle\notin \textsc{NonIso}$, so ist $G$ und $H$ isomorph und es existiert ein $c$ so dass $c$ ein Graph-Isomorphismus zwischen $G$ und $H$ ist.
            \qed
      \item $L\in$ coNP $\Leftrightarrow$ $\bar L\in$ NP.
            \textbf{Richtung $\Rightarrow$:} Angenommen $L\in$ coNP.
            Dann existiert ein polynomieller Falsifizierer $F$ für $L$.
            Da $F$ ein Falsifizierer für $L$ ist, ist $F$ ein Verifizierer für $\bar L$ (nach Definition von coNP).
            Da $F$ polynomiell ist, ist $\bar L\in$ NP (nach Definition von NP).

            \textbf{Richtung $\Leftarrow$:} Angenommen $\bar L\in$ NP.
            Dann existiert ein polynomieller Verifizierer $V$ für $\bar L$ (nach Definition von NP).
            Da $V$ ein Verifizierer für $\bar L$ ist, ist $V$ ein Falsifizierer für $L$ (nach Definition von coNP).
            Da $V$ polynomiell ist, ist $L\in$ coNP (nach Definition von coNP).
            \qed
      \item P $\subseteq$ NP $\cap$ coNP.
            Angenommen $L\in$ P. Nun existiert eine Turingmaschine $M$ die $L$ in polynomieller Zeit entscheidet.
            Insbesondere entscheided $M$ auch $\bar L$ in polynomieller Zeit.

            Nun ist aus $M$ ein polynomieller Verifizierer für $L$ zu konstruieren indem man $M$ so modifiziert, dass es das Zertifikat $c$ ignoriert und nur $w$ prüft.

            Genau so lässt sich mit $M$ ein polynomieller Verifiziere für $\bar L$ konstruieren indem man $M$ wie zuvor modifiziert und das Ergebnis invertiert.

            \qed
      \item NP $\neq$ coNP $\Rightarrow$ P $\neq$ NP.
            Angenommen NP $\neq$ coNP. Es ist uns bekannt, dass sowohl $NP$ als auch $coNP$ nicht leer sind, und, dass der Schnitt von $NP$ und $coNP$ nicht leer ist (enthält $P$).

            Das heißt, dass eins von beiden gilt: Es existiert ein $L\in$ NP mit L $\notin$ coNP oder es existiert ein $L\in$ coNP mit L $\notin$ NP.

            Einer dieser Fälle impliziert den anderen, mit $L\in$ NP und L $\notin$ coNP $\implies$ $\bar L\in$ coNP und $\bar L\notin$ NP.

            Wir können also annehmen, dass es ein $L\in$ NP mit L $\notin$ coNP gibt.

            Das heißt, dass es einen polynomiellen Verifizierer $V$ für $L$ gibt, aber keinen polynomiellen Falsifizierer $F$ für $L$.
            Der polynomielle Falsifizierer $F$ für $L$ kann auch hals Polynomieller Verifizierer für $\bar L$ verstanden werden.

            Wir wollen zeigen, dass $P\neq NP$. Wir wissen, dass $P\subseteq NP$ und müssen damit ein $L\in$ NP mit L $\notin$ P zeigen.

            Unser Kandidat für $L$ ist das $L$ mit $L\in$ NP und $L \notin$ coNP. Dieses $L$ kann nicht in $P$ sein, da es sonst auch in $coNP$ wäre.
            \qed
    \end{enumerate}
  }
\end{exercise}

\end{document}
