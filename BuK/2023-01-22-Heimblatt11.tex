\documentclass[answers]{submit}
\homework

%TODO: Hier Gruppennummer, Namen, sowie Matrikelnummern einfügen.
\exercisegroup{27}{Eli Kogan-Wang (7251030) \\David Noah Stamm (7249709) \\ Bogdan Rerich (7248483) \\Jan Schreiber(7253698)}

% Folgendes braucht nicht geändert werden
\exercisetl{Universität Paderborn \\ Prof. Dr. Johannes Blömer}
\exerciselecture{Berechenbarkeit und Komplexität -- WS 2022/2023}
\exercisenumber{11}
\exercisehandin{23. Januar 2023 -- 13:00 Uhr}

%% Generic %%
\newcommand{\Q}{\mathbb{Q}}
\newcommand{\N}{\mathbb{N}}
\newcommand{\C}{\mathbb{C}}
\newcommand{\Z}{\mathbb{Z}}
\newcommand{\R}{\mathbb{R}}
\newcommand{\cO}{\mathcal{O}}
\newcommand{\abs}[1]{\left\vert #1 \right\vert}
\newcommand{\norm}[1]{\left\Vert #1 \right\Vert}
\DeclareMathOperator*{\argmax}{arg\,max}
\DeclareMathOperator*{\argmin}{arg\,min}

\newcommand{\la}{\lambda}
\newcommand{\p}{{\textup P}}
\newcommand{\np}{{\textup NP}}
\newcommand{\cst}{{\tt{cost}}}
\newcommand{\val}{{\tt{value}}}


%% PSEUDOCODE %%%%

\newenvironment{pseudocode}[1]%
{\vspace*{-0.5\baselineskip} \upshape \begin{tabbing}
    99 \= bla \= bla \= bla \= bla \= bla \= bla \= bla \= bla \kill
    #1 \+ \\}%
    {\end{tabbing}}
\newcommand{\keyword}[1]{\texttt{#1}}
\newcommand{\IF}{\keyword{IF} }
\newcommand{\THEN}{\keyword{THEN} \+ \\}
\newcommand{\ELSE}{\< \keyword{ELSE} \\}
\newcommand{\END}{\< \keyword{END} \- \\ }
\newcommand{\OR}{\keyword{OR} }
\newcommand{\AND}{\keyword{AND} }
\newcommand{\RETURN}{\keyword{RETURN} }
\newcommand{\FOR}{\keyword{FOR} }
\newcommand{\WHILE}{\keyword{WHILE} }
\newcommand{\TO}{\keyword{TO} }
\newcommand{\DO}{\keyword{DO} \+ \\ }
\newcommand{\REPEAT}{\keyword{REPEAT} \+ \\ }
\newcommand{\UNTIL}{\< \keyword{UNTIL} \- }


\newcommand{\bestl}{{\tt Beste-L"osung}}
\newcommand{\lb}{{\tt untere-Schranke}}
\newcommand{\mcut}{{\textsc{MaxCut}}}
\newcommand{\opt}{{\textup opt}}
\newcommand{\dt}{{\textup{DTIME}}}
\newcommand{\sat}{{\textup{SAT}}}
\newcommand{\satd}{{\textup{3SAT}}}
\newcommand{\satz}{{\textup{2SAT}}}
\newcommand{\clique}{{\textup{Clique}}}
\newcommand{\ggt}{{\textrm{ggT}}}
\newcommand{\ol}{\overline}
\newcommand{\tm}{{\textrm T}_M}
\newcommand{\msb}{\textrm{MSB}}
\newcommand{\rsopt}{\textrm{RS}_{\textup opt}}
\newcommand{\rsent}{\textrm{RS}_{\textup ent}}
\newcommand{\rsval}{\textrm{RS}_{\textup val}}
\newcommand{\tspopt}{\textrm{TSP}_{\textup opt}}
\newcommand{\tspent}{\textrm{TSP}_{\textup ent}}
\newcommand{\etspent}{\textrm{ETSP}_{\textup ent}}

\newcommand{\tspval}{\textrm{TSP}_{\textup val}}
\newcommand{\etsp}{\textrm{ETSP}}
\newcommand{\nt}{{\textup{NTIME}}}
\newcommand{\tn}{{\textrm{T}_N}}
\newcommand{\ein}{{\textrm{Eintrag}}}
\newcommand{\phis}{\phi_{\textup Start}}
\newcommand{\phia}{\phi_{\textup{accept}}}
\newcommand{\phie}{\phi_{\textup Eintrag}}
\newcommand{\phim}{\phi_{\textup move}}
\newcommand{\susu}{{\textup SubsetSum}}
\newcommand{\ghkreis}{{\textup gH-Kreis}}
\newcommand{\uhkreis}{{\textup uH-Kreis}}
\newcommand{\eue}{{\textup Exakte-"Uberdeckung}}
\newcommand{\kue}{{\textrm{Knoten"uberdeckung}}}
\newcommand{\clopt}{{\textup Clique}_{\textup opt}}
\newcommand{\unabopt}{{\textup Unabh"angig}_{\textup opt}}
\newcommand{\zq}{\Z_q}
\newcommand{\god}{\textrm{G"odel}}
\newcommand{\mr}{M^{\textrm{reject}}}
\newcommand{\diag}{\textrm{Diag}}
\newcommand{\exo}{\textrm{XOR}}
\newcommand{\use}{\textrm{Useful}}
\newcommand{\pot}{{\cal P}(Q)}

\newcommand{\remu}{R(1,3)}
\newcommand{\remum}{R(1,n)}
\newcommand{\rsq}{RS(q,m,n)}

\newcommand{\ee}{{\textup e}}

\newcommand{\bigO}{{\cal O}}
\newcommand{\bit}{\{0,1\}}
\newcommand{\bits}{\{0,1\}^*}

\newcommand{\rd}{\rhd}
\newcommand{\qa}{q_{{\textup accept}}}
\newcommand{\qr}{q_{{\textup reject}}}
\newcommand{\gehtu}[1]{%
  \stackrel{#1}{\smash{\rule{0.15mm}{1ex}}\rule[0.5ex]{2em}{0.15mm}}}
%\newcommand{\Beweis}{{\textbf Beweis.} }
\newcommand{\bla}{\sqcup}
\newcommand{\ent}{\stackrel{\wedge}{=}}

\newcommand{\panda}{PANDA}\hyphenation{PANDA}
 % Diverse praktische Kommandos aus Vorlesungsskript und Übungen
\usepackage{enumerate}
\usepackage{algorithm}
\usepackage{algpseudocode}



\begin{document}
\exerciseheader
\exercisetitle

\begin{exercise}[6]{P vs NP}
  Zeigen Sie, dass die folgende Variante von \textsc{SubsetSum} namens \textsc{BoundedSubsetSum} in P liegt:
  \[
    \set{\dtm[(S,W),t]}{
      \begin{array}{l}
        S=\{s_1,\dots,s_m\}\subseteq \N, \text{ mit $s_i\le 10$ für alle $i$, Funktion $W:S\rightarrow \N$ und } \\
        \text{es existiert Teilmenge $T\subseteq S$ und $(x_s)_{s\in T}\in\N^{\abs{T}}$ mit $x_s \le W(s)$,}     \\
        \text{sodass $\sum_{s\in T}x_s\cdot s =t$}
        % \text{$S=(s_1,\dots,s_m)\in \N^{m}$ mit $s_i\le 10$ für alle $i$ und es gibt}\\
        % \text{Indexmenge $I\subseteq \{1,\dots,m\}$, sodass $\sum_{i\in I}s_i=t$}
      \end{array}
    }
  \]
  \emph{Hinweis:} Um formal korrekt abzubilden, dass Elemente aus $S$ mehrfach enthalten sein können, wurde das Problem im Stile einer Multimenge modelliert.
  Die Basismenge ist $S$ und die Funktion $W$ bildet ab, wie oft jedes Element $s\in S$ in der Multimenge enthalten ist.
  Die Bedingung über die Faktoren $x$ fordert, dass ein Element $s\in T$ maximal $W(s)$ mal benutzt wird um das Ergebnis $t$ zu erhalten.

  \answer{
    Es wird angenommen, dass die Summe $W(s_1)+\dots+W(s_m)$ polynomiell in der Kodierung von $(S,W)$ ist.

    Wir beschreiben einen Algorithmus, welcher \textsc{BoundedSubsetSum} in P löst.

    Seien $S=\{s_1,\dots,s_m\}\subseteq \N$ und $W:S\rightarrow \N$ und $t\in \N$ gegeben.

    Wir definieren einen Zustand $(i,j)$ mit $i\in \{0,\dots,m\}$ und $j\in \N$.

    Dieser Zustand beschreibt, dass eine Teilmenge von $S'=\{s_1,\dots,s_i\}$
    und $(x_s)_{s\in S'}\in\N^{\abs{S'}}$ mit $x_s \le W(s)$ existiert,
    sodass $\sum_{s\in S'}x_s\cdot s = j$.

    Jeder Zustand $(i,j)$ zeigt auf weitere Zustände wie folgt:

    \begin{itemize}
      \item $(i+1,j)$, also man kann (mit beachtung von $s_{i+1}$) die gleiche Summe erreichen
      \item $(i+1,j+k\cdot s_{i+1})$, für alle $k\in \{1,\dots,W(s_{i+1})\}$
    \end{itemize}

    Vom Startzustand aus verwenden wir eine Breitensuche, um alle Zustände zu erreichen.
    Wenn wir den Zielzustand $(m,t)$ erreichen, dann ist die Lösung gefunden.

    Die Komplexität der Breitensuche (worst-case) ist $O(|V|+|E|)=O(l+l^2)=O(l^2)$,
    wobei $l$ die Anzahl der Zustände ist.

    Die erreichbaren Summen $t$, sind teil der Zahlen von $0$ bis $10\cdot (W(s_1)+\dots+W(s_m))$.

    Also ist die Anzahl der Zustände $l=O(m\cdot t)=O(m\cdot 10\cdot (W(s_1)+\dots+W(s_m)))$.

    Nun hängt die Komplexität von den Werten $W(s_i)$ ab.

    Da die Summe $W(s_1)+\dots+W(s_m)$ polynomiell in der Kodierung von $(S,W)$ ist,
    dann ist die Komplexität des Algorithmus quadratisch in der Kodierung von $(S,W)$,
    also \textsc{BoundedSubsetSum} ist in P.

  }
\end{exercise}

\begin{exercise}[6]{Approximationsalgorithmus}
  Sei $G=(V,E)$ ein ungerichteter Graph.
  Eine Partition von $V$ in drei Mengen $S_1,S_2,S_3$ nennen wir einen 3-Cut von $G$.
  Ferner bezeichnet $w(S_1,S_2,S_3)$ die Anzahl der Kanten von $G$, deren Endpunkte in unterschiedlichen Mengen $S_i$ liegen.
  Wir nennen $w(S_1,S_2,S_3)$ das Gewicht des 3-Cut $S_1,S_2,S_3$.

  Beim Max-3-Cut-Problem ist für einen gegebenen, ungerichteten Graphen $G=(V,E)$ ein 3-Cut gesucht, dessen Gewicht maximal unter allen 3-Cuts von $G$ ist.
  Betrachten Sie den folgenden Approximationsalgorithmus für das Max-3-Cut-Problem.

  Zeigen Sie, dass der Algorithmus \textsc{approx3Cut} einen Approximationsfaktor von $\frac{2}{3}$ erreicht.
  Das heißt, der Algorithmus berechnet für jeden Graphen einen 3-Cut, dessen Gewicht mindestens $\frac{2}{3}$ des Gewichts einer Partition mit maximalem Gewicht beträgt.

  \begin{algorithm}
    \caption{\textsc{approx3Cut}(G) mit Eingabe $G=(V,E)$}
    \begin{algorithmic}
      \State $S_1\leftarrow V,S_2\leftarrow \emptyset,S_3\leftarrow \emptyset$;
      \While{Es existiert Permutation $\pi:\{1,2,3\}\rightarrow \{1,2,3\}$ und ein $v\in S_{\pi(1)}$,
        \State \phantom{ile } sodass $w(S_1,S_2,S_3)<w(S_{\pi(1)}\setminus \{ v\}, S_{\pi(2)}\cup \{ v \} , S_{\pi(3)})$}
      \State $S_{\pi(1)}\leftarrow  S_{\pi(1)}\setminus \{ v \}$;
      \State $S_{\pi(2)}\leftarrow  S_{\pi(2)}\cup \{ v \}$;
      \EndWhile
      \State {\Return $S_1,S_2,S_3$}
    \end{algorithmic}
  \end{algorithm}

  \emph{Hinweis: }
  Beweisen Sie, dass der Algorithmus nicht hält, solange es noch ein $i\in \{ 1,2,3\}$ und ein $v\in S_i$ gibt, sodass mehr als $\frac{1}{3}$ der Nachbarn von $v$ ebenfalls in $S_i$ enthalten sind.
  Beweisen Sie auch, dass der Algorithmus in Polynomialzeit terminiert.

  \answer{

    \textbf{Behauptung:} \\

    Approx3Cut(G) ist polynomiell bezüglich der Koodierung von G.\\

    Die While Schleife terminiert bezüglich G in polynomiell Zeit, da in jedem Interationsschritt das Gewicht des 3 Cuts um 1 erhöht wird.
    Dieses kann maximal |E| betragen. Folglich muss die Schleife nach Endlich vielen Schritt (maximal |E|) terminieren.
    Außerdem gibt es insgesamt 3! Verschiedene Permutationen für $\pi$, da dieses aus $S_3$ kommt.
    Da diese Permututionen für jeden Knoten überprüft werden müssen, da die while schleife eine Laufzeit von 6*$\abs{V} \cdot \abs{E} \in \cO(\abs{V} \cdot \abs{E})$. Da dieses polynoiell bezüglich G ist und alle anderen Schritte des Algos eine Konstante Laufzeit haben, folgt die Polynomilität des Algo. \\




    \textbf{Behauptung: (1)} \\
    Der Algorithmus hält nicht, solange es noch ein $i\in \{ 1,2,3\}$ und ein $v\in S_i$ gibt,
    sodass mehr als $\frac{1}{3}$ der Nachbarn von $v$ ebenfalls in $S_i$ enthalten sind. \\

    Beweis durch Widerspruch. \\

    Angenommen der Algo. würde halten, obwohl eine solches v, wie in der Annahme existieren würde.

    o.B.d.A Sei $i=1$, v in der Menge $S_1$ enthalten und $N_1(v)$:= $\frac{\text{Anzahl der Nachbarn von v in } S_1}{\text{Anzahl der Nachbarn von v}} $. \\

    Nach Voraussetzung gilt: $N_1(v) > \frac{1}{3}$. \\

    $ \implies N_2(v)+N_3(v) < \frac{2}{3} \implies N_2(v) < \frac{1}{3} \lor N_3(v) < \frac{1}{3}$. \\

    Gelte dies o.B.d.A. für $S_2$. \\

    $\implies w(S_1,S_2,S_3)<w(S_{1}\setminus \{ v\}, S_{2}\cup \{ v \} , S_{3}) = w(S_1,S_2,S_3) + N_1(v)-N_2(v)$. \\

    Folglich kann der Algorithmus noch nicht gehalten haben, da eine weitere Iteration der While-Schleife ausgeführt werden kann. HIER BLITZ EINFÜGEN. $\qed$ \\

    Also gilt $\forall v \in V$, dass $N_i(v) \leq \frac{1}{3} \iff w(S_1,S_2,S_3) \geq \frac{2}{3}$.

    \textbf{Behauptung:}\\
    Der Approx3Cut hat eine Approximationsfaktor von $\frac{2}{3}$ \\

    Aus (1) folgt: \\

    Approx3Cut(G)= w(S) $\geq \frac{2\cdot \abs{E}}{3} \geq \frac{2 \cdot \text{opt(G)}}{3}$ \\

    $\iff \frac{\text{Approx3Cut(G)}}{\text{opt(G)}} \geq \frac{2}{3}$




  }
\end{exercise}

\begin{exercise}[6]{Rekursive Aufzählbarkeit und Entscheidbarkeit -- Klausuraufgabe 18/19 + 20/21}
  \begin{enumerate}[a)]
    \item \emph{Beweisen} Sie, ob folgende Sprache rekursiv aufzählbar oder, durch geeignete Reduktion, nicht rekursiv aufzählbar ist
          \[
            L_{\theenumi} = \set{\dtm[M]x}{\text{DTM $M$ gestartet mit $x$ erreicht eine Konfiguration mit leerem Band.}}
          \]
    \item \emph{Beweisen} Sie, ob folgende Sprache entscheidbar oder, durch geeignete Reduktion, nicht entscheidbar ist
          \[
            L_{\theenumi} = \set{\dtm[M]x}{\text{DTM $M$ berechnet bei Eingabe $x$ die Binärdarstellung von $\abs{x}^2$.}}
          \]
  \end{enumerate}
  \answer{

    a) \\

    Sei im folgenden $M_x$ eine Turingmaschine: \\

    $M_x$ bei Eingabe w $\in \{0,1\}^*$

    \begin{enumerate}
      \item (Formcheck) Prüfe ob w = $ \langle M \rangle$x, wobei M die Gödelisierung einer Turingmaschine ist und x $\in \{0,1\}^*$.
      \item Widerhole für i=1,2,3,....
            \begin{enumerate}
              \item Simuliere M bei Eingabe x für i Schritte. Prüfe nach jedem Schritt, ob das Band leer ist.
              \item Falls dies in einem Schritt der Fall ist, so akzeptiere.
            \end{enumerate}
    \end{enumerate}

    Behauptung: L($M_x$)=$L_a$. \\

    Angenommen w $\in L_a$ \\

    $\implies$ w = $ \langle M \rangle$ x und die Turingmaschine erreicht eine Konfiguration mit einem leeren Band. \\

    $\implies$ w wird in (1) nicht abgelehnt und da in 2 a) durch die Simulation alle möglichen Konfigurationen irgendwann erreicht folglich auch die mit einem leeren Band und die Turingmaschine hält nach diesem.

    $\implies$ w $\in$ L(M)

    Angenommen w $\notin L_a$.

    $\implies$ Entweder w $ \neq \langle M \rangle$x, wird also in (1) abgelehnt oder w = $ \langle M \rangle$x, aber in keiner Konfiguration wird ein leeres Band erreicht.\\

    Folglich wird w in (1) nicht angelehnt, aber es kann in 2a) keine Konfiguration gefunden werden, in der ein leeres Band erreicht wird.\\

    $\implies$ w wird in jedem Iterationsschritt abgelehnt.

    $\implies$ w $\notin$ L(M)

    $\implies$ L(M)=$L_a \qed$.

    Da $M_x$ offensichtlicher Weise rekursiv aufzählbar ist, da falls M bei Eingabe x in eine Endlosschleife kommt, auch $M_x$ in eine Endlosschleife kommt.

    $\implies L_a$ ist rekursiv aufzählbar.

    \vspace{1cm}
    \hrule

    b) \\

    Sei $$L_H=\left\{\langle M \rangle x \mid M \text{ ist eine Turingmaschine die bei Eingabe } x \text{ hält } \right\}$$

    Das Halteproblem, welches bekanntlich nicht entscheidbar ist.\\

    Wir zeigen: $L_H \leq L_b$ \\

    Zuerst die Reduktionsfunktion $f: \{0,1\}^* \rightarrow \{0,1\}^*$.\\

    Sei $w \in \{0,1\}^*$, aber keine Kodierung $ \langle M \rangle x$, mit $M$ eine Gödelisierung einer Turingmaschine und $x \in \{0,1\}^*$.\\

    Dann ist $f(w):=w$, was auch keine Gödelisierung einer Turingmaschine ist.\\

    Sei $w = \langle M \rangle x$ mit $M$ eine Gödelisierung einer Turingmaschine und $x \in \{0,1\}^*$.\\

    Nun ist $f(w)$ eine Turingmaschine $M'$, die bei Eingabe $y$ wie folgt verhält:\\

    \begin{enumerate}
      \item Simuliere $M$ bei Eingabe $x$, bis zur Terminierung.
      \item Schreibe die Binärdarstellung von $|y|^2$ auf das Band.
    \end{enumerate}

    Wir zeigen: $f$ ist eine Reduktion von $L_H$ nach $L_b$.\\

    Angenommen $w \in L_H$.\\

    Also $w = \langle M \rangle x$ mit $M$ eine Gödelisierung einer Turingmaschine und $x \in \{0,1\}^*$.\\

    Da $w \in L_H$ hält $M$ bei Eingabe $x$.\\

    $\implies$ $f(w) = \langle M' \rangle$.

    Bei Eingabe $y$ wird $M'$, die Binärdarstellung von $|y|^2$ auf das Band schreiben, da $M$ bei Eingabe $x$ hält.\\

    Also ist $f(w) \in L_b$.

    Angebommen $w \notin L_H$.\\

    Wenn $w \notin L_H$ ist $w$ keine Kodierung $ \langle M \rangle x$, mit $M$ eine Gödelisierung einer Turingmaschine und $x \in \{0,1\}^*$.\\

    $\implies$ $f(w) = w$.

    Also ist $f(w) \notin L_b$, da $f(w)$ keine Gödelisierung einer Turingmaschine ist.

    Wenn $w \notin L_H$ ist $w$ eine Kodierung $ \langle M \rangle x$, mit $M$ eine Gödelisierung einer Turingmaschine und $x \in \{0,1\}^*$.\\

    Da $w \notin L_H$ hält $M$ nicht bei Eingabe $x$.\\

    $\implies$ $f(w) = \langle M' \rangle$.

    Bei Eingabe $y$ wird $M'$ nicht halten, da $M$ nicht bei Eingabe $x$ hält.\\

    Also wird $M'$ nicht die Binärdarstellung von $|y|^2$ auf das Band schreiben.\\

    Also ist $f(w) \notin L_b$.

    Damit ist $f$ eine Reduktion von $L_H$ nach $L_b$.\\

    $\implies L_H \leq L_b$.

    Da das Halteproblem nicht entscheidbar ist, ist $L_b$ nicht entscheidbar.\\
  }
\end{exercise}

\begin{exercise}[6]{NP-Vollständigkeit -- Klausuraufgabe 20/21}
  Betrachten Sie die folgende Sprache.
  \[
    \textsc{Max-SAT} :=
    \set{\dtm[\phi,k]}{
      \begin{array}{l}
        \text{$\phi$ ist eine Boolsche Formel in KNF, $k\in\N$ und es} \\
        \text{existiert eine Belegung der Variablen, sodass }          \\
        \text{mindestens $k$ Klauseln von $\phi$ erfüllt sind.}
      \end{array}}
  \]

  Zeigen Sie, dass die Sprache \textsc{Max-SAT} NP-vollständig ist. Nutzen Sie für Ihre Reduktion eine der drei folgenden NP-vollständigen Sprachen
  \begin{enumerate}[a)]
    \item $\textsc{SAT} = \set{\dtm[\phi]}{\text{$\phi$ ist eine erfüllbare Boolesche Formel in KNF.}}$
    \item $\textsc{3SAT} = \set{\dtm[\phi]}{\text{$\phi$ ist eine erfüllbare Boolesche Formel in 3-KNF.}}$
    \item $\textsc{VertexCover} = \set{\dtm[G,k]}{\text{$G$ ist ein ungerichteter Graph mit einer $k$-Knotenüberdeckung.}}$
  \end{enumerate}
  \answer{
    Es ist folgendes zu Zeigen: \\

    \begin{enumerate}
      \item Max-Sat $\in$ NP
      \item SAT $\leq_p$ Max-Sat
    \end{enumerate}

    Zu (1):

    Betrachte folgende Turingmaschine:

    V bei Eingabe x $\in \{0,1\}^*$:
    \begin{enumerate}
      \item (Formcheck) Prüfe, ob x = $\langle \phi,k,B \rangle$ eine Kodierung ist, wobei $\phi$ eine boolesche Formel in KNF ist, k $\in \N$ und B eine Belegung von $\phi$
      \item Prüfe, ob B mindestens k Klauseln der KNF erfüllt. Falls dies der Fall ist so lehne x ab.
      \item Akzeptiere.
    \end{enumerate}

    \textbf{BH.: V ist Polynomiell} \\

    \begin{enumerate}
      \item Der Formcheck kann trivialerweise in polynomieller Zeit durchgeführt werden.
      \item Da hierbei nur maximal die Anzahl der Klauseln in $\phi$ überprüft werden müssen, ist dies in poly. Zeit bezüglich $\phi$ möglich.
      \item Trivialerweise in Polynomieller Zeit möglich
    \end{enumerate}

    Da B eine Belegung von $\phi$, welche jedem Literal aus diesem einen Wahrheitswert zuordnet, hat dieses bezüglich $\phi$ eine polynomielle Länge. \\

    \textbf{BH.: V ist Verifzierer von Max-SAT} \\

    Angenommen $\langle \phi,k \rangle \in $ Max-SAT. d.h. es existiert eine Belegung B von $\phi$, welche k Klauseln erfüllt. \\

    $\implies$ V akzeptiert x=$\langle \phi,k,B \rangle$ nach (3). \\

    $\implies$ x $\in L(v)$ \\

    Angenommen $\langle \phi,k \rangle \notin $ Max-SAT \\

    Beweis durch Widerspruch: \\

    Angenommen es würde eine Belegung B von $\phi$ geben s.d. \\

    x = $\langle \phi,k,B \rangle \in L(V)$

    $\implies$ (Nach (3)) $\langle \phi,k \rangle \in $ Max-SAT WIEDERSPRUCHS BLITZ EINFÜGEN.

    $\implies$ V ist ein Verifizierer von Max-SAT.

    $\implies$ Max-SAT $\in NP$ \\

    Zu (2): \\

    Behauptung: \textbf{SAT} $\leq_p$ \textbf{Max-Sat}. \\

    Betrachte folgende Reduktionsfunktion: \\

    $$f(w)=\begin{cases}
        \langle \phi,k\rangle &
        \begin{array}{l}
          \text{wenn w} = \langle \phi \rangle \text{,wobei } \phi \text{ die Koodierung einer KNF ist} \\
          \text{und k die Anzahl ihrer Klauseln}                                                        \\
        \end{array} \\
        \delta                & \text{ wenn w} \neq \langle \phi \rangle \text{ sonst}
      \end{cases}$$ \\

    Wobei $\delta$ eine Formel ist, welche nicht in KNF ist. \\

    Die Laufzeit von f(w) ist trivialerweise polynomiell und berechenbar,da diese nur die Anzahl der Klauseln von $\phi$ bestimmt. \\

    Behauptung: w $\in$ SAT $\iff$ f(w) $\in$ Max-Sat \\

    Angenommen w $\in$ SAT\\

    $\implies$ w wird unter f auf $\langle \phi,k\rangle$ abgebildet
    und es existiert eine Erfüllende Belegung B von $\phi$. \\

    $\implies$ Folglich muss B jede Klauseln in $\phi$ erfüllen (,da KNF) und B erfüllt damit k Klauseln.

    $\implies$ f(w) $\in$ Max-Sat. \\ \\

    Angenommen w $\notin$ Sat. \\

    Also ist w entweder von der flasche Form, wird dann also auf $\delta$ abgeldet,
    welches nicht in Max-Sat ist, da nicht in KNF oder auf  $\langle \phi,k\rangle$ \\

    Angommen f(w) $\in $ Max-Sat

    $\implies$ Es existiert eine Belegung, welche K klauseln erfüllen würde.
    Nach wahl von k ist diese Belegung dann auch eine erfüllende Belegung von $\phi$ \\

    $\implies$ w $\in$ SAT WIDERSPRUCHSPFEIL EINFÜGEN.

    $\implies$ w $\in$ SAT $\iff$ f(w) $\in$ Max-Sat \\

    $\implies$ \textbf{SAT} $\leq_p$ \textbf{Max-Sat}.

    $\xrightarrow{1,2}$ Max Sat ist NP vollständig. $\qed$


  }
\end{exercise}


\end{document}
