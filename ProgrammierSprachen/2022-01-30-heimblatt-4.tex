\documentclass[a4paper,12pt]{article}
\usepackage{fancyhdr}
\usepackage{fancyheadings}
\usepackage[ngerman,german]{babel}
\usepackage[utf8]{inputenc}
%\usepackage[latin1]{inputenc}
\usepackage[active]{srcltx}
\usepackage{enumerate}
\usepackage{graphicx}
\usepackage{ifthen}
\usepackage{listings}
\usepackage{struktex}
\usepackage{hyperref}
\usepackage{longtable}
\usepackage{gauss}
\usepackage{dsfont}
\usepackage{amssymb}
\usepackage{minted}
\usepackage{stmaryrd}

\newcommand{\x}{\cdot}
\newcommand{\ran}[2]{\langle #1,#2\rangle}
\renewcommand{\Re}{\mathbb{R}}
\makeatletter
\newcommand{\Spvek}[2][r]{%
  \gdef\@VORNE{1}
  \left(\hskip-\arraycolsep%
  \begin{array}{#1}\vekSp@lten{#2}\end{array}%
  \hskip-\arraycolsep\right)}

\def\vekSp@lten#1{\xvekSp@lten#1;vekL@stLine;}
\def\vekL@stLine{vekL@stLine}
\def\xvekSp@lten#1;{\def\temp{#1}%
  \ifx\temp\vekL@stLine
  \else
  \ifnum\@VORNE=1\gdef\@VORNE{0}
  \else\@arraycr\fi%
  #1%
  \expandafter\xvekSp@lten
  \fi}
\makeatother

%%%%%%%%%%%%%%%%%%%%%%%%%%%%%%%%%%%%%%%%%%%%%%%%%%%%%%
%%%%%%%%%%%%%% EDIT THIS PART %%%%%%%%%%%%%%%%%%%%%%%%
%%%%%%%%%%%%%%%%%%%%%%%%%%%%%%%%%%%%%%%%%%%%%%%%%%%%%%
\newcommand{\Fach}{Programmiersprachen}
\newcommand{\Name}{Name:Björn Bick, Max Herting, Miriam Nippel, Eli Kogan-Wang}
\newcommand{\Semester}{WS 21/22}
\newcommand{\Uebungsblatt}{4} %  <-- UPDATE ME
%%%%%%%%%%%%%%%%%%%%%%%%%%%%%%%%%%%%%%%%%%%%%%%%%%%%%%
%%%%%%%%%%%%%%%%%%%%%%%%%%%%%%%%%%%%%%%%%%%%%%%%%%%%%%
\setlength{\parindent}{0em}
\topmargin -1.0cm
\oddsidemargin 0cm
\evensidemargin 0cm
\setlength{\textheight}{9.2in}
\setlength{\textwidth}{6.0in}

%%%%%%%%%%%%%%%
%% Aufgaben-COMMAND
\newcommand{\Aufgabe}[1]{
  {
  \vspace*{0.5cm}
  \textsf{\textbf{Aufgabe #1}}
  \vspace*{0.2cm}

  }
}
%%%%%%%%%%%%%%
\hypersetup{
    pdftitle={\Fach{}: Übungsblatt \Uebungsblatt{}},
    pdfauthor={\Name},
    pdfborder={0 0 0}
}

\lstset{ %
language=java,
basicstyle=\footnotesize\tt,
showtabs=false,
tabsize=2,
captionpos=b,
breaklines=true,
extendedchars=true,
showstringspaces=false,
flexiblecolumns=true,
}



\title{Übungsblatt \Uebungsblatt{}}
\author{\Name{}}

\begin{document}
\pagestyle{fancy}
\lhead{\sf \large \Fach{} \\ \small \Name{}}
\rhead{\sf \Semester{} \\}
\vspace*{0.2cm}
\begin{center}
  \LARGE \sf \textbf{Übungsblatt \Uebungsblatt{}}
\end{center}
\vspace*{0.2cm}

\newcommand{\dd}{\text{d}}
%%%%%%%%%%%%%%%%%%%%%%%%%%%%%%%%%%%%%%%%%%%%%%%%%%%%%%
%% Insert your solutions here %%%%%%%%%%%%%%%%%%%%%%%%
%%%%%%%%%%%%%%%%%%%%%%%%%%%%%%%%%%%%%%%%%%%%%%%%%%%%%%

\Aufgabe{1}

\begin{enumerate}[a)]
  \item \begin{minted}{prolog}
istStudi('Alice', 'Jäckel').
istStudi('Pius', 'Hölzenbeck').
istStudi('Adina', 'Walter').
istStudi('Kathi', 'Meister').
istStudi('Valeska', 'Warmer').
istStudi('Sönke', 'Römer').
istStudi('Leonard', 'Koch').

hatMatrikelnummer('Alice', 'Jäckel', 9523343).
hatMatrikelnummer('Pius', 'Hölzenbeck', 9523377).
hatMatrikelnummer('Adina', 'Walter', 9523418).
hatMatrikelnummer('Kathi', 'Meister', 8533463).
hatMatrikelnummer('Valeska', 'Warmer', 9523888).
hatMatrikelnummer('Sönke', 'Römer', 9523574).
hatMatrikelnummer('Leonard', 'Koch', 8925543).

userZurMatrikelnummer(ajaekel, 9523343).
userZurMatrikelnummer(phoelzer, 9523377).
userZurMatrikelnummer(adwalter, 9523418).
userZurMatrikelnummer(kmeister, 8533463).
userZurMatrikelnummer(vwarmer, 9523888).
userZurMatrikelnummer(sroemer, 9523574).
userZurMatrikelnummer(leokoch, 8925543).

haeltVorlesung('Einführung in Geschichte', 'Rogge').
haeltVorlesung('Philosophie', 'Oestrovsky').
haeltVorlesung('Alte Sprachen', 'Eckbauer').
haeltVorlesung('Latein', 'Finke').
haeltVorlesung('Mathematik', 'Benthin').
haeltVorlesung('Englisch', 'Finke').
haeltVorlesung('Kunstgeschichte', 'Eckbauer').

hoertVorlesung(ajaekel, 'Einführung in Geschichte').
hoertVorlesung(ajaekel, 'Philosophie').
hoertVorlesung(phoelzer, 'Philosophie').
hoertVorlesung(adwalter, 'Einführung in Geschichte').
hoertVorlesung(adwalter, 'Alte Sprachen').
hoertVorlesung(adwalter, 'Latein').
hoertVorlesung(kmeister, 'Mathematik').
hoertVorlesung(kmeister, 'Englisch').
hoertVorlesung(vwarmer, 'Mathematik').
hoertVorlesung(vwarmer, 'Kunstgeschichte').
hoertVorlesung(sroemer, 'Einführung in Geschichte').
hoertVorlesung(sroemer, 'Philosophie').
hoertVorlesung(sroemer, 'Latein').
hoertVorlesung(leokoch, 'Mathematik').
hoertVorlesung(leokoch, 'Alte Sprachen').
hoertVorlesung(leokoch, 'Latein').
  \end{minted}
  \item \begin{minted}{prolog}
hoert(Vorlesung, Vorname, Nachname) :-
    istStudi(Vorname, Nachname),
    hatMatrikelnummer(Vorname, Nachname, Matrikelnummer),
    userZurMatrikelnummer(User, Matrikelnummer),
    hoertVorlesung(User, Vorlesung).

schreibtPruefung(Vorlesung, User) :-
    userZurMatrikelnummer(User, Matrikelnummer),
    Matrikelnummer >= 9000000,
    hoertVorlesung(User, Vorlesung).

wasHoerenDieStudentenNoch(Professor, Vorlesung) :-
    haeltVorlesung(EineVorlesung, Professor),
    hoertVorlesung(User, EineVorlesung),
    hoertVorlesung(User, Vorlesung),
    \+ haeltVorlesung(Vorlesung, Professor).
  \end{minted}
\end{enumerate}

\Aufgabe{2}

\begin{enumerate}[a)]
  \item \begin{minted}{prolog}
istInListe(Liste, Elem) :-
    Liste = [Elem | _].

istInListe(Liste, Elem) :-
    Liste = [_ | Rest],
    istInListe(Rest, Elem).
  \end{minted}
  \item \begin{minted}{prolog}
istLetztesElement(Liste, Elem) :-
    Liste = [Elem].

istLetztesElement(Liste, Elem) :-
    Liste = [_ | Rest],
    istLetztesElement(Rest, Elem).

gibAusLetztesElement(Liste) :-
    istLetztesElement(Liste, Elem),
    write(Elem).
  \end{minted}
  \item \begin{minted}{prolog}
laengeListe(Liste, Laenge) :-
    Liste = [] -> Laenge is 0;
    Liste = [_ | Rest],
    laengeListe(Rest, LaengeRest),
    Laenge is LaengeRest + 1.
  \end{minted}
  \item \begin{minted}{prolog}
konkatenationListe(Liste1, Liste2, Liste3) :-
    Liste1 = [],
    Liste2 = Liste3.

konkatenationListe(Liste1, Liste2, Liste3) :-
    Liste1 = [Elem | Rest],
    konkatenationListe(Rest, Liste2, Rest2),
    Liste3 = [Elem | Rest2].
  \end{minted}
\end{enumerate}

\Aufgabe{3}
\begin{enumerate}[a)]
  \item \begin{minted}{prolog}
fibonacci(Index, Fibonacci) :-
  Index = 0 -> Fibonacci is 0;
  Index = 1 -> Fibonacci is 1;
  Index > 1,
  Index1 is Index - 1,
  Index2 is Index - 2,
  fibonacci(Index1, Fibonacci1),
  fibonacci(Index2, Fibonacci2),
  Fibonacci is Fibonacci1 + Fibonacci2.
\end{minted}
  \item \begin{minted}{prolog}
hanoi(0, _, _, _).

hanoi(Anzahl, Start, Ziel, Hilf) :-
    Anzahl > 0,
    Anzahl1 is Anzahl - 1,
    hanoi(Anzahl1, Start, Hilf, Ziel),
    write(Start),
    write(' -> '),
    write(Ziel),
    nl,
    hanoi(Anzahl1, Hilf, Ziel, Start).

% Aufruf mit bspw. ?- hanoi(4, 'A', 'B', 'C').
  \end{minted}
\end{enumerate}

\Aufgabe{4}

\begin{enumerate}[a)]
  \item \begin{minted}{prolog}
sagtWahrheitWennLügt(milan, yasmin).
sagtWahrheitWennLügt(yasmin, samuel).

lügtWennSagtWahrheit(milan, yasmin).
lügtWennSagtWahrheit(yasmin, samuel).
\end{minted}
  \item \begin{minted}{prolog}
lügtWennSagtWahrheit(samuel, milan).
lügtWennSagtWahrheit(samuel, yasmin).
\end{minted}
  \item \begin{minted}{prolog}
werLügt(X) :-
  lügtWennSagtWahrheit(X, Y),
  sagtWahrheitWennLügt(Y, Z),
  lügtWennSagtWahrheit(Z, Y).
% Y sagt die Wahrheit.
\end{minted}
\end{enumerate}

\Aufgabe{5}
\begin{minted}{prolog}
die_sonne_scheint. % R1
das_wetter_ist_gut :- die_sonne_scheint. % R2
\end{minted}
\begin{minted}{text}
Beweis:
-das_wetter_ist_gut % Q
das_wetter_ist_gut, -die_sonne_scheint % R2
----
-die_sonne_scheint.
die_sonne_scheint % R1
----
[] Widerspruch
\end{minted}

%%%%%%%%%%%%%%%%%%%%%%%%%%%%%%%%%%%%%%%%%%%%%%%%%%%%%%
%%%%%%%%%%%%%%%%%%%%%%%%%%%%%%%%%%%%%%%%%%%%%%%%%%%%%%
\end{document}





