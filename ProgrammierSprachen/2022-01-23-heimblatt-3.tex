\documentclass[a4paper,12pt]{article}
\usepackage{fancyhdr}
\usepackage{fancyheadings}
\usepackage[ngerman,german]{babel}
\usepackage[utf8]{inputenc}
%\usepackage[latin1]{inputenc}
\usepackage[active]{srcltx}
\usepackage{enumerate}
\usepackage{graphicx}
\usepackage{ifthen}
\usepackage{listings}
\usepackage{struktex}
\usepackage{hyperref}
\usepackage{longtable}
\usepackage{gauss}
\usepackage{dsfont}
\usepackage{amssymb}
\usepackage{minted}
\usepackage{stmaryrd}

\newcommand{\x}{\cdot}
\newcommand{\ran}[2]{\langle #1,#2\rangle}
\renewcommand{\Re}{\mathbb{R}}
\makeatletter
\newcommand{\Spvek}[2][r]{%
  \gdef\@VORNE{1}
  \left(\hskip-\arraycolsep%
  \begin{array}{#1}\vekSp@lten{#2}\end{array}%
  \hskip-\arraycolsep\right)}

\def\vekSp@lten#1{\xvekSp@lten#1;vekL@stLine;}
\def\vekL@stLine{vekL@stLine}
\def\xvekSp@lten#1;{\def\temp{#1}%
  \ifx\temp\vekL@stLine
  \else
  \ifnum\@VORNE=1\gdef\@VORNE{0}
  \else\@arraycr\fi%
  #1%
  \expandafter\xvekSp@lten
  \fi}
\makeatother

%%%%%%%%%%%%%%%%%%%%%%%%%%%%%%%%%%%%%%%%%%%%%%%%%%%%%%
%%%%%%%%%%%%%% EDIT THIS PART %%%%%%%%%%%%%%%%%%%%%%%%
%%%%%%%%%%%%%%%%%%%%%%%%%%%%%%%%%%%%%%%%%%%%%%%%%%%%%%
\newcommand{\Fach}{Programmiersprachen}
\newcommand{\Name}{Name:Björn Bick, Max Herting, Miriam Nippel, Eli Kogan-Wang}
\newcommand{\Semester}{WS 21/22}
\newcommand{\Uebungsblatt}{3} %  <-- UPDATE ME
%%%%%%%%%%%%%%%%%%%%%%%%%%%%%%%%%%%%%%%%%%%%%%%%%%%%%%
%%%%%%%%%%%%%%%%%%%%%%%%%%%%%%%%%%%%%%%%%%%%%%%%%%%%%%
\setlength{\parindent}{0em}
\topmargin -1.0cm
\oddsidemargin 0cm
\evensidemargin 0cm
\setlength{\textheight}{9.2in}
\setlength{\textwidth}{6.0in}

%%%%%%%%%%%%%%%
%% Aufgaben-COMMAND
\newcommand{\Aufgabe}[1]{
  {
  \vspace*{0.5cm}
  \textsf{\textbf{Aufgabe #1}}
  \vspace*{0.2cm}

  }
}
%%%%%%%%%%%%%%
\hypersetup{
    pdftitle={\Fach{}: Übungsblatt \Uebungsblatt{}},
    pdfauthor={\Name},
    pdfborder={0 0 0}
}

\lstset{ %
language=java,
basicstyle=\footnotesize\tt,
showtabs=false,
tabsize=2,
captionpos=b,
breaklines=true,
extendedchars=true,
showstringspaces=false,
flexiblecolumns=true,
}



\title{Übungsblatt \Uebungsblatt{}}
\author{\Name{}}

\begin{document}
\pagestyle{fancy}
\lhead{\sf \large \Fach{} \\ \small \Name{}}
\rhead{\sf \Semester{} \\}
\vspace*{0.2cm}
\begin{center}
  \LARGE \sf \textbf{Übungsblatt \Uebungsblatt{}}
\end{center}
\vspace*{0.2cm}

\newcommand{\dd}{\text{d}}
%%%%%%%%%%%%%%%%%%%%%%%%%%%%%%%%%%%%%%%%%%%%%%%%%%%%%%
%% Insert your solutions here %%%%%%%%%%%%%%%%%%%%%%%%
%%%%%%%%%%%%%%%%%%%%%%%%%%%%%%%%%%%%%%%%%%%%%%%%%%%%%%

In diesem Übungsblatt sind die Programmieraufgaben in
``Schreibe dein Programm'' geschrieben.

\Aufgabe{1}

\begin{minted}{racket}
(: pattern-a-n-m (natural natural -> string))

(define pattern-a-n-m
  (lambda (n m)
    (cond ((= n 0) "")
          ((= m 1) (string-append "#" (pattern-a-n-m (- n 1) 1)))
          (else (string-append (pattern-a-n-m n 1) "\n"
                               (pattern-a-n-m n (- m 1)))))))

(: pattern-a-n (natural -> string))

(define pattern-a-n
  (lambda (n) (string-append (pattern-a-n-m n n) "\n")))


(define pattern-a (lambda () (write-string (pattern-a-n 10))))

(write-string "Pattern A:\n")
(pattern-a)

(: repeat-n (natural string -> string))




(define pattern-b (lambda () (write-string (pattern-b-n 10))))

(: pattern-b-n (natural -> string))

(define pattern-b-n
  (lambda (n) (pattern-b-from-n 1 10)))

(define repeat-n
  (lambda (n s)
    (cond ((= n 0) "")
          (else (string-append (repeat-n (- n 1) s) s)))))

(: pattern-b-from-n (natural natural -> string))

(define pattern-b-from-n
  (lambda (from n)
    (cond ((= from 1) (string-append
                        (repeat-n n "#")
                        "\n"
                        (pattern-b-from-n (+ from 1) n)))
          ((= from n) (string-append (repeat-n n "#") "\n"))
          (else (string-append "#" (repeat-n (- n 2) " ") "#\n"
                                (pattern-b-from-n (+ from 1) n))))))

(write-string "Pattern B:\n")
(pattern-b)


(define pattern-c (lambda () (write-string (pattern-c-n 10))))

(: pattern-c-n (natural -> string))

(define pattern-c-n (lambda (n) (pattern-c-from-n 1 n)))

(: pattern-c-from-n (natural natural -> string))

(define pattern-c-from-n
  (lambda (from n)
    (cond ((= from 1) (string-append
                        (repeat-n n "#")
                        "\n"
                        (pattern-c-from-n (+ from 1) n)))
          ((= from n) (string-append (repeat-n n "#") "\n"))
          (else (string-append
                  "#"
                  (repeat-n (- from 2) " ")
                  "#"
                  (repeat-n
                    (-
                      (- n 2)
                      (+ (- from 2) 1))
                      " ")
                  "#\n"
                  (pattern-c-from-n (+ from 1) n))))))

(write-string "Pattern C:\n")
(pattern-c)
\end{minted}

\Aufgabe{2}

\begin{minted}{racket}
; binomial
(: binomial (natural natural -> natural))

(define binomial
  (lambda (n k)
    (cond ((= k 0) 1)
          ((= k n) 1)
          (else (+ (binomial (- n 1) k) (binomial (- n 1) (- k 1)))))))

; pascal triangle
(: pascal-triangle (natural -> string))
(: pascal-row (natural -> string))
(: pascal-row-helper (natural natural -> string))

(define pascal-triangle
  (lambda (n)
    (cond ((= n 0) (pascal-row 0))
          (else (string-append (pascal-triangle (- n 1)) "\n" (pascal-row n))))))

(define pascal-row-helper ; n = row, k = columns to print left
  (lambda (n k)
    (cond ((= k 0) (number->string (binomial n k)))
          (else
            (string-append
              (pascal-row-helper n (- k 1))
              " "
              (number->string (binomial n k)))))))

(define pascal-row (lambda (n)
  (pascal-row-helper n n)))

; (write-string (pascal-triangle 10))

; read : (-> any)

; read number then print pascal triangle
(define read-pascal-triangle (lambda ()
  (write-string (string-append (pascal-triangle (read)) "\n"))))

(read-pascal-triangle)
\end{minted}

\Aufgabe{3}

\begin{minted}{racket}
; random : (natural -> natural)

(: integrate ((real -> real) natural real real -> real))
; takes function to integrate, number of samples, lower bound, upper bound

(: random-real (real real -> real))

; use random natural smaller than 1000000000

(: random-precision natural)

(define random-precision 1000000000)

(define random-real
  (lambda (lower upper)
    (+
      lower
      (*
        (- upper lower)
        (/
          (random random-precision)
          random-precision)))))

(define integrate
  (lambda (f n a b)
    (cond
      ((= n 1) (f (random-real a b)))
      (else
        (/
          (+
            (* (integrate f (- n 1) a b) (- n 1))
            (f (random-real a b)))
          n)))))

; integral from 0 to 1 of \frac{4}{1+x^2}

(write-string
  (number->string
    (exact->inexact
      (integrate
        (lambda (x)
          (/ 4 (+ 1 (* x x))))
          50
          0
          1))))
\end{minted}

\Aufgabe{4}
\begin{enumerate}[a)]
  \item Seiteneffekt sind nicht wünschenswert, weil Seiteneffekte
        die Analyse von Programmabläufen erschweren.
        Die Seiteneffekte verändern den globalen Zustand des Programms,
        sodass das Verhalten der Programmabläufe nicht unbedingt
        gleich dem vorherigen Verhalten ist.
  \item Es ist nicht praktikabel ganz auf Seiteneffekte zu verzichten,
        weil ein Programm für viele Anwendungsfälle Seiteneffekte
        ausführen muss. Will man beispielsweise das Ergebnis einer
        Berechnung ausgeben, so verwendet man Seiteneffekte um den
        Zustand des Ausgabegeräts zu verändern.
  \item Der Zusammenhang zwischen globalen Variablen und Seiteneffekten
        ist der, dass sowohl das Verändern als auch Einlesen von globalen
        Variablen einen Seiteneffekt entspricht. Durch die Verwendung von
        globalen Variablen kann das Verhalten des Programmes von anderen
        Programmteilen beeinflusst werden.
\end{enumerate}

\Aufgabe{5}
% John schreibt ein Programm, das Logmeldungen in eine Datei schreibt. Allerdings wird eine beim Start
% des Programms bereits existierende Logdatei überschrieben. Beschreiben Sie dieses Szenario mit den in
% der Vorlesung besprochenen Ausprägungen des Fehlerbegriffs. (3 P.)

Beim Start des Programms wird die bereits existierende Logdatei überschrieben.
Dies kann als systematischer Design-Fehler interpretiert werden, da durch
das Verhalten, dass die Logdatei überschrieben wird Informationen über den
vorherigen Programmablauf verloren gehen.

Üblicherweise schreibt man auftretende Fehler und Exceptions in eine Logdatei,
sodass in diesem Fall die Fehler und Exceptions des vorherigen Programmablaufs
verloren gehen.

Man kann vermuten, dass der Programmentwickler nicht erwartet hat, dass eine Logdatei
schon vorhanden ist. Das Überschreiben der vorherigen Logdatei ist in diesem Fall
unerwartetes Verhalten und beschreibt einen Fehlerablauf mit abschließendem Fehlerzustand.

%%%%%%%%%%%%%%%%%%%%%%%%%%%%%%%%%%%%%%%%%%%%%%%%%%%%%%
%%%%%%%%%%%%%%%%%%%%%%%%%%%%%%%%%%%%%%%%%%%%%%%%%%%%%%
\end{document}





