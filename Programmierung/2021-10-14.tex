\documentclass[a4paper,12pt]{article}
\usepackage[utf8]{inputenc}
\usepackage[ngerman]{babel}
\usepackage[top=1in, bottom=1.25in, left=1.25in, right=1.25in]{geometry}
\usepackage{minted}
\usepackage{blindtext}
\usepackage{fancyhdr}
\usepackage{titling}
\usepackage{amssymb}
\usepackage{mathtools}
\usepackage{hyperref}


\renewcommand{\footrulewidth}{0.4pt}

\setlength\headheight{15pt}
\setlength{\parskip}{1em}

\title{Programmierung Notizen}
\author{Eli Kogan-Wang}
\date{\today}

\pagestyle{fancy}
\fancyhf{}
\lhead{\thetitle}
\rhead{\thedate}
\lfoot{\theauthor}
\rfoot{Page \thepage}


\begin{document}
% \maketitle
% \thispagestyle{fancy}
\renewcommand{\abstractname}{Abstract}
% show link to slides in footnotes in footnote in abstract: https://prog.cs.uni-paderborn.de/jupyter/user/elikoga/files/lecture/Skript/book.pdf#chapter.1

\begin{abstract}
  This is the notes in programming for today.
\end{abstract}




\section*{Was ist Programmierung?}

Heute ist werden viele Begriffe eingeführt.
Die Folien befinden sich hauptsächlich in Kapitel 1 vom Skript.
\footnote{Die Folien waren grob hier: \url{https://prog.cs.uni-paderborn.de/jupyter/user/elikoga/files/lecture/Skript/book.pdf\#chapter.1}}




\end{document}
